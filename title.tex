%%%%%%%%%%%%%%%%%%%%%%%%%%%%%%%% TITLE PAGE %%%%%%%%%%%%%%%%%%%%%%%%%%%%%%%%%%%%%%%%%%%%%
\begin{titlepage}
\singlespacing
\title{Surface-wave analysis and its application on regional arrays}
\author{Ge Jin\\
\vspace{272pt}\\
Submitted in partial fulfillment \\
of the requirements
for the degree of \\
Doctor of Philosophy \\
in the Graduate School of Arts and Sciences \\
\vspace{24pt}\\
COLUMBIA UNIVERSITY \\}
\maketitle
\end{titlepage}

%%%%%%%%%%%%%%%%%%%%%%%%%%%%%%%% COPYRIGHT %%%%%%%%%%%%%%%%%%%%%%%%%%%%%%%%%%%%%%%%%%%%%
\pagenumbering{gobble}
\vspace{10cm}
\singlespacing
\null
\begin{center}
\begin{bottompar}
{\large \copyright  \hspace{1mm} 2015 \\
Ge Jin\\
All rights reserved}
\end{bottompar}

%%%%%%%%%%%%%%%%%%%%%%%%%%%%%%%% THESIS ABSTRACT %%%%%%%%%%%%%%%%%%%%%%%%%%%%%%%%%%%%%%%%%%%%%
\doublespacing
\large
{\bf ABSTRACT}\\
Surface-wave analysis and its application on regional arrays

Ge Jin
\end{center}
\doublespacing

We develop several new techniques to better retrieve earth's structure by analyzing surface waves. These techniques are applied in the regional studies to understand the tectonic structures and geodynamic processes.

We create an automated method to retrieve surface-wave phase velocities using dense seismic arrays. The method is based on the notion of using cross-correlation to measure phase variations between nearby stations. Frequency-dependent apparent phase velocities are inverted from the phase-variation measurements via the Eikonal equation. The multi-pathing interference is corrected using amplitude measurements via the Helmholtz equation. The coherence between nearby-station waveforms, together with other data-selection criteria, helps to automate the entire process. We build up the Automated Surface-Wave Measuring System (ASWMS) that retrieves structural phase velocity directly from raw seismic waveforms for individual earthquakes without human intervention. 
This system is applied on the broad-band seismic data recorded by the USArray from 2006-2014, and obtain Rayleigh-wave phase-velocity maps at the periods of 20-100~s. The high-frequency maps correlate well with the ambient-noise results. At all frequencies, a significant contrast in Rayleigh-wave phase velocity between the tectonically active western US and the stable eastern US can be observed, with the phase-velocity variations in the western US being 1-2 times greater.
The Love wave phase-velocity maps are also calculated. We find that overtone interference may produce systematic bias for the Love-wave phase-velocity measurements.

We apply surface-wave analysis on the data collected by a PASSCAL array near the D'Entrecasteaux Island (DI), Papua New Guinea. The array comprises 31 inland and 8 off-shore broad-band seismic sensors, and were operated from March 2010 to July 2011. 
We adopt the ASWMS to retrieve phase velocities from earthquake signals, and apply the ambient-noise analysis to obtain the Rayleigh-wave phase velocities at high frequencies. The multi-band phase velocities are inverted for a three-dimensional shear-velocity model of the crust and the upper mantle. The result reveals a localized lithosphere extension along the rift axis beneath the DI, with a shear-velocity structure similar to an adiabatic-upwelling mantle. West of the DI, slow shear-velocity anomaly can be observed at shallow mantle depth (30-60~km), which we interpret either as the present of extra partial melt due to the slow melt extraction, or as the existence of un-exhumed felsic crustal material mixing with the surrounding mantle.

Love waves contain important information to constrain the upper-mantle radial anisotropy. However, Love-wave fundamental-mode phase-velocity measurements are usually contaminated by overtone interference.
We simulate the interference pattern using two plane waves traveling together at different phase velocities. The results indicate large phase variance due to the interference and explain the systemic bias observed in the real data. 
A correction scheme is developed that utilizes amplitude measurements to correct for the interference effect. The synthetic tests show the correction can significantly reduce the phase-velocity variance and the bias generated by the interference.

\raggedbottom
\pagebreak

%%%%%%%%%%%%%%%%%%%%%%%%%%%%%%%% TABLE OF CONTENTS %%%%%%%%%%%%%%%%%%%%%%%%%%%%%%%%%%%%%%%%%%%%%
\frontmatter

\pagestyle{fancy}
\pagenumbering{roman}
\chead{}
\lhead{}
\rhead{}
\cfoot{\thepage}
\renewcommand{\headrulewidth}{0pt}

\tableofcontents
\addtocontents{toc}{~\hfill\textbf{Page}\par}
\raggedbottom
\pagebreak

%%%%%%%%%%%%%%%%%%%%%%%%%%%%%%%% LIST OF FIGURES %%%%%%%%%%%%%%%%%%%%%%%%%%%%%%%%%%%%%%%%%%%%%
\listoffigures
\addcontentsline{toc}{chapter}{List of Figures}
\addtocontents{lof}{~\hfill\textbf{Page}\par}
\raggedbottom
\pagebreak

%%%%%%%%%%%%%%%%%%%%%%%%%%%%%%%% LIST OF TABLES %%%%%%%%%%%%%%%%%%%%%%%%%%%%%%%%%%%%%%%%%%%%%
\listoftables
\addcontentsline{toc}{chapter}{List of Tables}
\addtocontents{lot}{~\hfill\textbf{Page}\par}
\raggedbottom
\pagebreak

%%%%%%%%%%%%%%%%%%%%%%%%%%%%%%%% ACKNOWLEDGMENTS %%%%%%%%%%%%%%%%%%%%%%%%%%%%%%%%%%%%%%%%%%%%%
\begin{center}
{\bf ACKNOWLEDGMENTS }
\end{center}
\addcontentsline{toc}{chapter}{Acknowledgments}
{\fontsize{11}{14}
\selectfont

\pagebreak

%%%%%%%%%%%%%%%%%%%%%%%%%%%%%%%% DEDICATION %%%%%%%%%%%%%%%%%%%%%%%%%%%%%%%%%%%%%%%%%%%%%
\vspace*{\fill}
\begin{center}
\begingroup

\large
\begin{CJK*}{UTF8}{bkai}
	吾生也有涯,而知也無涯。\\
	以有涯隨無涯,智之始。\\
\end{CJK*}
My life is limited, while knowledge has no limit.\\
To use my limited life exploring the unlimited, \\
that's the beginning of my wisdom.

\endgroup
\end{center}
\vspace*{\fill}
\addcontentsline{toc}{chapter}{Dedication}

\raggedbottom
\pagebreak

