%%%%%%%%%%%%%%%%%%%%%%%%%%%%%%%% TITLE PAGE %%%%%%%%%%%%%%%%%%%%%%%%%%%%%%%%%%%%%%%%%%%%%
\begin{titlepage}
\singlespacing
\title{Surface-wave analysis and its application on regional arrays}
\author{Ge Jin\\
\vspace{272pt}\\
Submitted in partial fulfillment \\
of the requirements
for the degree of \\
Doctor of Philosophy \\
in the Graduate School of Arts and Sciences \\
\vspace{24pt}\\
COLUMBIA UNIVERSITY \\}
\maketitle
\end{titlepage}

%%%%%%%%%%%%%%%%%%%%%%%%%%%%%%%% COPYRIGHT %%%%%%%%%%%%%%%%%%%%%%%%%%%%%%%%%%%%%%%%%%%%%
\pagenumbering{gobble}
\vspace{10cm}
\singlespacing
\null
\begin{center}
\begin{bottompar}
{\large \copyright  \hspace{1mm} 2015 \\
Ge Jin\\
All rights reserved}
\end{bottompar}

%%%%%%%%%%%%%%%%%%%%%%%%%%%%%%%% THESIS ABSTRACT %%%%%%%%%%%%%%%%%%%%%%%%%%%%%%%%%%%%%%%%%%%%%
\doublespacing
\large
{\bf ABSTRACT}\\
Surface-wave analysis and its application on regional arrays

Ge Jin
\end{center}
\doublespacing

We develop several new techniques to better retrieve earth's structure by analysing surface waves. These techniques are applied in the regional studies to understand the regional tectonic structure and geodynamic processes.

We create an automated method to retrieve surface-wave phase velocity using dense seismic arrays. The method is based on the notion of using cross-correlation to measure phase variations between nearby stations. Frequency-dependent apparent phase velocity are inverted from the phase-variation measurements via Eikonal equation. And the multi-pathing interference effect are corrected from amplitude measurements via Helmholtz equation. The coherence between the nearby-station waveforms, together with other data-selection criteria, help to automate the entire process. We build up the Automated Surface-wave Measuring System (ASWMS) that can retrieve structural phase-velocity directly from raw seismic waveforms for individual earthquakes without human intervention. 
We apply this system to the board-band seismic data recorded on the USArray from 2006-2014, and retrieve Rayleigh-wave phase-velocity maps between at the periods of 20-100~s. The high-frequency maps correlate well with the ambient-noise results. At all frequencies, a significant contrast in Rayleigh-wave phase velocity between the tectonically active western US and the stable eastern US can be observed, with the phase velocity variations in the western US being 1-2 times greater.

We apply surface-wave analysis on the data collected by a PASSCAL array near D'En

The Love wave phase-velocity maps are also calculated. We find that overtone interference may produce systemic bias for the Love-wave phase-velocity measurements.

\raggedbottom
\pagebreak

%%%%%%%%%%%%%%%%%%%%%%%%%%%%%%%% TABLE OF CONTENTS %%%%%%%%%%%%%%%%%%%%%%%%%%%%%%%%%%%%%%%%%%%%%
\frontmatter

\pagestyle{fancy}
\pagenumbering{roman}
\chead{}
\lhead{}
\rhead{}
\cfoot{\thepage}
\renewcommand{\headrulewidth}{0pt}

\tableofcontents
\addtocontents{toc}{~\hfill\textbf{Page}\par}
\raggedbottom
\pagebreak

%%%%%%%%%%%%%%%%%%%%%%%%%%%%%%%% LIST OF FIGURES %%%%%%%%%%%%%%%%%%%%%%%%%%%%%%%%%%%%%%%%%%%%%
\listoffigures
\addcontentsline{toc}{chapter}{List of Figures}
\addtocontents{lof}{~\hfill\textbf{Page}\par}
\raggedbottom
\pagebreak

%%%%%%%%%%%%%%%%%%%%%%%%%%%%%%%% LIST OF TABLES %%%%%%%%%%%%%%%%%%%%%%%%%%%%%%%%%%%%%%%%%%%%%
\listoftables
\addcontentsline{toc}{chapter}{List of Tables}
\addtocontents{lot}{~\hfill\textbf{Page}\par}
\raggedbottom
\pagebreak

%%%%%%%%%%%%%%%%%%%%%%%%%%%%%%%% ACKNOWLEDGMENTS %%%%%%%%%%%%%%%%%%%%%%%%%%%%%%%%%%%%%%%%%%%%%
\begin{center}
{\bf ACKNOWLEDGMENTS }
\end{center}
\addcontentsline{toc}{chapter}{Acknowledgments}
{\fontsize{11}{14}
\selectfont

\pagebreak

%%%%%%%%%%%%%%%%%%%%%%%%%%%%%%%% DEDICATION %%%%%%%%%%%%%%%%%%%%%%%%%%%%%%%%%%%%%%%%%%%%%
\vspace*{\fill}
\begin{center}
\begingroup

\large
\begin{CJK*}{UTF8}{bkai}
	吾生也有涯,而知也無涯。\\
	以有涯隨無涯,智之始。\\
\end{CJK*}
My life is limited, while knowledge has no limit.\\
To use my limited life exploring the unlimited, \\
that's the beginning of my wisdom.

\endgroup
\end{center}
\vspace*{\fill}
\addcontentsline{toc}{chapter}{Dedication}

\raggedbottom
\pagebreak

