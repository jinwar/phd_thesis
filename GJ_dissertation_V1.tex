\documentclass[12pt,oneside]{book}

\usepackage[margin=1in]{geometry}    % See geometry.pdf to learn the layout options. There are lots.
\geometry{letterpaper}               % ... or a4paper or a5paper or ... 
%\geometry{landscape}                % Activate for for rotated page geometry
%\usepackage[parfill]{parskip}       % Activate to begin paragraphs with an empty line rather than an indent
\usepackage{graphicx}
\usepackage{amsmath}
\usepackage{amsfonts}
\usepackage{amssymb}
\usepackage{epstopdf}
\usepackage{lipsum}
\usepackage{wasysym}
\newenvironment{bottompar}{\par\vspace*{\fill}}{\clearpage}
\usepackage{textcomp}
\usepackage{multirow}
\usepackage{lscape}
\usepackage{natbib}
\setlength{\bibsep}{12pt}
\usepackage{CJK}
\usepackage{tipa}
\usepackage{booktabs}
\usepackage{tocloft}
\setcounter{tocdepth}{2}		  %level to include in Contents: -1: part, 0: chapter, 1: section, etc.
\usepackage{longtable}
\usepackage{setspace}
\usepackage[mathlines]{lineno}
\usepackage{url}
\usepackage{ccaption}
\doublespacing
\newcommand{\degree}[1]{\mbox{$#1^{\circ}$}}
\renewcommand{\bibname}{References}

\usepackage{fancyhdr}
\usepackage{appendix}

\renewcommand\cftchapaftersnum{.}
\renewcommand\cftsecaftersnum{.}

%\renewcommand\thechapter{\arabic{chapter}}
%\renewcommand\thesection{\arabic{section}}
%\renewcommand\thesubsection{\arabic{section}.\arabic{subsection}}


%\DeclareGraphicsRule{.tif}{png}{.png}{`convert #1 `dirname #1`/`basename #1 .tif`.png}


\date{2014}                                           % Activate to display a given date or no date



\begin{document}

%%%%%%%%%%%%%%%%%%%%%%%%%%%%%%%% TITLE PAGE %%%%%%%%%%%%%%%%%%%%%%%%%%%%%%%%%%%%%%%%%%%%%
\begin{titlepage}
\singlespacing
\title{Surface-wave analysis and its application on regional arrays}
\author{Ge Jin\\
\vspace{272pt}\\
Submitted in partial fulfillment \\
of the requirements
for the degree of \\
Doctor of Philosophy \\
in the Graduate School of Arts and Sciences \\
\vspace{24pt}\\
COLUMBIA UNIVERSITY \\}
\maketitle
\end{titlepage}

%%%%%%%%%%%%%%%%%%%%%%%%%%%%%%%% COPYRIGHT %%%%%%%%%%%%%%%%%%%%%%%%%%%%%%%%%%%%%%%%%%%%%
\pagenumbering{gobble}
\vspace{10cm}
\singlespacing
\null
\begin{center}
\begin{bottompar}
{\large \copyright  \hspace{1mm} 2015 \\
Ge Jin\\
All rights reserved}
\end{bottompar}

%%%%%%%%%%%%%%%%%%%%%%%%%%%%%%%% THESIS ABSTRACT %%%%%%%%%%%%%%%%%%%%%%%%%%%%%%%%%%%%%%%%%%%%%
\doublespacing
\large
{\bf ABSTRACT}\\
Surface-wave analysis and its application on regional arrays

Ge Jin
\end{center}
\doublespacing

We develop several new techniques to better retrieve earth's structure by analyzing surface waves. These techniques are applied in the regional studies to understand the tectonic structures and geodynamic processes.

We create an automated method to retrieve surface-wave phase velocities using dense seismic arrays. The method is based on the notion of using cross-correlation to measure phase variations between nearby stations. Frequency-dependent apparent phase velocities are inverted from the phase-variation measurements via the Eikonal equation. The multi-pathing interference is corrected using amplitude measurements via the Helmholtz equation. The coherence between nearby-station waveforms, together with other data-selection criteria, helps to automate the entire process. We build up the Automated Surface-Wave Measuring System (ASWMS) that retrieves structural phase velocity directly from raw seismic waveforms for individual earthquakes without human intervention. 
This system is applied on the broad-band seismic data recorded by the USArray from 2006-2014, and obtain Rayleigh-wave phase-velocity maps at the periods of 20-100~s. The high-frequency maps correlate well with the ambient-noise results. At all frequencies, a significant contrast in Rayleigh-wave phase velocity between the tectonically active western US and the stable eastern US can be observed, with the phase-velocity variations in the western US being 1-2 times greater.
The Love wave phase-velocity maps are also calculated. We find that overtone interference may produce systematic bias for the Love-wave phase-velocity measurements.

We apply surface-wave analysis on the data collected by a PASSCAL array near the D'Entrecasteaux Island (DI), Papua New Guinea. The array comprises 31 inland and 8 off-shore broad-band seismic sensors, and were operated from March 2010 to July 2011. 
We adopt the ASWMS to retrieve phase velocities from earthquake signals, and apply the ambient-noise analysis to obtain the Rayleigh-wave phase velocities at high frequencies. The multi-band phase velocities are inverted for a three-dimensional shear-velocity model of the crust and the upper mantle. The result reveals a localized lithosphere extension along the rift axis beneath the DI, with a shear-velocity structure similar to an adiabatic-upwelling mantle. West of the DI, slow shear-velocity anomaly can be observed at shallow mantle depth (30-60~km), which we interpret either as the present of extra partial melt due to the slow melt extraction, or as the existence of un-exhumed felsic crustal material mixing with the surrounding mantle.

Love waves contain important information to constrain the upper-mantle radial anisotropy. However, Love-wave fundamental-mode phase-velocity measurements are usually contaminated by overtone interference.
We simulate the interference pattern using two plane waves traveling together at different phase velocities. The results indicate large phase variance due to the interference and explain the systemic bias observed in the real data. 
A correction scheme is developed that utilizes amplitude measurements to correct for the interference effect. The synthetic tests show the correction can significantly reduce the phase-velocity variance and the bias generated by the interference.

\raggedbottom
\pagebreak

%%%%%%%%%%%%%%%%%%%%%%%%%%%%%%%% TABLE OF CONTENTS %%%%%%%%%%%%%%%%%%%%%%%%%%%%%%%%%%%%%%%%%%%%%
\frontmatter

\pagestyle{fancy}
\pagenumbering{roman}
\chead{}
\lhead{}
\rhead{}
\cfoot{\thepage}
\renewcommand{\headrulewidth}{0pt}

\tableofcontents
\addtocontents{toc}{~\hfill\textbf{Page}\par}
\raggedbottom
\pagebreak

%%%%%%%%%%%%%%%%%%%%%%%%%%%%%%%% LIST OF FIGURES %%%%%%%%%%%%%%%%%%%%%%%%%%%%%%%%%%%%%%%%%%%%%
\listoffigures
\addcontentsline{toc}{chapter}{List of Figures}
\addtocontents{lof}{~\hfill\textbf{Page}\par}
\raggedbottom
\pagebreak

%%%%%%%%%%%%%%%%%%%%%%%%%%%%%%%% LIST OF TABLES %%%%%%%%%%%%%%%%%%%%%%%%%%%%%%%%%%%%%%%%%%%%%
\listoftables
\addcontentsline{toc}{chapter}{List of Tables}
\addtocontents{lot}{~\hfill\textbf{Page}\par}
\raggedbottom
\pagebreak

%%%%%%%%%%%%%%%%%%%%%%%%%%%%%%%% ACKNOWLEDGMENTS %%%%%%%%%%%%%%%%%%%%%%%%%%%%%%%%%%%%%%%%%%%%%
\begin{center}
{\bf ACKNOWLEDGMENTS }
\end{center}
\addcontentsline{toc}{chapter}{Acknowledgments}
{\fontsize{11}{14}
\selectfont

\pagebreak

%%%%%%%%%%%%%%%%%%%%%%%%%%%%%%%% DEDICATION %%%%%%%%%%%%%%%%%%%%%%%%%%%%%%%%%%%%%%%%%%%%%
\vspace*{\fill}
\begin{center}
\begingroup

\large
\begin{CJK*}{UTF8}{bkai}
	吾生也有涯,而知也無涯。\\
	以有涯隨無涯,智之始。\\
\end{CJK*}
My life is limited, while knowledge has no limit.\\
To use my limited life exploring the unlimited, \\
that's the beginning of my wisdom.

\endgroup
\end{center}
\vspace*{\fill}
\addcontentsline{toc}{chapter}{Dedication}

\raggedbottom
\pagebreak



%%%%%%%%%%%%%%%%%%%%%%%%%%%%%%%% BEGIN CHAPTERS %%%%%%%%%%%%%%%%%%%%%%%%%%%%%%%%%%%%%%%%%%
\mainmatter

%%%%%%%%%%%%%%%%%%%%%%%%%%%%%%%% CHAPTER 1/INTRODUCTION %%%%%%%%%%%%%%%%%%%%%%%%%%%%%%%%%%%%%%%%%%

\singlespacing
\chapter{Introduction}
\label{ch:intro}
\doublespacing

\pagestyle{fancy}
\pagenumbering{arabic}
%\fancyhead[LE,RO]{\thepage} %puts page number on top left for even pages, top right for odd pages
\chead{}
%\lhead{}
%\rhead{\thepage}
\cfoot{\thepage}
\renewcommand{\headrulewidth}{0pt} %controls thickness of line underneath header- same for footrulewidth
\thispagestyle{fancy}

Earthquake is like lightning, each strike lights up the darkness of the earth's interior. For centuries scientists have studied the propagation of the seismic waves, which is the elastic energy released by earthquakes, and used them to explore the deep structures and the dynamic processes of the planet that we all live on. 

Among the different types of seismic waves, surface waves draw the attention of many seismologists as the late-arriving phases which usually have the largest amplitude. They propagate along the shallow part of the earth, and carry the information of the heterogeneities in crust and upper mantle.
Surface waves are dispersed, since different frequencies are sensitive to  structures at different depths, which makes it more challenging to quantitatively describe their propagation. 
They can be measured by different means. One of the most common methods is to estimate phase velocity, the speed at which the phase of each individual frequency varies along the path. 
The frequency-dependent phase velocities can then be converted into the shear velocities at depth through a non-linear inversion.

Based on the particle motion direction, surface waves can be divided into two categories: Rayleigh waves and Love waves. Although they sample the similar part of the earth, the structures they each suggests sometimes are not consistent. This so-called Love-Rayleigh discrepancy is the most important constrain on the earth's radial anisotropy, which is related to mantle flow pattern, melt geometry, crystal alignment, etc \citep[e.g.][]{Gaherty:2004is,Gaherty:2007cc,Holtzman:2010fta}.

The estimation of surface-wave phase velocity depends on single-station phase measurements. Several techniques have been developed to estimate surface-wave frequency-dependent phase, from utilizing basic Fourier transform \citep{Forsyth:2005id}, to applying multi-channel narrow-band filters \citep[e.g.][]{Levshin:1992ve}, to cross-correlating the waveform with either synthetic waveforms \citep[e.g.][]{Gee:1992ww}, or the waveform from another station on the same great-circle path \citep[e.g.][]{Landisman:1969gt}.

The phase variations at stations can then be inverted for two-dimensional phase-velocity maps. At this stage, several methods are developed under different assumptions. The first category of methods is based on the straight-ray theory, in which the phase observations are represented as integral of phase slowness along the source-receiver path \citep[e.g.][]{Nettles:2008ha}, or between the two stations on the same great-circle path \citep[e.g.][]{Yao:2005ha,Foster:2014kna}. The second category corrects for the ray-bending and finite-frequency effect which can be remarkable for surface waves \citep[e.g.][]{Evernden:1954ui,Zhou:2005fk}. Methods in this catalog include the two-station method with arrival-angle correction \citep{Foster:2014kna}, Eikonal tomography \citep{Lin:2009fx}, and finite-frequency kernel estimation \citep{Zhou:2006gna}. The final catalog of phase-velocity inversion methods utilizing the amplitude measurements to reduce the multi-pathing interference, which includes the two-plane-wave method \citep{Forsyth:2005id} and the recently developed Helmholtz tomography \citep{Lin:2011fw}.

The deployments of large and dense arrays become more popular  nowadays in seismic experiments. Substantial amount of data are collected in real time and open to the public. 
Among the experiments, the Transportable Array (TA) under the USArray program stands out by its data volume and coverage area. The TA project includes more than 400 board-band seismic sensors deployed with $\sim$70-km grid spacing, spanning from the west coast to the east to cover the entire US continent. The communication system installed at the sites allows for real-time data transportation and publication. Such a large and growing dataset is ideal for surface-wave analysis, however an automated system is required to accommodate the speed of data growth.

We develop a new technique to meet the challenge in Chapter~\ref{ch:aswms}. The Automated Surface-Wave phase-velocity Measuring System (ASWMS) is based on the cross-correlation technique to measure frequency-dependent phase delays between all possible nearby station pairs. 
Coherence between the waveforms is used to identify low-quality measurements.  
The measured phase delays, together with single-station amplitude measurements, are then inverted for two-dimensional phase-velocity maps at each frequency.

We successfully automate the entire process. By combining the data-fetching interface from IRIS DMC, we develop a program that can operate by itself and update the US Rayleigh-wave phase-velocity results on a weekly basis. The program is adopted by IRIS DMC as a data product. The phase velocity results,  together with the ASWMS, are available for the public to download through IRIS website. 

After the methodology development, we turn our focus into its regional applications. In Chapter~\ref{ch:png}, we apply the surface-wave analysis in the region near the D'Entrecasteaux Islands (DI), Papua New Guinea (PNG).

The DI locate at the tip of the Woodlark rifting system, where the continental rift is actively transforming into sea-floor spreading \citep[e.g.][]{Taylor:1999ur, Ferris:2006tr}. Previous studies show the crust beneath the DI has been thinned by 30-50\%, while the body-wave tomography results suggest a upwelling mantle \citep{Abers:2002uj, Ferris:2006tr}.
The islands expose high-pressure (HP) terranes comprise several gneiss domes with diameters of 20-30km. These gneiss domes are mostly felsic in composition (quartzofeldspathic), and bears the youngest-known ultrahigh-pressure (UHP) coesite eclogite \citep{Baldwin:2004wx, Baldwin:2008gm, Little:2011jy}. The HP/UHP rocks have a composition similar to continental crust. The isotropic analysis reveals a exhumation history at a plate-tectonic rate from $\sim$100-km depth since 5-8 Ma \citep[e.g.][]{Baldwin:2008gm, Gordon:2012hu}.

The exhumation mechanism of the UHP rocks remains debated, and the two existing competing models are: subduction reversal that extracts UHP rocks along the paleo-subduction channel as a low-angle unroofing process \citep{Hill:1992jd, Webb:2008fc}, and thinning of overlying crust to allow penetration of buoyant continental rocks to the surface as diapirs \citep{Ellis:2011jh, Little:2011jy}.

In order to provide better constrains on the geodynamic process in this area, we deployed 31 land-based and 8 ocean bottom broad-band seismometers, with $\sim$20-km station spacing, to cover a $2.5^\circ \times 2.5^\circ$ area around the DI. The stations collected seventeen months of data, on which we perform surface-wave analysis to retrieve a three-dimensional shear-velocity model of the region.

In order to better constrain the shallow structure, we apply the ambient-noise technique to obtain phase velocities at higher frequencies.
The idea of applying cross correlation on ambient noise to recover the green function between two simultaneously-recording stations can be traced back as early as \citet{Aki:1957un}. 
It was not until fifty years later that the method became popular because of the availability of dense seismic networks \citep[e.g.][]{Shapiro:2005kz,Bensen:2007hl}. 
In the PNG case, we perform the ambient-noise measurements in the frequency domain, following the algorithm developed \citet{Ekstrom:2009iv} and \citet{Ekstrom:2013dr}. 
A new Bessel-function waveform fitting technique is developed to better estimate the phase delays between the stations. The phase delays are later inverted for phase-velocity maps using a ray-theory tomography method.

The ambient-noise analysis produces high resolution phase-velocity maps at the 10-17~s periods. Together with the earthquake measurements from the ASWMS, we retrieve the robust phase velocities of Rayleigh waves in a wide frequency range (10-60 s in period), which provide good constrains on the crustal and upper mantle structures.

For each grid on the map, the phase-velocity dispersion curve is then extracted to invert for the shear-velocity structure independently. This inversion is non-linear and non-unique \citep{Herrmann:2004wl}, and the output is largely dependent on the initial model \citep[e.g.][]{Foti:2009uo}. In order to produce reliable shear-velocity results, we first make effort to build up good initial models. 
The receiver-function result \citep{Abers:2012ul} is adopted to constrain the crustal thickness. We apply a grid-search over various parameters to find the simple three-layer model that best fits the observation. 
Then the model is randomly perturbed for 100 times, each serves as an initial model to invert for shear-velocity model. 
The inverted models are then averaged to obtain the final model, and the shear-velocity variance due to the initial-model dependence can be estimated. 
The shear-velocity results show a slow anomaly in the shallow mantle west of Goodenough island, suggesting the existence of un-exhumed continental crust material and/or partial melt.

Although we successfully apply our methods on Rayleigh waves, systematic bias is found in the phase-velocity measurements when  the same techniques are applied on Love waves. 
The bias is mainly generated by overtone interference, which needs to be handled with special care for the Love-wave single-mode measurements \citep[e.g.][]{Thatcher:1969hg,Forsyth:1975tp,Gaherty:1996uf}.

The overtone interference is more acute for Love waves due to the similar group velocities of the fundamental- and higher-mode Love waves, especially for oceanic structures \citep{Nettles:2011bb}. It can produce large phase and amplitude variations for single-branch measurements.  
Such variations do not significantly affect long-path measurements, since the interference pattern is periodical and the bias can be averaged out over the path \citep{Nettles:2011bb}. But they may introduce large bias in the array-based regional studies which involve the estimation of local phase-gradient. In the USArray application, \citet{Foster:2014kr} show that the bias due to the interference is profound at middle-band frequencies (60-150~s), even the array locates in the center of the continent and only shallow events are selected. 

Based on the global and the regional observations, \citet{Nettles:2011bb} and \citet{Foster:2014kr} both suggest the bias is mainly controlled by the interference between the Love-wave fundamental mode and its first overtone.

In Chapter~\ref{ch:overtone} we construct a series of synthetic tests to simulate the pattern of overtone interference and its influence on the single-branch phase-velocity measurements. A correction technique that has the potential to correct this bias is developed and tested on the synthetic data.

The target of this dissertation is to better retrieve earth's structure from surface waves, so we can provide more constrains on geodynamic processes. 
We develop robust and automatic phase-velocity measuring methods for both earthquake and ambient\nobreakdash-noise based data. These techniques are successfully applied on the data from USArray and a PASSCAL temporary array at the D'Entrecasteaux Islands, Papua New Guinea. 
Based on the measurement results, we shed light on the effect of thermal and compositional variations on the upper-mantle shear velocities, and the geodynamic process of continental rift extension and UHP rocks exhumation.
We also propose a correction scheme that can reduce the phase-velocity bias due to overtone interference in Love-wave studies, so the mantle radial anisotropy can be better constrained.

\raggedbottom
\pagebreak

%%%%%%%%%%%%%%%%%%%%%%%%%%%%%%%%%%%%%%%%%%%%%%%%%%%%%%%%%%%%%%%%%%%%%%%%%%%%%%%%%%%%%
%%%%%%%%%%%%%%%%%%%%%%%%%%%%%%%%% CHAPTER 2/FIRST PAPER %%%%%%%%%%%%%%%%%%%%%%%%%%%%%%%%%%%%%%%%%%%%%
%\renewcommand\thechapter{\arabic{chapter}}
\singlespacing
\chapter[Surface-Wave Measurement Based on Cross Correlation]{Surface-Wave Phase-Velocity Measurement Based on Multi-Channel Cross Correlation}
\label{ch:aswms}
\doublespacing

\thispagestyle{fancy}

%\footnotesize
\begin{raggedright}
{\bf Note: } A slightly modified version of this chapter has been submitted to Geophysical Journal International (2014).
\footnote{AUTHORS:  Ge Jin$^a$*,  James Gaherty$^a$\\
$^a$ Lamont-Doherty Earth Observatory, Columbia University, 61 Route 9W, Palisades, NY 10964, USA\\
* corresponding author: ge.jin@ldeo.columbia.edu}
\end{raggedright}
%\linenumbers
\normalsize

%%%%%%%%%%%%%%%%%%%%%%%%%%%%%%%% Chapter 3 PNG
%%%%%%%%%%%%%%%%%%%%%%%%%%%%%%%%%%%%%%%%%%%%%

\singlespacing
\chapter[Shear velocity structure of the DI, PNG]{Shear Velocity Structure of the D'Entrecasteaux Islands, Papua New Guinea from Rayleigh Wave Tomography}
\label{ch:png}
\doublespacing

%%%%%%%%%%%%%%%%%%%%%%%%%%%%%%%% Chapter 4 overtone %%%%%%%%%%%%%%%%%%%%%%%%%%%%%%%%%%%%%%%%%%%%%
\singlespacing
\chapter[Love-wave Overtone Interference]{Love-Wave Phase-velocity Estimation in the Presence of Multi-mode Interference}
\label{ch:overtone}
\doublespacing

%%%%%%%%%%%%%%%%%%%%%%%%%%%%%%%% Chapter 5 conclusion %%%%%%%%%%%%%%%%%%%%%%%%%%%%%%%%%%%%%%%%%%%%%
\singlespacing
\chapter[Conclusion]{Concluding Remarks}
\label{ch:conclusion}
\doublespacing

In this dissertation, we develop several new surface-wave measuring methods and apply them on different datasets to understand local tectonic structures.

In Chapter~\ref{ch:aswms}, we develop the Automated Surface-Wave Measuring System (ASWMS), which can retrieve surface-wave phase velocities from dense-array data without human intervention. We produce robust Rayleigh-wave phase-velocity maps of the US continent at the periods of 20-100s using the USArray data, and work with IRIS to provide weekly-updated results for general public. The phase-velocity maps clearly outline the major geological structures in the US continent. The spectrum analysis of these maps imply that in the upper mantle, thermal variation affects shear velocity more than compositional variations.

In Chapter~\ref{ch:png}, we apply surface-wave analysis in the region near the D'Entrecasteaux Islands, Papua New Guinea. We retrieve phase-velocity variations using both ambient-noise and earthquake signal. The phase velocities in a board period range (10-60 s) are inverted for the shear-velocity structure in the crust and the upper mantle. The results imply the existence of un-exhumed crustal material and/or partial melt in the shallow mantle west of the most recently exhumed Goodenough Island.

In Chapter~\ref{ch:overtone}, we focus on the effect of Love-wave overtone interference on the array-based phase-velocity measurements. We simulate the interference pattern by two plane waves travelling together at different phase velocities. We also develop a correction scheme that can significantly reduce the phase-velocity bias and variance due to the interference. 

%%%%%%%%%%%%%%%%%%%%%%%%%%%%%%%% REFERENCES %%%%%%%%%%%%%%%%%%%%%%%%%%%%%%%%%%%%%%%%%%%%%
%\backmatter
\cleardoublepage
\normalsize
\singlespacing
\bibliographystyle{elsarticle-harv}
\addcontentsline{toc}{chapter}{References}
\bibliography{thesis_ref}
\clearpage

\end{document}  
