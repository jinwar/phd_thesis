\documentclass[12pt,oneside]{book}
\usepackage[margin=1in]{geometry}    % See geometry.pdf to learn the layout options. There are lots.
\geometry{letterpaper}               % ... or a4paper or a5paper or ... 
%\geometry{landscape}                % Activate for for rotated page geometry
%\usepackage[parfill]{parskip}       % Activate to begin paragraphs with an empty line rather than an indent
\usepackage{graphicx}
\usepackage{amsmath}
\usepackage{amsfonts}
\usepackage{amssymb}
\usepackage{epstopdf}
\usepackage{lipsum}
\usepackage{wasysym}
\newenvironment{bottompar}{\par\vspace*{\fill}}{\clearpage}
\usepackage{textcomp}
\usepackage{multirow}
\usepackage{lscape}
\usepackage{natbib}
\setlength{\bibsep}{12pt}
\usepackage{tipa}
\usepackage{booktabs}
\usepackage{tocloft}
\setcounter{tocdepth}{2}		  %level to include in Contents: -1: part, 0: chapter, 1: section, etc.
\usepackage{longtable}
\usepackage{setspace}
\usepackage[mathlines]{lineno}
\usepackage{url}
\usepackage{ccaption}
\doublespacing
\newcommand{\degree}[1]{\mbox{$#1^{\circ}$}}
\renewcommand{\bibname}{References}

\usepackage{fancyhdr}
\usepackage{appendix}

\renewcommand\cftchapaftersnum{.}
\renewcommand\cftsecaftersnum{.}

%\renewcommand\thechapter{\arabic{chapter}}
%\renewcommand\thesection{\arabic{section}}
%\renewcommand\thesubsection{\arabic{section}.\arabic{subsection}}


%\DeclareGraphicsRule{.tif}{png}{.png}{`convert #1 `dirname #1`/`basename #1 .tif`.png}


\date{2014}                                           % Activate to display a given date or no date

\begin{document}

%%%%%%%%%%%%%%%%%%%%%%%%%%%%%%%% TITLE PAGE %%%%%%%%%%%%%%%%%%%%%%%%%%%%%%%%%%%%%%%%%%%%%
\begin{titlepage}
\singlespacing
\title{Surface-wave propagation and phase-velocity structure from observations on the USArray Transportable Array}
\author{Anna E. Foster \\
\vspace{272pt}\\
Submitted in partial fulfillment \\
of the requirements
for the degree of \\
Doctor of Philosophy \\
in the Graduate School of Arts and Sciences \\
\vspace{24pt}\\
COLUMBIA UNIVERSITY \\}
\maketitle
\end{titlepage}

%%%%%%%%%%%%%%%%%%%%%%%%%%%%%%%% COPYRIGHT %%%%%%%%%%%%%%%%%%%%%%%%%%%%%%%%%%%%%%%%%%%%%
\pagenumbering{gobble}
\vspace{10cm}
\singlespacing
\null
\begin{center}
\begin{bottompar}
{\large \copyright  \hspace{1mm} 2014 \\
Anna E. Foster\\
All rights reserved}
\end{bottompar}

%%%%%%%%%%%%%%%%%%%%%%%%%%%%%%%% THESIS ABSTRACT %%%%%%%%%%%%%%%%%%%%%%%%%%%%%%%%%%%%%%%%%%%%%
\doublespacing
\large
{\bf ABSTRACT}\\
Surface-wave propagation and phase-velocity structure from observations on the USArray Transportable Array\\
Anna E. Foster
\end{center}
\doublespacing

We address questions relating to the velocity structure of the Earth in three ways: mapping the phase-velocity structure of the western United States, examining deviations of wave paths due to lateral variations in velocity, and demonstrating that Love wave fundamental-mode phase measurements from array methods can be significantly contaminated by overtone interference, dependent on differences in fundamental-mode and first-overtone phase-velocity structure. All of the studies presented in this work use USArray Transportable Array data, which allow for dense, high-quality measurements at an unprecedented level. 

To image the uppermost mantle beneath the western US, we improve upon single-station phase measurements by differencing them to produce a baseline data set of phase measurements along inter-station paths, for both Love and Rayleigh waves from 25--100 s. Additional measurements of the arrival angle and local phase velocity are made using a mini-array method similar to beamforming. The arrival-angle measurements are used to correct the two-station baseline measurements and produce a corrected data set. Both the baseline and corrected data sets are separately inverted, producing phase-velocity maps on a \degree{0.5}\nobreakdash-by\nobreakdash-\degree{0.5} grid. We select the corrected maps as the preferred models for Rayleigh waves, with better fits to the data and more consistent measurements. We find that arrival-angle measurements for Love waves may be biased by overtone interference, and hence select the baseline maps as the preferred models for Love waves. The final set of phase-velocity maps is consistent with expectations from known geologic features, and is useful for both calculation of phase for regional paths and studies of radial anisotropy within the region. 

We use the mini-array method to make observations of the deviations of waves from the great-circle path. Measured arrival angles vary from \degree{0} to $\pm\degree{15}$. We compile results from earthquakes in small source regions, allowing the observation of bands of arrival-angle anomalies crossing the footprint of the USArray Transportable array in the propagation direction. These bands of deviations may result from heterogeneous velocity structure within the array, or on the larger source-to-array path. We use two global tomographic models to predict arrival-angle anomaly patterns, with both ray-theory-based prediction methods and measurements on synthetic waveforms calculated using SPECFEM3D Globe, a finite element package. We show that both models predict well the long-wavelength patterns of anomalies observed, but not the short-wavelength variations. Experiments with crustal structure indicate that greater heterogeneity is needed in the models. Predictions from the spectral-element-method synthetic waveforms contain the type of complexity seen in the observed patterns, and not obtained with the ray-theoretical methods, indicating that full synthetics are needed to compare model predictions to observed arrival-angle anomalies. 

We further examine possible overtone interference in the mini-array arrival-angle and local phase-velocity measurements for Love waves at long periods. Love wave fundamental-mode and higher-mode waves at the same period travel with similar group velocity, making them difficult to separate; the waves have different phase velocities, resulting in a beating interference pattern that oscillates with distance. We show this interference pattern for single-station, two-station, and mini-array phase-velocity measurements. Using measurements on synthetic waveforms calculated using both mode summation and SPECFEM3D Globe, we show that contamination of single-station measurements can largely be explained by interference between the fundamental and first-higher mode only. Interference causes small variations in the single-station phase velocity, up to 1\%, and the oscillations about the expected values are asymmetric. The two array-based measurement techniques can be thought of as a spatial gradient over the single-station phase measurements, and consequently much larger variations are observed, 10--20\%, and the results are biased to higher phase velocities. We conclude that overtone contamination must be carefully considered prior to attributing array-based Love wave phase-velocity anomalies to Earth structure.  

\raggedbottom
\pagebreak

%%%%%%%%%%%%%%%%%%%%%%%%%%%%%%%% TABLE OF CONTENTS %%%%%%%%%%%%%%%%%%%%%%%%%%%%%%%%%%%%%%%%%%%%%
\frontmatter

\pagestyle{fancy}
\pagenumbering{roman}
\chead{}
\lhead{}
\rhead{}
\cfoot{\thepage}
\renewcommand{\headrulewidth}{0pt}

\tableofcontents
\addtocontents{toc}{~\hfill\textbf{Page}\par}
\raggedbottom
\pagebreak

%%%%%%%%%%%%%%%%%%%%%%%%%%%%%%%% LIST OF FIGURES %%%%%%%%%%%%%%%%%%%%%%%%%%%%%%%%%%%%%%%%%%%%%
\listoffigures
\addcontentsline{toc}{chapter}{List of Figures}
\addtocontents{lof}{~\hfill\textbf{Page}\par}
\raggedbottom
\pagebreak

%%%%%%%%%%%%%%%%%%%%%%%%%%%%%%%% LIST OF TABLES %%%%%%%%%%%%%%%%%%%%%%%%%%%%%%%%%%%%%%%%%%%%%
\listoftables
\addcontentsline{toc}{chapter}{List of Tables}
\addtocontents{lot}{~\hfill\textbf{Page}\par}
\raggedbottom
\pagebreak

%%%%%%%%%%%%%%%%%%%%%%%%%%%%%%%% ACKNOWLEDGMENTS %%%%%%%%%%%%%%%%%%%%%%%%%%%%%%%%%%%%%%%%%%%%%
\begin{center}
{\bf ACKNOWLEDGMENTS }
\end{center}
\addcontentsline{toc}{chapter}{Acknowledgments}
{\fontsize{11}{14}
\selectfont
I am thankful for my advisors, who are all impressive scientists, teachers, and mentors. In particular, I thank G\"{o}ran Ekstr\"{o}m for volunteering to advise an unguided applicant, and teaching me that research is worth doing right, and writing is better when done with clarity. Your generous support for studies, conferences, and fieldwork was very much appreciated. I thank Meredith Nettles for demonstrating leadership and teaching me the importance of communication, in the classroom and on paper. You seem to fit more hours in the day, and I'm still not sure how. I thank Jim Gaherty for being there when I needed another opinion- I may not have asked as many questions as I should have, but I appreciated knowing I could. Thank you to Spahr Webb and Colleen Dalton for being on my defense committee, and for being supportive throughout my PhD. A million thanks to Vala Hjorleifsd\'{o}ttir for teaching me everything I know about SPECFEM, and for showing me that you can do it all and still laugh at the end of the day. Thanks to Geoff Abers and Doug Christensen for giving me the wonderful opportunity and experience of doing fieldwork in Alaska, and Art Lerner-Lam for doing the same in Bermuda. Thank you to Ben Holtzman, who showed me that art, music, and diverse interests in science can be complementary. 

To my parents, Jane and Jack, and my sisters, Emma and Maggie, thank you for giving me perspective and love no matter what. To Rafael, for inspiring me with your curiosity and giving me encouragement when I needed it most- thank you. To my classmates, especially Kat Allen, Alison Hartman, Elizabeth Pierce, Emmi Yonekura, and Meg Reitz, thank you for making a challenging journey a fun one. To all the past and current students who have helped me in ways big and small- Ashley Shuler, Raj Moulik, Zach Eilon, Jin Ge, Yang Zha, Celia Eddy, Jiyao Li, Alex Lloyd, and many others...thank you all. Thank you to Andrew Goodwillie and the yoga class, for giving me a chance to refocus each week. Thanks to Scott Frank, Jim Lewkowicz, Aaron Ferris, and Shelly Johnson, for getting me into seismology and convincing me to apply to graduate school. Lamont-Doherty Earth Observatory was a wonderfully unique place to do science, and the faculty, researchers, staff, post-docs, and students I met there made it that way. }

\pagebreak

%%%%%%%%%%%%%%%%%%%%%%%%%%%%%%%% DEDICATION %%%%%%%%%%%%%%%%%%%%%%%%%%%%%%%%%%%%%%%%%%%%%
\vspace*{\fill}
\begin{center}
\begingroup
\emph{To all the people in my life \\
who have helped me to learn.\\
}
\endgroup
\end{center}
\vspace*{\fill}
\addcontentsline{toc}{chapter}{Dedication}

\raggedbottom
\pagebreak

%%%%%%%%%%%%%%%%%%%%%%%%%%%%%%%% BEGIN CHAPTERS %%%%%%%%%%%%%%%%%%%%%%%%%%%%%%%%%%%%%%%%%%
\mainmatter

%%%%%%%%%%%%%%%%%%%%%%%%%%%%%%%% CHAPTER 1/INTRODUCTION %%%%%%%%%%%%%%%%%%%%%%%%%%%%%%%%%%%%%%%%%%
\singlespacing
\chapter{Introduction}
\label{ch:intro}
\doublespacing

\pagestyle{fancy}
\pagenumbering{arabic}
%\fancyhead[LE,RO]{\thepage} %puts page number on top left for even pages, top right for odd pages
\chead{}
%\lhead{}
%\rhead{\thepage}
\cfoot{\thepage}
\renewcommand{\headrulewidth}{0pt} %controls thickness of line underneath header- same for footrulewidth
\thispagestyle{fancy}

We live on a dynamic planet, but many of the ongoing processes that cause changes at the surface take place deep below our feet. Within the North American continent, mantle processes related to extension, subduction, and a strike-slip plate boundary \citep{Fordetal2014} are occurring, in addition to suggested lithospheric delamination \citep[e.g.,][]{Jonesetal1994, Zandtetal2004, Levanderetal2011} and a mantle plume \citep[e.g.,][]{ThompsonGibson1991, Parsonsetal1994, Waiteetal2006, Smithetal2009, Fouch2012}. Combined with the geologic history of past processes and a long-lived stable area of the continent, these are manifested in three-dimensional variations in temperature and composition. 

Because we cannot directly observe either the geodynamic processes or the physical variables of temperature and composition at depth, we rely on secondary observables like the velocity at which waves travel through rock. The translation of velocity into temperature and composition is an ongoing quest, primarily based on laboratory experiments \citep[e.g.,][]{Jordan1979, Trampertetal2001, Cammaranoetal2003, Xuetal2008}. However, before we can use these relationships to identify processes at depth, we must first obtain good constraints on the velocity structure. Different types of waves provide information on different areas of the earth, but surface waves provide particularly good constraints on the upper mantle. Such waves carry the integrated signature of the path traveled, and frequency-dependent sensitivity yields depth constraints. Additionally, the differing motions of Love and Rayleigh waves result in differing sensitivities to the vertical and horizontal shear velocity, yielding information on radial anisotropy. 

One of the challenges of observational seismology has always been the availability of data. We generally rely on earthquake sources, which are not distributed regularly in either time or space. Permanent networks such as the Global Seismographic Network are typically sparse. Many temporary arrays are dense but small in area, may not be in place for very long, or may suffer from quality issues due to their instrumentation or installation. This explains why the USArray Transportable Array (TA) project has been such a boon to the seismological community. 

The TA is a temporary installation of seismometers designed to cover the contiguous United States with a regular grid spacing of approximately 70~km. Beginning in 2006, roughly 400 instruments were deployed along the west coast for a 2-yr period. The instruments were then leap-frogged eastward, progressively expanding the covered area. As of 2014, the instruments have reached the east coast and are beginning to be deployed in Alaska. The TA also includes an electromagnetic component, with magnetotelluric instruments included at most stations.

The potential of this data set for imaging continental structure is huge, and has already produced many new results. Ambient noise tomography was first performed with TA data, and has yielded detailed images of the crust \citep[e.g.,][]{Shapiroetal2005, Linetal2008, Ekstrometal2009, Ekstrom2014}. New array-based surface-wave techniques have been developed, including Eikonal tomography \citep{Lin2009}, wave gradiometry \citep{Liang2009} and multiple-plane-wave tomography \citep{Forsyth2005, Yang2006a, Pollitz2010}. Many studies of the lithospheric structure have recently been undertaken, using converted body waves \citep[e.g.,][]{KFosteretal2014, Hopperetal2014} and full waveform inversions \citep[e.g.,][]{Yuanetal2014}. Studies using magnetotelluric data are adding information relating to fluids and melt in the crust \citep[e.g.,][]{Meqbeletal2014}. Focused studies provide images of regions like the Gulf Coast \citep{Evanziaetal2014} and even distant regions like northeast China \citep{Niu2014}. Still others investigate the 3\nobreakdash-D structure of the mantle \citep[e.g.,][]{Burdicketal2008, Obrebskietal2011, Porrittetal2014}, and the phase-velocity structure \citep[e.g.,][]{YangRitzwoller2008} 

In Chapter~\ref{ch:pv}, we make use of the dense and regular coverage of the TA to study the phase-velocity structure of the western US. We develop a ray-theoretical two-station phase measurement method that builds on an existing single-station method and provides better constraints on the regional phase-velocity anomalies. As with other two-station methods, the accuracy of the result is dependent on the accuracy of the event-station geometry. Because this is affected by off-great-circle propagation, we use a subset of nearby stations to estimate the arrival angle of the incoming wave, to correct the geometry and improve the measurement. This mini-array method also provides estimates of the local phase velocity. We apply these methods to more than 1600 earthquakes, each recorded at up to 400 stations, and invert the measurements to produce phase-velocity maps for Love and Rayleigh waves at periods from 25--100 s. We find that the arrival-angle corrections improve the result for Rayleigh waves, but may not be reliable for Love waves. This leads to the selection of our preferred models, and a comparison of the phase-velocity structure with known geologic features.

Despite the difficulty in making mini-array corrections for Love waves, we have produced high-quality phase-velocity maps for both Love and Rayleigh waves at a variety of periods. These maps provide constraints on mantle structure down to roughly 200-km depth. In order to more rigorously understand the potential and limitations of such models, it is important to test them for consistency with data other than the measurements originally used to make them. In Chapter~\ref{ch:pv}, we qualitatively compare the results with what we might expect from geologic features. Other ways to test the validity of a given model include experiments using the model to predict data not used in the inversion, or other types of measurements. A tool that has recently become available for such tests is the finite-element method for calculating waveforms, SPECFEM3D Globe \citep{KomatitschTromp2002a, KomatitschTromp2002b}. This freely available software package makes it possible to use a 3\nobreakdash-D model to produce synthetic waveforms that include many of the factors affecting observed data: lateral heterogeneity, anisotropy, attenuation, ellipticity, rotation, topography, bathymetry, and gravity. To facilitate its use, a growing database of synthetic waveforms calculated using this method and based on recent earthquakes is available online \citep{Tromp2010}. 

Because SPECFEM implicitly includes factors like wavefront healing and off-great-circle-path propagation, it is extremely useful for testing quantities that are highly responsive to small changes in the velocity model. One such quantity is the arrival angle of surface waves. These measurements are sensitive to the gradient of the phase-velocity structure \citep{Larson2002}; small changes in a model can have a large impact on the final result, and velocity contrasts, the strengths of which can be difficult to resolve with tomography, are particularly important. 

In Chapter~\ref{ch:aa}, we use the mini-array arrival-angle measurement method to investigate the propagation of Rayleigh waves. For small source regions with consistent results, we combine measurements to create composite maps of the variations in arrival angle as the waves cross the footprint of the array. Using the sensitivity of arrival angles to velocity heterogeneity, we compare these maps to predictions from current models of phase velocity and 3\nobreakdash-D elastic structure. The models used differ in both mantle structure and the treatment of the crust. Several prediction methods are used as well, including ray-theoretical methods and SPECFEM. While SPECFEM has been used in many applications \citep[e.g.,][]{MalcolmTrampert2011, Zhuetal2012, Daltonetal2014}, it has not previously been used to validate models in this sense. We demonstrate that current models predict well the long-wavelength signal of the arrival-angle anomaly pattern, but lack the small-scale heterogeneity needed to produce short-wavelength arrival-angle variations similar to the observations. 

In addition to the exceptional density and area of the TA, it is important to note the high quality of the TA data. It is this standard that has allowed many new observations to be made that normally are difficult, including amplitude observations \citep[e.g.,][]{EddyEkstrom2014}, attenuation \citep[e.g.,][]{PhillipsStead2008}, and the observations presented in this thesis. In particular, it was this level of quality that allowed us to observe that there was a discrepancy between the Love wave measurements made using different measurement methods in Chapter~\ref{ch:pv}. 

In Chapter~\ref{ch:ot}, we follow up on that difficulty in making high-quality Love wave measurements by investigating the effects of overtone interference on fundamental-mode phase-velocity measurements. We show the overtone interference pattern for single-station, two-station, and mini-array phase measurements. The results support the conclusions of \citet{Nettles2011} that global studies using single-station measurements are not biased, but indicate that array-based measurements may be biased. We provide a theoretical explanation for the observed patterns, which can largely be explained by the interference between the fundamental mode and first-higher overtone. We use measurements made on SPECFEM synthetic waveforms and mode-summation waveforms to support this explanation.

The goal of this dissertation is to improve our understanding of the complicated heterogeneous interior of the Earth, and the effect it has on surface-wave propagation. We approach this by imaging the regional phase-velocity variations within the western United States, examining the influence of global phase-velocity structure on the propagation of waves, and demonstrating that higher modes can influence measurements of fundamental-mode phase-velocity. The results may be used in future investigations, to obtain 3\nobreakdash-D structure, test our conceptions of structural heterogeneity, or improve array-based phase measurements. 

\raggedbottom
\pagebreak

%%%%%%%%%%%%%%%%%%%%%%%%%%%%%%%%%%%%%%%%%%%%%%%%%%%%%%%%%%%%%%%%%%%%%%%%%%%%%%%%%%%%
%%%%%%%%%%%%%%%%%%%%%%%%%%%%%%%% CHAPTER 2/FIRST PAPER %%%%%%%%%%%%%%%%%%%%%%%%%%%%%%%%%%%%%%%%%%%%%
\renewcommand\thechapter{\arabic{chapter}}
\singlespacing
\chapter[Surface-wave phase velocities of the western United States from a two-station method]{Surface-wave phase velocities of the western United States from a two-station method}
\label{ch:pv}
%\addcontentsline{lof}{chapter}{\protect\numberline{\thechapter\quad{}Chapter title}}
%\addtocontents{lof}{\contentsline{chapter}}{Chapter \thechapter: Surface-wave phase velocities of the western United States from a two-station method}
%\addtocontents{lof}{\protect\contentsline{chapter}{Chapter \thechapter: Surface-wave phase velocities of the western United States from a two-station method}}
%\addtocontents{lof}{\textbf{Chapter \thechapter: Surface-wave phase velocities of the western United States\\ from a two-station method}}
%\addtocontents{lot}{\textbf{Chapter \thechapter: Surface-wave phase velocities of the western United States\\ from a two-station method}}
\doublespacing

\thispagestyle{fancy}

%\footnotesize
\begin{raggedright}
{\bf Note: } A slightly modified version of this chapter has been published in Geophysical Journal International (2014), Vol. 196, pp. 1189--1206, http://dx.doi.org/10.1093/gji/ggt454
\footnote{AUTHORS:  Anna Foster $^a$*,  G\"oran Ekstr\"om $^a$, Meredith Nettles $^a$\\
$^a$ Department of Earth and Environmental Sciences, Columbia University, 61 Route 9W, Palisades, NY 10964, USA\\
* corresponding author: afoster@ldeo.columbia.edu}
\end{raggedright}
%\linenumbers
\normalsize

\section*{Abstract}
We calculate two-station phase measurements using single-station measurements made on USArray Transportable Array data for surface waves at periods from 25 to 100 s. The phase measurements are inverted for baseline Love and Rayleigh wave phase-velocity maps on a \degree{0.5}\nobreakdash-by\nobreakdash-\degree{0.5} grid. We make estimates of the arrival angle for each event at each station using a mini-array method similar to beamforming, and apply this information to correct the geometry of the two-station measurements. These corrected measurements are inverted for an additional set of phase-velocity maps. Arrival angles range from \degree{0} to $\pm \degree{15}$, and the associated corrections result in local changes of up to 4\% in the final phase-velocity maps. We select our preferred models on the basis of the internal consistency of the measurements, finding that the arrival-angle corrections improve the two-station phase measurements, but that Love wave arrival-angle estimates may be contaminated by overtone interference. Our preferred models compare favorably with recent studies of the phase velocity of the western United States. The corrected Rayleigh wave models achieve greater variance reduction than the baseline Rayleigh wave models, and the baseline Love wave models, which are more difficult to obtain, are robust and could be used in conjunction with the Rayleigh wave models to constrain radially anisotropic Earth structure. 

%\begin{keywords}
%Surface waves and free oscillations -- Seismic tomography -- North America -- Wave propagation.
%\end{keywords}

\section{Introduction}

The western United States has a varied and complex tectonic history, with areas undergoing extension and subduction juxtaposed with cratonic regions of long-lived stability. The present-day structure of the crust and mantle results in elastic heterogeneity, which can be mapped using seismic methods and used to inform our understanding of the geology. Because of their sensitivity, surface waves are particularly useful for investigating the crust and upper mantle, where many of the signatures of geodynamic processes are expected to be found. In addition, the depth of surface-wave sensitivity varies with frequency, providing constraints on the depth of velocity anomalies. 

The USArray Transportable Array (TA) program is an ongoing observational experiment, designed to obtain high-quality broadband seismic data spanning the continental United States, with the goal of improving knowledge of continental structure and tectonic processes.  Four hundred three-component broadband seismometers are installed with 70\nobreakdash-km grid spacing for 2\nobreakdash-yr time periods, beginning on the west coast of the United States and progressively stepping eastward (see www.usarray.org). With the data coverage provided by the TA, it is now possible to map properties of surface-wave propagation in a uniform manner across the western part of the country. High-quality phase-velocity maps can then be used to constrain models of 3\nobreakdash-D structure, as well as to predict earthquake signals and back-project recorded signals to learn more about the earthquake source. 
 
A wide variety of methods have been used to make measurements of the phase or phase velocity of surface waves in North America. Single-station methods measure the phase accumulated over the entire ray path from source to receiver, must account for source effects, and are sensitive to errors in earthquake location \citep[e.g.,][]{Trampert1995}.  Studies using single-station measurements to constrain the structure of North America include \citet{Lee1979}, \citet{vdLeeNolet1997}, \citet{Godey2003}, \citet{vanderlee2005}, \citet{Maroneetal2007}, \citet{Nettles&Dziewonski2008}, \citet{Pollitz2010}, and \citet{Yuanetal2011}. 

Multiple-station methods require two or more receivers, all recording the same event for earthquake-based techniques, and often impose specific geometric constraints related to the station alignment with the ray path. With the availability of high-quality TA data, the amount of seismic data has increased dramatically, making multiple-station methods a popular tool over the last several years. Many new and innovative techniques have been developed, including ambient noise tomography \citep[e.g.,][]{Shapiroetal2005}, Eikonal tomography \citep{Lin2009}, wave gradiometry \citep{Liang2009}, and multiple-plane-wave tomography \citep{Forsyth2005, Yang2006a}. These studies have achieved good resolution when applied to the TA data set, and results appear to be consistent. 

Two-station methods have not yet been applied to this data set. Such methods span the observational gap between ambient noise tomography, typically performed over very short inter-station paths and at short periods, and teleseismic single-station studies with long paths and at longer periods.  A two-station method reduces sensitivity to the source location and mechanism, an advantage over single-station methods. Because the measurement suppresses the influence of structure outside of the inter-station path, which can be short compared with the teleseismic path, these measurements should help constrain small-scale structure in a phase-velocity model inversion. 

A limitation of two-station methods is the restriction that the inter-station path lie approximately on the great-circle from the earthquake source. However, the grid-like design of the TA ensures many available paths and dense and even coverage of the region. It is also often assumed that the wave travels as a ray along the great-circle path from source to receiver. As early as the 1950s, it was recognized that heterogeneous velocity structure results in off-great-circle arrivals \citep{Evernden1954}. Indeed, contours of single-station phase measurements on TA data show visible variations in the wave front, evidence that deviations in the ray path result in off-great-circle arrivals. This deviation from the great-circle path is defined as the arrival angle, and several arrival-angle measurement methods have been developed \citep[e.g.,][]{Vidale1986,Lerner-Lam1989,Laske1994,Larson2002}.  In a previous study using a two-station method, \citet{Brisbourne1998} used measurements of arrival angle to discard data from events that did not arrive sufficiently aligned with the great-circle path, limiting the authors to a small number of events. Because of TA data coverage, this issue can instead be addressed by measuring the arrival angle and combining it with the inter-station phase observation.

All of the previously mentioned phase-measurement methods can in principle be applied to obtain both Love and Rayleigh wave measurements. Challenges arise when making Love wave measurements, however; greater noise on the horizontal components and a less-dispersed wave packet can prevent high-quality measurement in some circumstances. Because of this, few Love wave phase-velocity maps from TA data have been published; among these are phase-velocity maps from ambient noise tomography at periods of 8\nobreakdash--20~s from \citet{Linetal2008}, and examples at 60~s and 150~s from \citet{Yoshizawa2010}.

Love and Rayleigh waves have different sensitivity to vertical and horizontal shear velocity due to their differing particle motions. By making measurements of both wave types, radial anisotropy can be constrained, providing information about rock fabric and mantle flow patterns \citep[e.g.,][]{Becker2003}. Previous studies of radial anisotropy in the western US have indicated the presence of regional heterogeneity. Studies of upper-mantle radial anisotropy in North America have observed laterally varying strengths of 1\nobreakdash--5\% \citep{Nettles&Dziewonski2008,Yuanetal2011}. Studies using surface-wave measurements from ambient noise in the western US have also found radial anisotropy in the middle or mid-to-lower crust, with strengths up to 5\% \citep{Bensen2009,Moschetti2010}. Thus, radial anisotropy is a significant factor in velocity-structure studies, and increased knowledge of this parameter by the mapping of Love and Rayleigh wave dispersion has direct applications to geodynamic interpretations.

In this paper, we derive two-station phase measurements from a large data set of single-station phase measurements for Love and Rayleigh waves recorded on the USArray TA over a range of periods from 25 to 100~s. We estimate the arrival angle for each event at each station, and use this information to correct for off-great-circle arrival. We quantify the reduction in measurement uncertainty resulting from the use of the two-station method, as well as the effect of the arrival-angle corrections. The data set of phase measurements is inverted to obtain well-constrained models of the phase-velocity structure of the western US. We compare these models with several published studies and with known geologic structures. The Love and Rayleigh wave models are well-suited for investigations of the 3\nobreakdash-D velocity structure and radial anisotropy; we make the models available with this study to facilitate such future work. 

%%%%%%%%%%%%%%%%%%%%%%%%%%%%%%%%%%%%%%%%%%%%%%%%%%%%%%%%%%%%%%%%%%%%%%%%%%%%%%%%%%%%
%%%
\section{Methods}
Two-station methods are an effective way to limit the sensitivity of a measurement to heterogeneity along the inter-station path, canceling effects of both the source and structure outside the array \citep[e.g.,][]{Brisbourne1998}. In this study, we implement a new two-station method that builds directly on single-station phase measurements, differencing single-station measurements for pairs of stations along the source-receiver great-circle path. In addition, we make arrival-angle measurements for each single-station phase measurement and incorporate this additional information on ray geometry in our estimate of the inter-station phase velocity. 

This approach differs from traditional two-station methods \citep[e.g.,][]{Sato1955, Brune1963, Pilant1964, Knopoff1966, Bloch1968}, in which the unknown system filter, representing Earth's velocity structure between the two stations, is obtained by deconvolution of an input waveform recorded at one station from the output waveform recorded at a second station. Our approach takes advantage of modern methods for estimation of single-station phase delays \citep[e.g.,][]{Trampert1995, Ekstrom1997}, condensing the information in the waveforms at the two stations (the incoming and outgoing signals) to single-station phase measurements prior to interpreting the effect of the velocity structure (filter) between the two stations.

\subsection{Single-station phase measurements}
We make initial measurements of single-station phase delays using the method of \citet{Ekstrom1997}. The observed surface-wave signal $u^S$ can be described by a phase, $\varphi$, and amplitude, $A$, as $u^S = A \exp{\left[i\varphi\right]}$. A trial fundamental-mode model seismogram, $u^M = A^M \exp{\left[i\varphi^M\right]}$, is calculated using the source and receiver locations, focal mechanism, and predicted phase and amplitude effects of propagation in a long-wavelength 3\nobreakdash-D Earth model \citep[SH8U4L8;][]{DziewonskiWoodward1992}. The deviation in phase between $u^S$ and $u^M$ is attributed to a perturbation to the propagation phase, $\varphi_P = \varphi_P^M + \delta\varphi = \frac{\omega X}{c^M + \overline{\delta c}}$, where $c^M$ is the model phase velocity and $\overline{\delta c}$ is the average phase-velocity perturbation over the distance $X$ along the source--receiver path. 

The unknowns $A$ and $\overline{\delta c}$ are parameterized in terms of cubic B-spline polynomials. To solve for the corresponding coefficients, we first construct a whitening phase-matched filter using the model seismogram to suppress noise and amplify the fundamental-mode surface wave signal, $W^M = \frac{1}{A^M} \exp{\left[i\varphi^M\right]}$.  This filter is cross-correlated with the observed seismogram $u^S$ in the frequency domain.  The misfit between the resulting cross-correlation and the correlation of $W^M$ with $u^M$ is minimized to obtain the unknown coefficients. To avoid the common problem of cycle-skipping when calculating the phase at short periods, we require that the surface-wave dispersion curves vary smoothly with frequency and employ iterative frequency-band expansion, first using only longer periods, then gradually extending the passband to include shorter periods. Further details of the single-station method are given by \citet{Ekstrom1997}. 

\subsection{Two-station phase measurements}
We difference single-station phase-delay measurements obtained at a pair of stations $A$ and $B$ lying nearly on the same source-receiver great-circle path to obtain an average phase velocity attributable to the velocity structure between the two stations. The geometry of the system is shown in Figures~\ref{figpv:TSMgeometry} and \ref{figpv:AAcorrgeometry}. The distance between spherical wave fronts originating at the source and arriving at the two stations is denoted $D_{AB}$ (Fig.~\ref{figpv:TSMgeometry}a), and the inter-station distance $X_{AB}$. We define an angle $\beta$ between the inter-station path and the great-circle path from the source to receiver $B$. A surface wave may arrive at the pair of stations from an angle $\alpha$ off the great-circle path (Figs.~\ref{figpv:TSMgeometry}b, \ref{figpv:AAcorrgeometry}), and in this case the distance between spherical wave fronts arriving at the two stations will be the distance $D'_{AB}$. The angle between the inter-station path and the true, off-great-circle arrival path is denoted $\beta_c$. 

% 1								
\begin{figure} 
\begin{center}
\includegraphics[scale=1.]{Ch2Figs/F1_gcdiff_illus_v2} 
\caption[Two-station method source-receiver geometry]{Source-receiver geometry for the two-station measurement. (a) $\beta$ is the angle between the inter-station path and the source-to-station-B great-circle path. The distance used to calculate the inter-station propagation phase is $D_{AB}$, the distance between spherical wave fronts intersecting the two stations. (b) $\beta_{c}$ is the angle between the inter-station path and the estimated true arrival path, at an angle $\alpha$ from the source-to-station-B great-circle path. $\beta_c$ can be larger or smaller than $\beta$, as shown in the top right and bottom right panels, respectively. }
\label{figpv:TSMgeometry}
\end{center}
\end{figure}
%	
% 2
\begin{figure} 
\begin{center}
\includegraphics[scale=1]{Ch2Figs/F2_TSMaacorr_geometry2} 
\caption[Detailed two-station method station geometry]{Station geometry for the two-station method, with the great-circle-path arrival azimuth and the actual arrival azimuth separated by the arrival angle $\alpha$. $D_{AB}$ is the distance traveled between stations along the great-circle path. $D'_{AB}$ is the distance traveled between stations along the actual arrival path. }
\label{figpv:AAcorrgeometry}
\end{center}
\end{figure}
%

When an arriving surface wave does not deviate from the great-circle path, the average phase velocity $c$ due to the velocity structure between the stations can be written
%\[
\begin{linenomath*}
\begin{equation}\label{eqpv:nc}
c(\omega) = \frac{\omega D_{AB}}{\frac{\omega D_{AB}}{c_0} + (\delta \varphi_B - \delta \varphi_A)},
\end{equation} 
\end{linenomath*}
%\]
where $\omega$\, is angular frequency, $c_0$\, is the reference phase velocity at that frequency given by the Preliminary Reference Earth Model \citep[PREM;][]{PREM}, and $\delta\varphi_{i}$ is the phase anomaly measured at each station, calculated by subtracting the PREM reference phase $\varphi_i^0$ for station $i$ from the observed phase at station $i$, or $\varphi_{i}-\varphi_i^0$. 

When a wave arrives at a pair of stations at an angle other than that of the great-circle path, the actual distance traveled from the closer station to the farther station, $D'_{AB}$, is not equal to the inter-station distance $X_{AB}$ or to the difference in distance between stations along great-circle paths, $D_{AB}$ (Fig.~\ref{figpv:AAcorrgeometry}). Previous studies have noted that the inter-station distance $X_{AB}$ is always longer than $D'_{AB}$, resulting in a two-station measurement bias toward higher phase velocities \citep[e.g.,][]{Yao2006}. However, in comparison with the great-circle distance $D_{AB}$, which may be offset from the inter-station path by the angle $\beta$, the distance $D'_{AB}$ may be shorter or longer. Two-station measurements that account for these geometric variations will not necessarily be biased towards higher phase velocities, and two-station measurements corrected for arrival angle $\alpha$ will not always be lower in phase velocity than their uncorrected counterparts. 

With an estimate of the distance $D'_{AB}$, which we obtain as described in Section~\ref{sec:miniarraymethod}, Equation\ref{eqpv:nc} becomes 
%\[
\begin{linenomath*}
\begin{equation}\label{eqpv:wc}
c^{'}(\omega) = \frac{\omega D^{'}_{AB}}{\frac{\omega D_{AB}}{c_0} + (\delta \varphi_B - \delta \varphi_A)}.
\end{equation} 
\end{linenomath*}
%\]
The propagation phase $\varphi_{AB}$ attributed to the inter-station path, $X_{AB}$, is then calculated as:
%
\begin{linenomath*}
\begin{equation}
\varphi_{AB}(\omega) = \frac{\omega X_{AB}}{c^{'}(\omega)}.
\end{equation}
\end{linenomath*}
% 

\subsection{Arrival-angle estimates}\label{sec:miniarraymethod}

We wish to estimate the arrival angle $\alpha$ (Figs.~\ref{figpv:TSMgeometry}b, \ref{figpv:AAcorrgeometry}), and hence the distance $D'_{AB}$, to improve the accuracy of our description of the two-station geometry, as described above. Owing to the even and dense spatial coverage of the TA, we are able to use the single-station phase-delay data set to estimate the geometry of the wave front and the direction of propagation of the wave near each station. We do this using data from a small subset, or ``mini array'', of stations surrounding each station of interest. Measurements of particle-motion polarization across a small region have indicated that wavefields for surface waves in the period range 16--100~s are generally coherent \citep{Tanimoto2007}, justifying the use of a plane-wave approximation to measure the arrival angle. 

Using the superscript `$\rm{a}$' to identify quantities related to the location of interest, we estimate $\alpha^{\rm{a}}$ at a station $r^{\rm{a}}$ at a specified period using a mini array of $N$ receivers, denoted $r_i$ (Fig.~\ref{figpv:BFillus}). We select all stations within a radius $R$ of $r^{\rm{a}}$ having single-station phase measurements for a given earthquake. We vary the apparent source location along an arc, keeping the epicentral distance $L^{\rm{a}}_{\rm{0}}$ from $r^{\rm{a}}$ fixed. This effectively varies the arrival angle at station $r^{\rm{a}}$. For each trial source location $S^{\rm{a}}_j$, we consider the observed difference in phase between each station $r_i$ and station $r^{\rm{a}}$ to be due to the difference in epicentral distance divided by a local phase velocity $c^{\rm{a}}_j$, such that 
$\delta\varphi_{i} = \varphi_{i}-\varphi^{\rm{a}} = \omega \delta L_{ji} / c^{\rm{a}}_j + C_j.$ Here, $\delta L_{ji}$ is the epicentral distance for an apparent source location $S^{\rm{a}}_j$ and receiver $r_i$ differenced with the epicentral distance for $r^{\rm{a}}$, $L^{\rm{a}}_{\rm{0}}$. The constant $C_j$ accounts for errors in the observed phase at the reference station. 

% 3
\begin{figure} 
\begin{center}
\includegraphics[scale=1]{Ch2Figs/F3_regionalbestfit4} 
\caption[Mini-array method illustration]{Illustration of search algorithm to find the best-fit apparent source location ($S^{\rm{a}}_{\ast}$) for a given station ($r^{\rm{a}}$) and source ($S_{0}$), using stations $r_i$ that fall within a specified radius of $r^{\rm{a}}$. The distance between $S_{0}$ and $r^{\rm{a}}$, 
$L^{\rm{a}}_{0}$, remains fixed for candidate source locations $S_{j}$. }
\label{figpv:BFillus}
\end{center}
\end{figure}
%

We solve for the local phase velocity $c^{\rm{a}}_j$ in a least-squares sense. The misfit between the predicted and observed phase associated with each trial source location $S^{\rm{a}}_j$ and the corresponding best-fit local phase velocity $c^{\rm{a}}_j$ is:
\begin{linenomath*}
\begin{equation}\label{eqpv:bfaa}
\Phi^2_j = \sum_{i=1}^{N} \left[ \delta \varphi_i - \left( C_j + \frac{\omega \delta L_{ji}}{c^{\rm{a}}_j}\right)  \right]^2.
\end{equation}
\end{linenomath*}
We minimize this misfit for trial source locations corresponding to \degree{0.5} steps in arrival angle, and select the location with the smallest misfit as the preferred apparent source location $S^{\rm{a}}_{\ast}$, from which we calculate the arrival angle $\alpha^{\rm{a}}_{\ast}$ at station $r^{\rm{a}}$. The preferred local phase velocity $c^{\rm{a}}_{\ast}$ is also determined by the selection of $S^{\rm{a}}_{\ast}$. We repeat this process for all stations and events meeting a set of basic quality criteria. 

The arrival-angle estimation procedure thus yields an estimate of arrival angle for each station for each earthquake. The two-station phase-velocity calculation described in Equation~\ref{eqpv:wc} requires a single arrival-angle estimate for each station pair, and we calculate this angle using the midpoint of the two best-fit apparent source locations. The local phase velocity measurements $c^{\rm{a}}_{\ast}$, though incidental, provide independent measurements of phase velocity in the study area, and will be discussed in Section~\ref{sec:bfpvresults}. 

\subsection{Inversion}

For the determination of phase-velocity maps, we parameterize the area of the western US using a \degree{0.5}\nobreakdash-by\nobreakdash-\degree{0.5} pixel grid. To model the two-station phase data, we use a ray-theoretical framework with zero-width rays, thereby assuming each measurement is sensitive only to the inter-station path. The benefits and limitations of this approach are discussed in Section~\ref{sec:errordiscussion}. We calculate the predicted phase for each path $k$ as
%
\begin{linenomath*}
\begin{equation}
\varphi^{pred}_k = \sum_{i=1}^{N}{\omega A_{ki} p_i},
\end{equation}
\end{linenomath*}
%
where $A$ contains the length of each segment of the inter-station path crossing each pixel $i$, and $p_i$ is the phase slowness for each pixel. We invert for the model coefficients $p_i$ using a Cholesky inversion technique. We regularize the inversion by minimization of the model roughness, given by:
\begin{linenomath*}
\begin{equation}
R^2 = \sum_{i=1}^{N}{\left[ \sum_{j=N,S,E,W}{\left( \frac{p_i-p_j}{d_{ij}}\right) ^2}\right]},
%R^2 = \sum_{i=1}^{N}{\left[ \sum_{j=N,S,E,W}{\frac{\left(p_i-p_j\right)^2}{(d_{ij})^2}}\right]},
\end{equation}
\end{linenomath*}
%
where the index $j$ refers to the pixel north, south, east, and west of the $i^{th}$ pixel, and $d_{ij}$ is the distance between pixel centers, corrected for changes in pixel size with latitude. The misfit between model and data is given by:
\begin{linenomath*}
\begin{equation}
\chi^2 = \sum_k{\frac{w_k^2}{\sigma^2} \left( \varphi^{obs}_k - \varphi^{pred}_k \right)^2},
\end{equation}
\end{linenomath*}
where $w$ is the weight for a path with $n$ measurements. We choose $w_k$ to be $w_k=1+\log_{10}n_k$. The weight factor is included to allow better-sampled paths to influence the resulting model more than paths with a small number of observations. We choose to use a constant data uncertainty, $\sigma$, for each period, as described in Section~\ref{sec:datauncert}. The quantity to be minimized in the inversion is then
\begin{linenomath*}
\begin{equation}
\chi^2 + \gamma R^2,
\end{equation}
\end{linenomath*}
where $\gamma$ is a scalar damping parameter. 

%%%%%%%%%%%%%%%%%%%%%%%%%%%%%%%%%%%%%%%%%%%%%%%%%%%%%%%%%%%%%%%%%%%%%%%%%%%%%%%%%%%%
\section{Data and analysis}
We analyze data recorded on the USArray Transportable Array (TA) from January 2006 to December 2010. The size of our study area corresponds to the TA coverage through December 2010, and thus extends from the west coast of the US to approximately \degree{93}W longitude, between the southern and northern borders of the conterminous US. We use teleseismic earthquakes in the TA data set larger than magnitude 5.5, with focal depths shallower than 50~km, and epicentral distances in the range \degree{10}\nobreakdash--\degree{165}. Surface-wave measurements are made at 11 discrete periods between 25 and 100 seconds. Rayleigh waves are measured on the vertical component and Love waves on the transverse component.

\subsection{Single-station phase measurements}
Single-station phase measurements are calculated for more than 1600 events (Table~\ref{tablepv:numbers}). The quality of each measurement is determined based on fit criteria using the same rules as in \citet{Ekstrom1997}, and is denoted by A (high quality), B (good quality) or C (acceptable quality). For this study, we consider only A\nobreakdash-quality measurements at all periods. It is more difficult to obtain high-quality measurements at short periods, due to increased sensitivity to more heterogeneous shallow structure and the large variations in phase that follow. This results in fewer A\nobreakdash-quality data for short-period Love and Rayleigh waves. In addition, larger uncertainties in Love wave measurements are expected because of higher noise levels on the horizontal components. This greatly reduces the amount of A\nobreakdash-quality data for Love waves at periods shorter than 50 seconds. We experimented with including B\nobreakdash-quality measurements to expand the Love wave data set, but found that this led to an undesirable level of inconsistency in the phase measurements. Instead, we use the more limited data set of A\nobreakdash-quality measurements, as for the Rayleigh waves. The number of data collected at each step of the method for each wave type is detailed in Table~\ref{tablepv:numbers} for selected periods. 

%								TABLE- data numbers
\begin{table}
%\begin{table*}
%\begin{minipage}{80mm}
\caption[Number of phase measurements at selected periods]{Number of phase measurements at selected periods} \label{tablepv:numbers}
\resizebox{\textwidth}{!}{
\begin{tabular}{ c c c c c c c c c c }
\toprule
Wave type &Period &Events &Stations  &Single-station &\multicolumn{2}{c}{Two-station meas.} &\multicolumn{2}{c}{Inter-station paths} &Paths in\\ 
 &(s) &  & &measurements &Baseline &Corrected &Baseline &Corrected &pref. model\\
\midrule
\midrule
%		     Per  Evts  Stns    SSM      TSMN      TSMC     npath-n   npath-c npath-final
             &25  &411  &1056 &5 439   &14 085    &5 240     &12 026  &4 734   &3 892 \\ 
Love		 &35  &730  &1101 &11 585  &26 509    &14 848    &22 873  &13 317  &7 374 \\ 
Qual. A      &50  &1241 &1135 &42 758  &171 021   &115 534   &111 735 &80 797  &32 691 \\ 
			 &100 &1241 &1135 &42 758  &171 021   &128 730   &111 697 &89 540  &32 819 \\ \midrule
					      
             &25  &1096 &1158 &100 817 &682 976   &391 010   &222 750 &155 928 &56 558 \\ 
Rayleigh	 &35  &1375 &1158 &160 206 &1 212 066 &769 515   &302 741 &237 735 &78 735 \\ 
Qual. A      &50  &1642 &1159 &272 033 &2 205 538 &1 418 955 &364 076 &288 277 &92 725 \\ 
			 &100 &1642 &1159 &272 033 &2 205 538 &1 498 967 &364 235 &296 333 &94 090 \\ \bottomrule
\end{tabular}}

\medskip
\textit{Note:} Summary of the number of measurements available at each step of the analysis, including total numbers of events and stations, single-station phase measurements, two-station phase measurements with no arrival-angle corrections (baseline) and with arrival-angle corrections (corrected), and unique inter-station paths for both the baseline and corrected data sets. Numbers shown include all inter-station path lengths greater than 100~km, with $\beta \leq \degree{5}$, and misfits within the prescribed range for arrival-angle estimates. Final column shows the number of unique inter-station paths used to construct the preferred models, which includes path lengths between 350 and 750~km from the baseline data set for Love waves and the corrected data set for Rayleigh waves. 

%\end{minipage}
\end{table}	
%

\subsection{Baseline two-station phase measurements}
We calculate initial two-station phase measurements for this data set without applying arrival-angle corrections. This results in our baseline data set. An initial data selection is made using $\beta$, the difference in back-azimuth between the inter-station path and great-circle path between the event and farther station (Fig.~\ref{figpv:TSMgeometry}a), and the inter-station path length $X_{AB}$. The maximum value allowed for $\beta$, $\beta_{max}$, was determined after consideration of the variance in phase for measurements obtained on single inter-station paths. Previous studies have noted that allowing $\beta_{max}$ larger than a few degrees results in greater scatter in the resulting phase measurements \citep[e.g.,][]{Prindle2006}. However, when we account for the geometry of the two stations by using $D_{AB}$ in Equation~\ref{eqpv:nc}, we do not observe a large increase in variance, at least up to $\beta_{max} = \degree{5}$. Instead, we find that increasing $\beta_{max}$ reduces uncertainty of the median phase for each inter-station path by increasing the size of the data set. We choose $\beta \leq \degree{5}$ and initially set $X_{AB} > 100$ km, as we expect measurement error to be larger relative to the total phase for the shortest inter-station paths.

The baseline data set contains a small number of outlying measurements. We exclude these by, at each period, calculating the average phase velocity from all measurements and then removing the 1\% of measurements that deviate the most from this average. The resulting data set has up to 2.2~million two-station phase measurements at a single period (Table~\ref{tablepv:numbers}). Some inter-station paths align for several events, resulting in multiple phase measurements for a given path. For most paths, these repeat measurements are highly consistent, while for a small number of paths, the measurements span a wide range of phase values. This scatter may be caused by many factors, including multipathing, overtone interference, and finite-frequency propagation effects. These factors are explored more in Section~\ref{sec:errordiscussion}; here we select the median value for each inter-station path to reduce the influence of outliers. 

\subsection{Arrival-angle estimates}\label{sec:analysisminiarray}
We estimate the best-fit local phase velocity and arrival angle at all stations with at least three single-station phase measurements within a \degree{1}\nobreakdash-radius mini array for Rayleigh waves and a \degree{2}\nobreakdash-radius mini array for Love waves. These criteria were chosen with the goal of producing robust arrival-angle measurements, balancing the considerations that, for a single event, smaller radii can resolve shorter-wavelength variations in arrival angles and phase-velocity anomalies, while larger radii smooth both the effect of structure and noise in the measurements. Similarly, requiring more stations within a given radius provides better constraints on the arrival angle and local phase velocity but limits the measurements to only the best-covered areas. Most measurements are made using 7--8 stations. Measurements with best-fitting arrival angles larger than $\pm\degree{15}$ are discarded. Although some larger arrival-angle measurements may be real, particularly close to the ocean-continent boundary \citep{Tanimoto2007}, some may be spurious, or may represent local complexities in the wavefield that are beyond the scope of this study. This selection results in a small loss of data, with 97\% of the measurements being retained. We wish to use only good estimates of arrival angle and corresponding local phase velocity, which we define in terms of the misfit between the observed and predicted phase at all stations in the mini array. Although large misfits are not always associated with local phase velocities outside the expected range, or vice versa, we set a maximum allowed misfit for each wave type, scaled by the frequency. This results in 91\% of the remaining data being retained for Love waves and 98\% for Rayleigh waves. 

Individual estimates of local phase velocity are highly variable for different earthquakes, and can be outside the range of expected values (Fig.~\ref{figpv:bfpv_evt}). However, after averaging estimates at a given station for many events, noise and any remaining source effects are largely suppressed. The resulting maps show smooth spatial variations in velocity, and velocity anomalies with convincing size, strength, and location (Fig.~\ref{figpv:bfpv_all}). 

% 4
\begin{figure} 
\begin{center}
\includegraphics[scale=1]{Ch2Figs/F4_bfpv_oneevent} 
\caption[Local phase-velocity results for a single event]{Local best-fit phase-velocity results for 50-s Rayleigh waves for a single event located in Tonga. Measurements were made using a mini array with \degree{1} radius. }
\label{figpv:bfpv_evt}
\end{center}
\end{figure}
%
% 5
\begin{figure} 
\begin{center}
\includegraphics[scale=1]{Ch2Figs/F5_bfpv_median} 
\caption[Local phase-velocity maps from mini-array method]{Median local phase-velocity estimates at each station location for Love waves (top) and Rayleigh waves (bottom) at 25, 50, and 100-s period. Symbol at each station represents the median value for all estimates meeting the misfit criteria. }
\label{figpv:bfpv_all}
\end{center}
\end{figure}
%

\subsection{Corrected two-station phase measurements}\label{sec:analysiscorrtsm}
Using the arrival-angle estimates derived in the previous section, we calculate corrected two-station phase measurements for inter-station distances greater than 100 km. As for the uncorrected two-station measurements, we require $\beta_{c}$, the angle between the inter-station path and the backazimuth corresponding to the average arrival angle for the station pair (see Fig.~\ref{figpv:TSMgeometry}b), to be less than or equal to \degree{5}, and the 1\% of the data that lies farthest from the average corrected two-station phase velocity is discarded to remove outliers. 

The arrival angle used to make the phase correction corresponds to the midpoint between apparent source locations for the two stations. In some instances, the apparent source locations, and corresponding arrival angles, are nearly equivalent for the two stations, consistent with the energy propagating along a simple path and most refraction occurring outside the array. In other cases, the apparent source locations and arrival angles at the two stations are different, suggesting that refraction has taken place near the inter-station path. Examining two inter-station paths with many measurements (Fig.~\ref{figpv:tsaadiff}a) we find that, for the shorter inter-station distance, the difference in arrival angle for a single event at the two stations usually is small, with most values less than \degree{3}. For the longer inter-station path, the difference in arrival angle is typically larger, shown by the wider distribution of values. This is partially explained by geometrical differences in the two stations' orientations with respect to the source, but is also compatible with the notion that waves traveling longer inter-station paths are more likely to encounter velocity gradients, resulting in larger arrival-angle variations for the station pair. This observed pattern is representative of the full data set, as illustrated in Figure~\ref{figpv:tsaadiff}b. We choose to reduce our data set by selecting the median value of the two-station phase estimates for each inter-station path with multiple measurements. 

% 6
\begin{figure} 
\begin{center}
\includegraphics[scale=1]{Ch2Figs/F6_aadiff_hists_folded} 
\caption[Difference in estimated arrival angle for station pairs]{Difference in estimated arrival angle at two stations recording the same earthquake and satisfying the criteria for the two-station phase measurements. Arrival-angle measurements were made on 50-s Rayleigh waves, using a \degree{1} mini array. All figures are bar graphs folded about zero, shown as percent of the available data. (a) Top left shows results for a single, shorter (248 km) inter-station path with measurements from 68 different events. Top right shows results for a single, longer (847 km) inter-station path, with measurements from 54 events. (b) Bottom graphs show results for all inter-station path lengths between 100--350~km (left) and 750--1000~km (right). }
\label{figpv:tsaadiff}
\end{center}
\end{figure}
%

\subsection{Data uncertainties}\label{sec:datauncert}
An advantage of two-station methods over one-station methods is that several sources of uncertainty in the phase measurement are canceled. One-station measurements depend on the accuracy of the source location and focal mechanism, and phase anomalies reflect variations from the entire source-station path. Two-station measurements do not depend on these factors, and so the final datum should have a smaller associated uncertainty. 

For each period and wave type for both our baseline and corrected data sets, we calculate the data uncertainty parameter $\sigma$ using the variations in observed phase on a given inter-station path with multiple measurements. We compute the difference in phase between all two-station phase measurements on that single path.  These phase differences are compiled for all inter-station paths with multiple observations, resulting in a distribution with a mean of zero (Fig.~\ref{figpv:TSPHDIFF_R050}). For uncorrelated errors, this distribution of phase differences is expected to have a standard deviation of two times the uncertainty in a single two-station phase measurement, or $2\sigma$ \citep[e.g.,][]{Ekstrom1997}. Calculated values of $\sigma$ are listed in Table~\ref{tablepv:uncertainty}. As noted by \citet{Ekstrom1997}, this estimate excludes the uncertainty resulting from systematic errors, and hence should be considered a minimum estimate. 

% 7 
\begin{figure} 
\begin{center}
\includegraphics[scale=1]{Ch2Figs/F7_R050phdiff} 
\caption[Distribution of phase differences for uncertainty estimation]{Distribution of the differences in phase (in radians) between multiple two-station phase measurements from the same inter-station path, for 50-s Rayleigh waves. Number of measurements shown in millions. The standard deviation of this distribution is two times the standard deviation of the distribution of errors in two-station phase measurements. }
\label{figpv:TSPHDIFF_R050}
\end{center}
\end{figure}
%
%								TABLE- uncertainty
\begin{table}
\caption{Phase measurement uncertainties.}\label{tablepv:uncertainty}
%\resizebox{\textwidth}{!}{
\begin{center}
\begin{tabular}{ c  c c c  c c c }
\toprule%
\,  &\multicolumn{3}{c}{Love} &\multicolumn{3}{c}{Rayleigh} \\ \cmidrule(r){2-4}\cmidrule(r){5-7} 
Period &Single-station  &\multicolumn{2}{c}{Two-station} &Single-station   &\multicolumn{2}{c}{Two-station} \\
(s) &(A) &Baseline &Corrected &(A) &Baseline &Corrected \\ 
\toprule
%                TSM                 TSM    
%	  SSMA  base    corr   SSMR   base   corr  
25	 &0.903 &0.158 &0.078 &0.951 &0.306 &0.132 \\ 
27	 &0.757 &0.142 &0.084 &0.833 &0.237 &0.120 \\ 
30	 &0.589 &0.132 &0.061 &0.709 &0.179 &0.098 \\ \midrule
32	 &0.595 &0.295 &0.153 &0.759 &0.245 &0.133 \\ 
35	 &0.477 &0.256 &0.184 &0.648 &0.185 &0.109 \\ 
40	 &0.385 &0.213 &0.137 &0.533 &0.144 &0.092 \\ 
45	 &0.338 &0.182 &0.123 &0.454 &0.124 &0.084 \\ \midrule
50	 &0.403 &0.278 &0.239 &0.569 &0.180 &0.106 \\ 
60	 &0.319 &0.231 &0.203 &0.437 &0.135 &0.090 \\ 
75	 &0.262 &0.214 &0.198 &0.331 &0.094 &0.070 \\ 
100	 &0.215 &0.148 &0.139 &0.250 &0.067 &0.057 \\ \bottomrule 
\end{tabular}%}
\end{center}
\medskip
\textit{Note:} Estimated uncertainties (in radians) for single- and two-station phase measurements, based on $\frac{1}{2}$ the standard deviation of the distribution of phase differences for multiple measurements on the same path (two-station) or similar paths (single-station). Two-station measurement uncertainty reflects quality A data with no arrival-angle corrections (baseline) or with arrival-angle corrections (corrected). Two-station uncertainties are calculated using all path lengths greater than 100 km. A sample distribution is shown for two-station phase measurements of 50-s period Rayleigh waves in Figure~\ref{figpv:TSPHDIFF_R050}. Single-station phase uncertainties are from \citet{Ekstrom2011}. 

\end{table}
%

The two-station uncertainty estimates can be compared with those made by \citet{Ekstrom2011} for single-station phase measurements at the same periods. As expected, the two-station measurements show lower uncertainty at all periods (Table~\ref{tablepv:uncertainty}). For Rayleigh waves, the uncertainty for two-station measurements is smaller by a factor of 4\nobreakdash--8 at all periods. The uncertainties are also smaller for Love waves, though the differences are not as large: the two-station measurements have uncertainties smaller by only a factor of 2 or less at most periods, with the greatest reduction at the shortest periods. The differences in the reduction in uncertainty between Love and Rayleigh waves may indicate an important difference in the propagation behavior between the two wave types, which in turn affects the degree to which the two-station method can be successful. One possible factor may be overtone interference; though overtone contamination is found to be small for single-station measurements made using the \citet{Ekstrom1997} approach \citep{Nettles2011}, it may be large relative to the phase accrued along the shorter paths used in a two-station method. This effect is discussed further in Section~\ref{sec:errordiscussion}. Here, we conclude that the two-station method improves upon the single-station method for regional studies. 

A comparison between the uncertainties calculated for our baseline and corrected two-station measurements shows smaller uncertainties for the corrected data set at all periods. At shorter periods, the corrected uncertainty is smaller by a factor of $\thicksim$2, while at long periods the corrected uncertainty is smaller by a factor of 1.1--1.5. This reduction in uncertainty is likely due in part to improvements from the corrected two-station method, and in part to the selection of a smaller, high-quality data set, which is a result of making the mini-array estimates. The improvements from the method will be explored further in Section~\ref{sec:aaimprove}. 

\subsection{Inversion}\label{sec:analysisinversion}
We invert the baseline and corrected two-station phase data sets separately in order to determine which method gives better results. After performing trial inversions using the full data set as well as different subsets of the data, we choose to limit the inter-station path length such that $350~\rm{km} \leq X_{AB} \leq 750~\rm{km}$. This equates to roughly one-third of the full data set (Table~\ref{tablepv:numbers}). We choose to remove the longer paths to improve the short-wavelength resolution in our final models, and because longer paths increase the possibility that the wave front will fail to behave as a plane wave and distort the two-station measurements. We exclude shorter paths because the measurement errors in phase are larger relative to the propagation phase, translating to greater uncertainty in phase velocity than for longer paths. In our inversions, we use the data uncertainties calculated for the full data set, as given in Table~\ref{tablepv:uncertainty}. The value of the damping parameter $\gamma$ is chosen subjectively at each period, with smoothness being favored over higher variance reduction. The same damping parameter is used for inversion of the baseline and corrected datasets at each period. 

This analysis leads to two separate phase-velocity models at each period: the baseline model and the arrival-angle-corrected model. The variance reduction for all models is given in Table~\ref{tablepv:goodnessoffit} with respect to a model consisting of the mean retrieved phase velocity at each period. The variance reduction for Love waves is low at longer periods, reflecting variability in the data that cannot be fit by the model. The variance reduction for Rayleigh waves is high at all periods, ranging from 77--92\%. A measure of the goodness of fit, $\chi^2/N_{wt}$, is also given in the table, calculated from:
\begin{linenomath*}
\begin{equation}\label{eqpv:goodnessoffit}
\chi^2 = \sum_{i=1}^{N}{\frac{w_i^2}{\sigma_i^2} \left( \varphi_i - \varphi_i^P \right)^2} 
\text{ , and } N_{wt} = \sum_{i=1}^{N}{w_i^2}.
\end{equation}
\end{linenomath*}
The goodness-of-fit achieved ranges from 1.05 to 9.40, with values closer to 1 being more desirable. We discuss and select our preferred models in the following section. 

%								TABLE- Goodness of fit
\begin{table}
\caption[Model fits to two-station data]{Model fits to two-station data}\label{tablepv:goodnessoffit}
\resizebox{\textwidth}{!}{
\begin{tabular}{ c  c c c c  c c c c }
\toprule%
\,     &\multicolumn{4}{c}{Love} &\multicolumn{4}{c}{Rayleigh} \\ \cmidrule(r){2-5}\cmidrule(r){6-9} 
Period (s) &\multicolumn{2}{c}{Variance Reduction} &\multicolumn{2}{c}{Goodness of Fit} &\multicolumn{2}{c}{Variance Reduction} &\multicolumn{2}{c}{Goodness of Fit} \\
  &Baseline &Corrected &Baseline &Corrected &Baseline &Corrected &Baseline &Corrected \\ 
\toprule
%	  VR-b  VR-c    GF-b  GF-c   VR-b   VR-c  GF-b  GF-c
25	 &80.92 &90.53 &2.78 &4.70 &76.56 &86.87 &1.76 &4.69 \\ 
27	 &77.34 &88.57 &3.79 &4.47 &79.66 &87.73 &2.12 &4.67 \\ 
30	 &79.24 &87.72 &3.70 &9.32 &83.50 &89.37 &2.38 &5.08 \\ \midrule
32	 &50.26 &52.93 &2.47 &8.06 &82.54 &88.00 &1.34 &3.09 \\ 
35	 &52.10 &53.30 &2.52 &4.06 &86.64 &90.60 &1.58 &3.23 \\ 
40	 &50.45 &51.83 &3.24 &6.50 &88.95 &91.91 &1.88 &3.39 \\ 
45	 &45.88 &47.38 &4.75 &9.40 &89.84 &92.14 &2.02 &3.40 \\ \midrule
50	 &26.95 &27.06 &3.63 &4.61 &87.80 &91.63 &1.05 &2.03 \\ 
60	 &25.27 &25.76 &4.09 &5.09 &88.26 &91.22 &1.17 &1.97 \\ 
75	 &20.75 &21.02 &4.31 &5.03 &88.31 &90.90 &1.27 &1.76 \\ 
100	 &22.95 &22.90 &4.05 &4.54 &84.86 &86.57 &1.15 &1.37 \\ \bottomrule 
\end{tabular}}

\medskip
\textit{Note:} Variance reduction with respect to a weighted mean model achieved by inversion of the baseline and corrected data sets for Love and Rayleigh waves at each period. Goodness of fit, calculated as the total variance divided by the weighted number of measurements or $\chi^2/N_{wt}$, is also shown. 

\end{table}
%

%%%%%%%%%%%%%%%%%%%%%%%%%%%%%%%%%%%%%%%%%%%%%%%%%%%%%%%%%%%%%%%%%%%%%%%%%%%%%%%%%%%%
\section{Preferred Models}\label{sec:preferredmodel}

\subsection{Comparison with local phase-velocity models}\label{sec:bfpvresults}
We have measured local phase velocities in the process of making mini-array arrival-angle estimates, and averaged them over many events to generate local phase-velocity estimates at most stations (Fig.~\ref{figpv:bfpv_all}; Section~\ref{sec:analysisminiarray}). These local measurements are based on the phase information in multiple very short inter-station paths within each mini array. They are therefore independent of the models that result from inversion of our two-station measurements, which all derive from paths longer than 350~km. We compare the two results to assist in our assessment of the maps derived from the two-station measurements. 

Local phase-velocity estimates are very consistent with the phase-velocity models from inversion of baseline two-station measurements for Rayleigh waves. Evaluating the models at each station location where we have a median local phase-velocity estimate, we find correlation values of 91\nobreakdash--97\%. Furthermore, scatter plots of Rayleigh wave measurements from the two methods fall near a 1:1 line (Fig.~\ref{figpv:scatter}), suggesting there is no bias in the mini-array method and independently confirming our inversion results. The good agreement between the two sets of results also increases our confidence in the Rayleigh wave arrival-angle estimates from the mini-array approach. 

% 8
\begin{figure} 
\begin{center}
\includegraphics[scale=1]{Ch2Figs/F8_scatterplots_pix_bfpv_baseline} 
\caption[Comparison of two-station and local median phase-velocity maps]{Scatterplots showing the phase velocity for the baseline two-station model evaluated at each station location versus the estimated local median phase velocity at each station for Rayleigh waves (top) and Love waves (bottom) at 25 and 50-s period. Black line shows 1:1 relation.}
\label{figpv:scatter}
\end{center}
\end{figure}
%

Love wave estimates from the two methods are much less consistent, with correlation values of 32\nobreakdash--72\%. Love wave scatter plots at periods of 35~s and longer also show an offset of the mini-array measurements to higher phase velocities compared with the baseline two-station phase-velocity model (Fig.~\ref{figpv:scatter}). This offset may be due to overtone interference (Section~\ref{sec:errordiscussion}), or a geometrical effect of the mini-array method. It is possible that the corresponding Love wave arrival-angle estimates are also affected. We therefore choose not to use the Love wave arrival-angle estimates in this study, and identify the discrepancy between the two types of results as a topic for future study. Here, we select the phase-velocity maps from the baseline data set to be our preferred model for Love waves. 

\subsection{Improvements from arrival-angle corrections}\label{sec:aaimprove}

To investigate the extent to which arrival-angle corrections improve the two-station phase-velocity measurements for Rayleigh waves, we compare the variance reduction achieved by phase-velocity models derived from the corrected phase measurements with models derived from baseline uncorrected phase measurements (Table~\ref{tablepv:goodnessoffit}). At all periods, models made with corrected data produce greater variance reduction. This indicates that the corrected two-station phase measurements are more self-consistent throughout the study region than the uncorrected measurements. Values for the goodness of fit ($\chi^2/N_{wt}$) for the corrected model are generally higher than for the baseline model, because the uncertainty for the corrected data set is lower. If we remove the effect of the measurement uncertainty from the goodness of fit (Eq.~\ref{eqpv:goodnessoffit}), we find that the observed goodness of fit from the corrected models is 1.2--1.8 times better than that from the baseline model. 

We also compare the phase measured using waves traveling in one direction on an inter-station path versus measurements made using waves traveling the opposite direction on the same path. We use the median for all measurements traveling in one direction, or from station A to station B, and compare this with the median phase for all measurements on the same path using waves traveling from station B to station A (Fig.~\ref{figpv:ABBA_allper}a). Ideally, these reciprocal two-station measurements would have no directional dependence, since there are few physical causes of 1\nobreakdash-$\psi$ anisotropy for surface waves. Although reciprocal two-station measurements are well correlated overall, differences in directional median phase on reciprocal paths are common. This scatter likely results from several factors, including off-great-circle propagation. This inference is supported by the effect of arrival-angle corrections on the comparison: corrected two-station phase-velocity measurements show notably less scatter than uncorrected measurements. We quantify this effect by comparing the distribution of deviations from the ideal one-to-one line for uncorrected and corrected phase measurements. The standard deviations of these distributions for each period and wave type show a reduction in the spread of deviations for the corrected data set (Fig.~\ref{figpv:ABBA_allper}b). 

% 9 AB-BA figure
\begin{figure} 
\begin{center}
\includegraphics[scale=1]{Ch2Figs/F9_ABBA} 
\caption[Reciprocal-path phase differences for baseline and corrected two-station measurements]{(a) Example of differences in median phase measurements for waves traveling in one direction, station A to station B, with the median phase measurement for waves traveling in the opposite direction, B to A, for 50-s Rayleigh waves for a single path. Black dots: uncorrected measurements; gray dots: corrected measurements. The scatter from the one-to-one line (black) is reduced by applying arrival-angle corrections to the measurements. (b) Plot of the standard deviation of the distribution of deviations from the one-to-one line, as in (a), for Rayleigh waves at all periods for the full data set. We assume a mean of zero. In all cases, corrections reduce the differences in reciprocal-path phase. }
\label{figpv:ABBA_allper}
\end{center}
\end{figure}
%

Our experiments thus indicate that the arrival-angle correction improves the two-station measurement for Rayleigh waves. We select the maps derived from the corrected data as our preferred Rayleigh wave phase-velocity model.  

\subsection{Phase-velocity maps}

Because of the improvements in variance reduction and other quantitative measures, we have selected the phase-velocity maps from the corrected data set as our preferred models for Rayleigh waves (Section~\ref{sec:aaimprove}). Based on the likely inconsistencies in the Love wave arrival-angle estimates, we have selected the baseline models as our preferred models for Love waves (Section~\ref{sec:bfpvresults}). 

The final models are shown in Figures~\ref{figpv:pvmodels_R} and \ref{figpv:pvmodels_L} for 6 periods ranging from 25 to 100\nobreakdash-s period. Final models for all 11 periods can be found in Appendix~\ref{appendixA}. The phase velocity is given as a percent variation with respect to the model mean. Rayleigh wave phase velocities range from 3.4\nobreakdash--4.3~km/s, while Love wave phase velocities range from 3.7\nobreakdash--4.9~km/s. The wavelength of anomalies varies with period for both wave types. At 25\nobreakdash-s period, fast to slow variations occur over length scales of a few degrees, whereas at 100~s, variations occur smoothly over tens of degrees. At a given period, the pattern of anomalies is similar for Love and Rayleigh waves, but consistent with expected differences from the differing sensitivity kernels. The observed anomalies will be discussed further in Section~\ref{sec:geology}. 

% 10
\begin{figure} 
\begin{center}
\includegraphics[scale=1]{Ch2Figs/F10_Rayl_wt01wcqAsA} 
\caption[Rayleigh wave phase-velocity maps from corrected two-station measurements]{Rayleigh wave phase-velocity models from two-station phase measurements corrected for arrival angle at 25-, 30-, 40-, 50-, 75-, and 100-s period. The model includes the area with path coverage, expanded by one pixel in each direction. Values are given as a percent deviation with respect to the model mean at each period.}
\label{figpv:pvmodels_R}
\end{center}
\end{figure}
%
% 11 
\begin{figure} 
\begin{center}
\includegraphics[scale=1]{Ch2Figs/F11_Love_wt01ncqAsA} 
\caption[Love wave phase-velocity maps from baseline two-station measurements]{Love wave phase-velocity models from two-station phase measurements at 25-, 30-, 40-, 50-, 75-, and 100-s period. Baseline phase measurements were used. The model includes the area with path coverage, expanded by one pixel in each direction. Values are given as a percent deviation with respect to the model mean at each period. }
\label{figpv:pvmodels_L}
\end{center}
\end{figure}
%

\subsection{Sources of error} \label{sec:errordiscussion}

\subsubsection*{R\lowercase{ay} \lowercase{theory}}

Many commonly used methods for interpreting phase or travel-time measurements are based on ray theory, with the observation related to the intrinsic velocity by a ray-path integral. Ray theory is exact in the limit of infinite frequency, and generally applicable when the length scale of the heterogeneity is greater than the seismic wavelength.  When this is not the case, the sensitivity of the wave is not limited to the ray path, but rather includes a broader area surrounding it. The limitations of ray theory have been well documented \citep[e.g.,][]{Woodhouse&Girnius1982, Spetzleretal2002, Boschi2006}. The alternative is to include finite-frequency effects, usually by defining a 2\nobreakdash-D or 3\nobreakdash-D sensitivity kernel \citep[e.g.,][]{Meieretal1997, Yoshizawa&Kennett2002, Yoshizawa&Kennett2005, Zhouetal2004, Tromp2005, Peteretal2007, LinRitzwoller2010}.  

Comparisons of ray-theoretical and finite-frequency approaches have shown that 3\nobreakdash-D kernels can provide improvements in predicting fundamental-mode surface-wave travel-time anomalies and recovering small-scale heterogeneity \citep[e.g.,][]{Zhou2005, Peteretal2009}. However, accurate 3-D kernels are computationally expensive, and several authors have found that ray theory performs equally well or better than lesser approximations in situations with dense path coverage and with appropriate regularization \citep[e.g.,][]{Spetzleretal2002, Sieminski2004, Boschi2006, Trampert&Spetzler2006}. In light of this, and given the exceptional data coverage at most periods, we have confined the sensitivity to the ray path to simplify the calculations. A further simplification used here, common to two-station methods, involves the assignment of a phase velocity measured slightly off the inter-station path to the
inter-station path in the inversion. These limitations are most likely to affect small-scale structure, and we expect our choice of damping to help mitigate such effects in the models we present here. 

\subsubsection*{Plane-wave assumption}

A premise of many array-based measurements is that the wavefield can be approximated locally as a plane wave, even when the wave front is distorted on a larger scale moving through a heterogeneous earth. This is a commonly accepted assumption for teleseismic studies using arrays covering a small area. Even for such situations, however, \citet{Wielandt1993} noted biases in phase measurements introduced by non-planar wavefield geometries. 

The spatial variation in arrival angles we observe for a single event at many different stations is likely related to the effects of a non-planar wavefield. By correcting the two-station geometry using the arrival angles, we remove much, but probably not all, of the error resulting from such wavefield distortions. These errors are likely largest for a station pair with very different arrival angles, indicating scattering along the inter-station path. At the shortest periods used in this study, the observed arrival angles increase in amplitude, suggesting that we may be approaching the limit of the plane-wave assumption at these periods. 

\subsubsection*{Overtones}

Several authors have shown that overtones can affect the single-station phase measurement method, but with a sufficient distribution of sources and path lengths, the effect is not systematic \citep{Boore1969,Forsyth1975,Nakanishi&Anderson1983,Nettles2011}. It is not well known how overtones might influence multiple-station methods such as the two-station and mini-array methods used here, though some investigations have been made regarding short-period and active-source data \citep[e.g.,][]{Forbriger2003, Kimman2011}. Due to the similar fundamental-mode and overtone group velocities of Love waves, contamination of fundamental-mode phase measurements is a more serious concern for this wave type \citep[e.g.,][]{Nettles2011}. We have already mentioned overtone contamination as a possible explanation for the smaller reduction in uncertainty for two-station phase measurements compared to single-station phase measurements for Love waves, and for the spuriously high local mini-array phase-velocity estimates for Love waves. By using the broad path distribution of the TA, a large and globally-distributed number of earthquake sources, and not using the Love wave arrival-angle estimates, we believe that the phase-velocity models presented here are not strongly affected by overtone interference. 

\subsubsection*{Anisotropy}

Surface-wave azimuthal anisotropy has been observed in the western United States, with average strengths of approximately 1\nobreakdash-2\% in the period range we discuss here \citep[e.g.,][]{Marone2007,Beghein2010}. However, we expect the azimuthal averaging that is part of our inversion to reduce the effect of azimuthal anisotropy, particularly given the excellent azimuthal coverage afforded by the TA. The high correlation between the preferred models and the mini-array results, which should be affected by anisotropy in different ways, suggests that the azimuthal averaging has the expected effect. This high correlation is especially indicative for Rayleigh waves, which normally show the strongest azimuthal signal. 

%%%%%%%%%%%%%%%%%%%%%%%%%%%%%%%%%%%%%%%%%%%%%%%%%%%%%%%%%%%%%%%%%%%%%%%%%%%%%%%%%%%%
\section{Discussion}

\subsection{Effect of arrival-angle corrections}\label{sec:aadiff}

In this study, we find that estimated arrival angles at each period span the allowed range from \degree{-15} to \degree{+15}, with generally larger angles at shorter periods. Locally, the difference in estimated phase velocity resulting from corrections for arrival angle can be up to 4\% with respect to the model mean, although most values fall in the range 0\nobreakdash--1\%. These differences are similar to previous observations: \citet{Alsina1993} used a wave-front reconstruction method similar to our mini-array method, and found that typical deviations for a wave traveling a largely oceanic path before crossing the European continental margin were no greater than $\sim$\degree{8} at 15--100 s period, resulting in an error in inter-station phase velocity of less than $1\%$. 

The change in the Rayleigh wave velocity models due to arrival-angle corrections is illustrated in Figure~\ref{figpv:pvmodels_R_aadiff}, where the left column is the model derived from baseline (uncorrected) phase measurements, the middle column is the model derived from corrected phase measurements, and the right column is the difference. The values shown indicate the change in velocity due to the arrival-angle corrections as a percent of the mean velocity of the two models for each period: blue indicates the corrected model is faster than the baseline, while red indicates the corrected model is slower. At all periods, 70--75\% of the pixels show a change to slower velocities. 

% 12
\begin{figure} 
\begin{center}
\includegraphics[scale=1]{Ch2Figs/F12_model_aadiff_new} 
\caption[Comparison of Rayleigh wave phase-velocity maps from baseline and corrected two-station measurements]{Rayleigh wave phase-velocity models from baseline two-station phase measurements (left), and models from phase measurements corrected for arrival angle (middle) at 25-, 50-, and 100-s period, top to bottom. Phase-velocity values are shown in the area with path coverage, expanded by one pixel in each direction. Right column shows the difference between the two models (corrected minus baseline) as a percent of the average velocity of the two models. }
\label{figpv:pvmodels_R_aadiff}
\end{center}
\end{figure}
%

For Rayleigh waves at short periods (25 s), the changes to the velocity model resulting from the corrections are concentrated along the west coast, where the ocean-continent transition crossed by all waves with sources in the southwest, west, and north Pacific is expected to cause large deviations.  For longer-period Rayleigh wave models, 50\nobreakdash--100 s, the changes in velocity are smaller and more diffusely distributed across the western US, likely reflecting the reduced sensitivity of these waves to the relatively large velocity variations in structure at shallower depths, and their larger sensitivity to greater depths, where velocity variations are expected to be weaker.

We expect that Love wave phase-velocity models would be similarly affected by arrival-angle corrections. This would likely mean changes in velocity of several percent, with larger changes at shorter periods and concentrated near the west coast. As with the Rayleigh wave models, we would not expect arrival-angle corrections to affect the first-order features in the Love wave phase-velocity maps. 

\subsection{Comparison with published models}
Western North America has been well studied by seismic tomography, both in the context of global models and in higher-resolution regional studies. We compare our phase-velocity models with earlier results, to note features that are consistent across models and wavelengths, and those that provide new information. Here, we compare our study with results from three published models (Fig.~\ref{figpv:Otherstudies}). 

% 13
\begin{figure}
\begin{center}
\includegraphics[scale=1]{Ch2Figs/F13_otherstudies} 
\caption[Comparison of phase-velocity maps from this study and other published studies]{Comparison between this study (left) and published studies (right). Top: 50-s Rayleigh wave model from this study and the phase-velocity model derived from S362ANI and CRUST 2.0 (sampled at $\degree{2}\times\degree{2}$ pixels). Middle: 50-s Love wave model from this study and the phase-velocity model associated with the study of Nettles and Dziewonski (2008) (sampled at $\degree{1}\times\degree{1}$ pixels). Bottom: 32-s Rayleigh wave model from this study and from the University of Colorado at Boulder \citep{Lin2009}, which uses measurements of ambient seismic noise ($\degree{0.2}\times\degree{0.2}$ pixels). At each period, the models are plotted for shared pixels only. }
\label{figpv:Otherstudies}
\end{center}
\end{figure}
%

The first comparison is with the combination of a global radially-anisotropic mantle velocity model, S362ANI \citep{Kustowskietal2008}, and a global crustal model, CRUST 2.0 \citep{CRUST2}. These models are often used together to predict velocities or seismograms when the effects of global 3\nobreakdash-D structure are desired \citep[e.g.,][]{Tromp2010}. S362ANI was developed in a ray-theoretical framework using teleseismic body-wave travel times, fundamental-mode surface-wave measurements, and overtone data from waveform inversions. CRUST 2.0 was developed by compiling seismic data on crustal thickness, velocity, and density, averaged globally for similar geological and tectonic settings. These measurements were combined with published ice and sediment thicknesses to create a collection of 1\nobreakdash-D crustal profiles. Each \degree{2}\nobreakdash-by\nobreakdash-\degree{2} grid cell is assigned one such profile, composed of ice, water, soft sediments, hard sediments, and upper, middle, and lower crust. We calculate phase velocities from these models using local depth profiles. An example is shown in the top row of Figure~\ref{figpv:Otherstudies}, for Rayleigh waves at 50~s. We find that the long-wavelength structure from the global model is also well recovered in our model; however, there are noticeable differences at the pixel scale. These differences are primarily due to the lack of spatial resolution of mantle structure in the global model. A good example of this is the Yellowstone hotspot, a strong, short-wavelength, slow mantle feature that is not resolved in S362ANI. Short-wavelength structure in the global phase-velocity map originates from CRUST 2.0. The correlation between the two models shown is 86.5\%. Similar correlations are found for both Love and Rayleigh waves at periods of 50~s or longer. At periods shorter than 50~s, the phase-velocity structure in the two models is different, owing to the lack of smaller-scale structure in the global model. 

The second comparison study, that of \citet{Nettles&Dziewonski2008} (ND2008), finds the radially anisotropic shear-velocity structure of North America with a resolution of a few hundred kilometers while simultaneously inverting for lower-resolution global structure. This focused study with variable resolution provides improved constraints on continental-scale structure over a low-resolution global study. The associated phase-velocity map for 50\nobreakdash-s Love waves \citep{NettlesThesis} is compared with our results in Figure~\ref{figpv:Otherstudies}. We find that the long-wavelength signal present in ND2008 is captured by our regional study, with increased detail in the extent and magnitudes of the anomalies in our maps. This is particularly apparent in the Basin and Range, where we observe a ring of slow velocities enclosing the Basin and Range in our study, but a broad, lower-amplitude low-velocity anomaly is seen in ND2008. The correlation value is 86.5\%. Analogous effects are observed at other periods and for Rayleigh wave phase-velocity maps, with correlation values ranging from 77\% for 35\nobreakdash-s Love waves to 91\% for 50\nobreakdash-s Rayleigh waves. 

The third study we use for comparison is a regional phase-velocity model derived using USArray TA data by \citet{Lin2009}. This study uses the method of Eikonal tomography, in which recordings of ambient seismic noise are cross-correlated to compute the empirical Green function and phase travel times \citep{Linetal2008}. These measurements are interpolated to form a phase travel-time surface on a \degree{0.2}\nobreakdash-by\nobreakdash-\degree{0.2} grid, which is then related to the local phase speed and direction of wave propagation using the Eikonal equation. In comparing our 32\nobreakdash-s Rayleigh wave models (Fig.~\ref{figpv:Otherstudies}), we see that the two models are highly consistent despite the lower resolution in our study ($\degree{0.5} \times \degree{0.5}$), with correlation values of 89.4\%, calculated using the models sampled at the center of every pixel of \citet{Lin2009}. Differences arise where \citet{Lin2009} resolve very short-wavelength features, on the order of the station spacing (70 km). The strength of the heterogeneity is also very similar between the models. Many velocity anomalies, such as the one located near the Rio Grande Rift (Fig.~\ref{figpv:geology}), show very little difference. The largest differences in strength of velocity anomalies, such as at Yellowstone hotspot, are less than 0.1~km/s. At shorter periods, the phase-velocity maps show similar correlations of 88--90\%, and higher correlations are found with the longer-periods models of \citet{LinRitzwoller2011}, derived using Helmholtz tomography. 

In all cases, the study with higher resolution has a slightly larger velocity range, as might be expected, but the differences in the strength of heterogeneity are small. Based on these comparisons, recent global and regional models of the western US are overall consistent, with the regional models and TA data providing additional information on short-wavelength structure. Regional results from different methods are also quite consistent, bringing the seismological community close to obtaining a consensus on Rayleigh wave phase-velocity structure in the western US. Very few Love wave models from TA data have been published; the Love wave results from this study help fill the knowledge gap for smaller-scale Love wave phase-velocity structure of the western US.  

\subsection{Geologic features}\label{sec:geology}

Surface waves are sensitive to intrinsic velocity structure over a range of depths. For this reason, anomalies cannot be interpreted directly in terms of heterogeneity at depth. However, many anomalies co-locate with geologic features (Fig.~\ref{figpv:geology}) in the depth range of their maximum sensitivity.

% 14 Geology
\begin{figure} 
\begin{center}
\includegraphics[scale=1]{Ch2Figs/F14_geology_Anders1992_bdrys} 
\caption[Major geologic features in the western United States]{Topographic map showing major geologic provinces: Coast Range, Cascade Ranges, Columbia Plateau flood basalts, HLP and NW B\&R (High Lava Plains and Northwest Basin and Range), SRP (Snake River Plain), YS (Yellowstone), NE B\&R (Northeast Basin and Range), Rocky Mountains, Great Plains, RGR (Rio Grande Rift), Colorado Plateau, WF (Wasatch Front), Southern Basin and Range, Great Basin and Northern Basin and Range, Sierra Nevada, Great Valley. Topography from GeoMapApp; geologic provinces after \citet{Simpson1992}. } \label{figpv:geology}
\end{center}
\end{figure}
%

For example, in the 25\nobreakdash-s Rayleigh wave phase-velocity model (Fig.~\ref{figpv:pvmodels_R}), phase velocities are mainly representative of lower crustal and uppermost mantle velocity structure. Slow anomalies are co-located with the Yellowstone hotspot and the Snake River Plain, as well as in the vicinity of the High Lava Plains. Slow anomalies are also associated with the edges of the Basin and Range extensional province, notably along the western edge adjacent to the Sierra Nevada mountain range, and the eastern edge, extending through the Wasatch Front and the transition to the western Colorado plateau. These areas are the most seismically active portions of the Basin and Range \citep{Panchaetal2006}. In the northern Basin and Range, two small high-velocity anomalies appear in the same location as areas of thinned crust \citep{Klempereretal1986}. On the eastern edge of the Colorado Plateau, slow anomalies are located along the northern Rio Grande Rift, extending northward beneath the Rocky Mountains in Colorado, possibly reflecting the velocity contrast between the deeper crustal root of the Rockies and nearby upper mantle velocities. 

There does not appear to be a coherent velocity anomaly that can be associated with the subduction of the Juan de Fuca plate in Cascadia; instead, we see a moderately low velocity anomaly at approximately \degree{47}N, flanked by average velocities to the south and north. This is consistent with a regional shear-wave velocity study, that finds low velocities extending to the east at \degree{47}N at shallow depths, 4\nobreakdash--16~km, underlain by faster material presumed to be the Juan de Fuca slab \citep{Calkins2011}.  These features are adjacent to a high-velocity anomaly co-located with the Columbia River flood basalts to the east in Washington state. 

Fast anomalies are observed in the Great Valley region, which has been interpreted as fast oceanic crust \citep{Godfreyetal1997}, and along the southwestern edge of the study region, which comprises the southern Basin and Range and northern tip of the Gulf of California. The thinned crust in this region may contribute to the fast velocity anomaly \citep{Lewisetal2001}. Finally, the entire eastern edge of the study area shows high velocities, likely representing the western edge of the North American craton. 

At 50\nobreakdash-s period, slow Rayleigh wave anomalies cover the extent of the Basin and Range, an area characterized by medium to high heat flow \citep{Lysak1992}. The strongest slow anomalies align with the Yellowstone hotspot, the Snake River Plain, the High Lava Plains, and the Rio Grande Rift. At this period, Rayleigh waves are primarily sensitive to the upper mantle, and any reduced-velocity signal from the root of the Rocky Mountains is no longer apparent. Fast regions are located along the eastern edge of the study area, following the edge of the craton. At 100\nobreakdash-s period, which corresponds to a maximum sensitivity at approximately 150\nobreakdash-km depth, the dominant contrast is between slow velocities in the non-cratonic western US and fast velocities in the craton to the east. The exception to this first-order trend is the moderate velocities observed beneath the Columbia River flood basalt province. 

Love wave phase velocities provide a different perspective on the velocity structure, with greater sensitivity to crustal structure at all periods. At short periods (25\nobreakdash--35~s), many of the same features discussed in the Rayleigh wave models are observed. The strongest low-velocity anomaly is located in the northern Rio Grande Rift/southern Rocky Mountain region. Slow anomalies are present in the area of the Yellowstone hotspot and ringing the Basin and Range. Fast anomalies are observed in the Columbia River flood basalts and the northeastern portion of the study area. There is a fast anomaly in the southwestern region of the study area adjacent to the Gulf of California, which may correspond to the aforementioned region of thinned crust. And finally, a slow anomaly on the Gulf coast of Texas is prominent in the Love wave models, likely caused by the thick sediments there \citep{McGookey1975}. 

The 50\nobreakdash-s period Love wave model is very similar to the 25\nobreakdash-s Rayleigh wave model. Slow anomalies are co-located with the Colorado Plateau, possibly reflecting the thicker crust underlying this geologic feature \citep{Zandtetal1995}. Slow anomalies are also observed in the regions of the Snake River Plain, High Lava Plains, and around the edges of the Basin and Range, including the Sierra Nevada. Fast anomalies are again found near the Gulf of California, the eastern edge of the study area, and the northern Cascades. 

At 100\nobreakdash-s period, Love wave phase-velocity anomalies map the transition between the Basin and Range and cratonic North America even more sharply than the Rayleigh wave anomalies do. The slowest velocities are found along the Rio Grande Rift, Colorado Plateau, and in the northern Basin and Range. The fast anomaly observed in the northern Cascades at 50~s persists at 100~s, suggesting that it may be associated with the subducted Juan de Fuca plate.

%%%%%%%%%%%%%%%%%%%%%%%%%%%%%%%%%%%%%%%%%%%%%%%%%%%%%%%%%%%%%%%%%%%%%%%%%%%%%%%%%%%%
\section{Conclusions}
A two-station phase-velocity measurement method that accounts for arrival-angle variations is described and tested using USArray data for the western United States. This method cancels some sources of data uncertainty associated with the single-station method, and therefore makes possible investigations of smaller phase-delay signals. The two-station observations are used to map Love and Rayleigh wave phase velocities between 25 and 100 s at a uniform resolution of approximately 200~km across the TA footprint. Arrival-angle corrections have modest, but systematic, effects on the final Rayleigh wave maps, and lead to an increase in the reduction of variance, reflecting more consistency in the measurements. 

Analysis of the Love wave data is more challenging. The original single-station data set is smaller owing to higher noise levels on the horizontal components. Additionally, the reduction in measurement uncertainty associated with construction of two-station observations is smaller than that seen for Rayleigh waves. Local phase-velocity estimates for Love waves are unusually high in comparison with two-station phase-velocity estimates.  We speculate that the effect of overtone interference is exacerbated by the mini-array measurement method. As a result, we choose not to apply the arrival-angle corrections to the Love wave measurements until undertaking further study of these effects.

Despite these difficulties, the preferred phase-velocity models for both Love and Rayleigh waves are well constrained and provide good fits to the data. This is due in large part to the grid of the TA, which allows for even path coverage, in both area and the range of path lengths, and effective azimuthal averaging. Additionally, the dense nature of the grid allows the calculation of arrival-angle estimates and associated local phase-velocity measurements, which provide independent information in addition to their utility in correcting the two-station measurements. The final phase-velocity models contain anomalies that compare well with both geologic features and with other published studies of the western US. The Love and Rayleigh wave phase-velocity maps developed here provide constraints for future studies of the radially anisotropic 3\nobreakdash-D structure of North America. 

The data and the models presented in this study are available at:\\ \url{www.ldeo.columbia.edu/~afoster/research.html}. 
 
%%%%%%%%%%%%%%%%%%%%%%%%%%%%%%%%%%%%%%%%%%%%%%%%%%%%%%%%%%%%%%%%%%%%%%%%%%%%%%%%%%%%
\section*{Acknowledgments}
The seismic waveforms used in this study come from the USArray component of the EarthScope facility. We are grateful to everyone involved in the operation and distribution of data from the Transportable Array and the regional networks that contribute to USArray. We thank the IRIS Data Management Center for providing excellent and robust methods of access to these data. We thank Associate Editor Gabi Laske, Toshiro Tanimoto, and an anonymous reviewer for helpful comments on the manuscript. This research was funded by the National Science Foundation award EAR-0952285. 


%%%%%%%%%%%%%%%%%%%%%%%%%%%%%%%%%%%%%%%%%%%%%%%%%%%%%%%%%%%%%%%%%%%%%%%%%%%%%%%%%%%%
%%%%%%%%%%%%%%%%%%%%%%%%%%%%%%%% CHAPTER 3/PAPER 2 %%%%%%%%%%%%%%%%%%%%%%%%%%%%%%%%%%%%%%%%%%%%%
\singlespacing
\chapter{Arrival-angle anomalies across the USArray Transportable Array}
\label{ch:aa}
%\addtocontents{lof}{\textbf{Chapter \thechapter: Title}}
%\addtocontents{lot}{\textbf{Chapter \thechapter: Title}}
\thispagestyle{fancy}
\doublespacing

%\footnotesize
\begin{raggedright}
{\bf Note:} A slightly modified version of this chapter has been published in Earth and Planetary Science Letters (2014), http://dx.doi.org/10.1016/j.epsl.2013.12.046
\footnote{AUTHORS: Anna Foster$^{a}$*, G\"oran Ekstr\"om$^{a}$, Vala Hj\"orleifsd\'ottir$^{b}$\\
$a$ Department of Earth and Environmental Sciences, Columbia University, 61 Route 9W, Palisades, NY 10964, USA\\
$b$ Instituto de Geof\'{i}sica, Universidad Nacional Aut\'{o}noma de M\'{e}xico, 
Ciudad Universitaria, Circuito de la Inv. Cient\'{i}fica s/n, Coyoac\'{a}n, C.P. 04510, M\'{e}xico D.F.\\
* corresponding author:  afoster@ldeo.columbia.edu}
\end{raggedright}
%\linenumbers
\normalsize

%%%ABSTRACT
\section*{Abstract}
We construct composite maps of surface-wave arrival-angle anomalies using clustered earthquakes and an array method for measuring wave-front geometry. This results in observations of arrival angles covering the entire footprint of the USArray Transportable Array during 2006--2010. Bands of arrival-angle deviations in the propagation direction indicate the presence of heterogeneous velocity structure both inside and outside of the array. We compare the observed patterns to arrival angles predicted using two global tomographic models, the mantle model S362ANI and the surface-wave-dispersion model GDM52. We use both ray-theory-based prediction methods and measurements on synthetic data calculated using a spectral-element method. Both models and all prediction methods produce similar mean arrival angles and long-wavelength patterns of anomalies which are similar to the observations. Predicted short-wavelength features generally do not agree with the observations. The spectral-element method produces some complexity that is not obtained using the ray-theory-based methods; this predicted complexity is similar in character to the observed patterns, but does not match them. 
%
%\begin{keyword}
%surface wave \sep USArray \sep phase velocity \sep arrival angle \sep western US
%\end{keyword}
%\end{frontmatter}

%\bibliographystyle{elsarticle-harv}
%\linenumbers*[1]
%%%%%%%%%%%%%%%%%%%%%%%%%%%%%%%%%%%%%%%%%%%%%%%%%%%%%%%%%%%%%%%%%%%%%%%%%%%%%%%%%%%%
\section{Introduction}
The study of surface waves has revealed significant complexity in the wave field resulting from refraction and scattering across heterogeneous velocity structures \citep[e.g.,][]{Mastersetal1984, LayKanamori1985}. The resultant deviation from the great-circle path can be measured by the arrival angle, a quantity describing the directionality of the incoming wave. A related quantity is the polarization, a measurement of both the directionality (arrival angle) and the ellipticity of the wave. Arrival-angle anomalies result from the gradient of the velocity structure they cross; the linear approximation of their sensitivity has been described by \citet{WoodhouseWong1986}, and these and other variations of the equations \citep[e.g.,][]{Larson1998} can be used to predict or invert arrival-angle anomalies.

There are two main ways in which surface-wave arrival angles can be measured: single, three-component-station methods and array-based methods. Within the former group, there exist a variety of time-domain \citep[e.g.,][]{Flinn1965,MontalbettiKanasewich1970,Vidale1986,Jacksonetal1991,Larson2002}, spectral \citep[e.g.,][]{Lerner-Lam1989,Laske1994}, and both time- and frequency-dependent techniques for measuring polarization \citep[e.g.,][]{Jurkevics1988,Paulssenetal1990}. Array-based methods are typically some variation of beamforming \citep[e.g.,][]{LevshinBerteussen1979,ZywickiRix2005,Tanimoto2007,DeCacquerayetal2011}. This is the approach we take here, using a ``mini-array'' method that fits the predicted phase from varying backazimuths to observations from a small subset of stations to find the best-fitting arrival angle.

Arrival-angle and polarization measurements are used for a variety of applications. For earthquake source studies, these include the location of earthquakes, especially small magnitude local events \citep[e.g.,][]{Ruudetal1988,BakerStevens2004}, and identification of Rayleigh waves (and other phases) and subsequent surface-wave magnitude estimation, with application to Nuclear Test Ban verification \citep[e.g.,][]{Selby2001}. Due to their sensitivity to lateral gradients in velocity, arrival-angle anomalies can be used with phase measurements to determine velocity structure. This is particularly valuable for determining small-scale structure \citep[e.g.,][]{Laske1995,LaskeMasters1996,Yoshizawaetal1999} and for discriminating between isotropic and anisotropic velocity structure \citep[e.g.,][]{Grunewald1988,LaskeMasters1998,Larson1998}. Additional applications include seismic exploration \citep[e.g.,][]{Takahashi1995}, orienting seismometer components \citep[e.g.,][]{Laske1995,Larson2002,EkstromBusby2008}, and updating or benchmarking tomographic models based on the arrival-angle predictions from synthetic data \citep[e.g.,][]{Larson2002,Jietal2005}.

The first goal of this paper is to demonstrate the robustness and consistent repeatability of arrival-angle observations made using the mini-array method across the USArray. This provides a snapshot of the wave field over the active portion of the array. The second goal is the construction of maps of arrival-angle deviations across the entire footprint of USArray, creating a more comprehensive view of wave propagation across the United States. The third goal is a qualitative characterization of the anomaly patterns and direct comparison with those predicted by current models of global 3\nobreakdash-D structure. Because arrival-angle deviations are more sensitive to small-scale structure than typically used phase or travel-time measurements, these observations and comparisons provide a different perspective on wave propagation and model accuracy. 

%%%%%%%%%%%%%%%%%%%%%%%%%%%%%%%%%%%%%%%%%%%%%%%%%%%%%%%%%%%%%%%%%%%%%%%%%%%%%%%%%%%%
\section{Method}

The mini-array method used to make arrival-angle estimates in this study requires single-station measurements of phase. We obtain these measurements following \citet{Ekstrom1997}. This method describes a seismogram with a phase, $\varphi$, and amplitude, $A$. A trial fundamental-mode model seismogram $u^M$ is calculated using the source and receiver locations, focal mechanism, and predicted phase and amplitude effects of propagation in a reference Earth model \citep[SH8U4L8;][]{DziewonskiWoodward1992}. It is then iteratively matched to the observed surface-wave signal $u^S$ by minimizing the misfit. This is initially performed at long periods, then the passband is progressively expanded to include shorter periods, which, combined with a requirement for surface-wave dispersion curves that vary smoothly with frequency, prevents cycle-skipping. Further details of the single-station method are given by \citet{Ekstrom1997}. 

For an array with dense station coverage, we use a small subset of the single-station phase data near the location of interest (a mini array) to estimate the geometry of the wave front and the direction of propagation of the wave near each station. Details of the method are presented in \citet{Fosteretal2014}; here, we provide a short summary.  

To estimate the arrival angle $\alpha^{\rm{a}}$ at a station $r^{\rm{a}}$ at a specified period, we select all $N$ receivers, $r_i$, within some radius of station $r^{\rm{a}}$. We require a minimum number of stations within the mini array to make the measurement. We vary the apparent source location along an arc, fixing the epicentral distance from $r^{\rm{a}}$ (Fig.~\ref{figaa:BFillusaa}). This effectively varies the arrival angle at station $r^{\rm{a}}$. For each trial source location $S^{\rm{a}}_j$, we consider the difference in phase between each station $r_i$ and station $r^{\rm{a}}$ to be due to the inter-station distance divided by a local phase velocity $c^{\rm{a}}_j$, plus an unknown phase offset at $r^{\rm{a}}$. We solve for the local phase velocity $c^{\rm{a}}_j$ in the least-squares sense. This results in a misfit between predicted and observed phase associated with each trial source location $S^{\rm{a}}_j$. We select the trial source location with the smallest misfit to be the best-fit apparent source location $S^{\rm{a}}_{\ast}$, which corresponds to the best-fit arrival angle at station $r^{\rm{a}}$, $\alpha^{\rm{a}}_{\ast}$. The difference between the arrival angle and the backazimuth to $S^{\rm{a}}_{\ast}$ is the arrival-angle anomaly. This process is repeated for all stations for a selected event. 
	
% 1	
\begin{figure} 
\begin{center}
\includegraphics[scale=1]{Ch3Figs/FIG1_method.pdf} 
\caption[Mini-array method illustration]{Illustration of search algorithm to find the best-fit apparent source location ($S_j$) for a given station ($r^{a}$) and source ($S_{0}$), using stations $r_i$ that fall within a specified radius. The distance between $S_{0}$ and $r^{a}$, 
$L^a_{0}$, remains fixed. Figure taken from \citet{Fosteretal2014}. }
\label{figaa:BFillusaa}
\end{center}
\end{figure}
%

%%%%%%%%%%%%%%%%%%%%%%%%%%%%%%%%%%%%%%%%%%%%%%%%%%%%%%%%%%%%%%%%%%%%%%%%%%%%%%%%%%%%
\section{Data}\label{sec:data}
We use data recorded on the USArray Transportable Array (TA) from January 2006 to December 2010. A key component of the TA program is its ``rolling'' nature; the array aims to cover the continental United States uniformly with 70\nobreakdash-km grid spacing, but does so in installments (www.usarray.org). We therefore use events recorded on an evolving array that, in aggregate, covers the area between \degree{93}--\degree{125}W longitude and \degree{25}--\degree{50}N latitude. We consider all events greater than magnitude 5.5, with focal depths less than 50 km and epicentral distances in the range of \degree{10}--\degree{165}. The examples chosen for this paper were primarily selected for the large number of high-quality single-station measurements (300--400) that could be made for each event. 

%%%%%%%%%%%%%%%%%%%%%%%%%%%%%%%%%%%%%%%%%%%%%%%%%%%%%%%%%%%%%%%%%%%%%%%%%%%%%%%%%%%%
\section{Observations}
We apply the mini-array arrival-angle measurement method to the TA data set of single-station measurements. We select a radius of \degree{1} for the mini array; this size is small enough to observe small-scale patterns, while being large enough to encompass several stations and ensure a good measurement. We require a minimum of 3 stations to make an arrival-angle estimate, with most mini-array measurements using 7--8 stations. Some measurements include as many as 12 stations. Previous studies have observed surface-wave arrival-angle anomalies of up to \degree{15} \citep{Lerner-Lam1989} and \degree{5}--\degree{30} \citep{Laske1994} in the period range used here, 25--100 s. We limit our grid search to arrival-angle anomalies of $\pm\degree{15}$, to avoid the effects of very small-scale complexity and spurious measurements. We find that most observed deviations fall within this range. 

An example of arrival-angle anomaly measurements for a single event can be seen in Figure~\ref{figaa:phasecontours}. Positive values indicate clockwise rotation from the great-circle path; negative values indicate counter-clockwise rotation. Some characteristics of the pattern are consistent for all events: as the wave field moves across the array, arrival angles are generally coherent in the propagation direction. This results in the banded appearance of the anomalies. Comparing arrival-angle deviations with contours of the single-station phase measurements for the same event (Fig.~\ref{figaa:phasecontours}), it is clear that the variations in the anomalies correspond to variations in the wave front, with arrival-angle anomalies of zero aligning more or less with peaks or valleys in the contours. Arrival angles thus provide a useful way to characterize the wave-front propagation quantitatively. 

% 2						
\begin{figure} 
\begin{center}
\includegraphics[scale=1]{Ch3Figs/FIG2_phasecontours.pdf} 
\caption[Comparison of single-station phase contours and arrival-angle anomaly measurements for a single event]{Left: Contours of single-station phase measurements (yellow lines) at all stations (black dots) recording an event located at \degree{19.65}S, \degree{168.10}E near Vanuatu Islands. Phase increases from west-southwest to east-northeast. Black lines show spherical wave fronts. Right: Example of mini-array arrival-angle anomaly estimates for the same event, using a mini-array of \degree{1} radius. Red shows angles clockwise from the great-circle-path arrival, indicated by the black lines; blue shows counter-clockwise arrivals. Both figures calculated for Rayleigh waves at 50 s period. }
\label{figaa:phasecontours}
\end{center}
\end{figure}
%

The effects of wavelength and depth sensitivity are evident when comparing measurements made at different periods (Fig.~\ref{figaa:compareperiods}). Although at all periods, the arrival-angle anomalies span the allowed range from \degree{-15} to \degree{+15}, anomalies are larger at short periods. At 25--35 s, Rayleigh waves have significant sensitivity to crustal structure. This is reflected in the complex patterns observed. At 50~s, there is less sensitivity to the crust, and the anomaly pattern is simpler. The width of the bands increases, reflecting both increasing wavelength and what we expect to be the expression of smoother mantle structure. By 100~s, the pattern is relatively simple and should be predominantly a result of mantle structure. 

% 3						
\begin{figure} 
\begin{center}
\includegraphics[scale=1]{Ch3Figs/FIG3_compareperiods.pdf} 
\caption[Arrival-angle anomaly estimates at varying periods]{Arrival-angle anomaly estimates for Rayleigh waves at 25, 35, 50, and 100 s period, for an event located at \degree{16.45}S, \degree{173.06}W near Samoa. The source is approximately \degree{85} away, with a backazimuth of \degree{243}. Black lines indicate the direction of wave propagation along the great-circle path. }
\label{figaa:compareperiods}
\end{center}
\end{figure}
%

For this paper, we choose to focus on only 50-s Rayleigh wave results. This allows us to see strong and interesting patterns in the arrival-angle anomalies across the array, without some of the complicating effects of the heterogeneous crust, though we will show that crustal structure is still important. Additionally, most current tomographic models do not include detailed shallow crustal structure, and 50-s Rayleigh waves will provide a better comparison with predictions from these models. 

%%%%%%%%%%%%%%%%%%%%%%%%%%%%%%%%%%%%%%%%%%%%%%%%%%%%%%%%%%%%%%%%%%%%%%%%%%%%%%%%%%%%
\section{Composite maps}
Observations of the arrival angle for an individual event show a range of different patterns, as well as variations within these patterns as the waves progress across the array. The TA is designed to cover the entire contiguous United States, providing a unique opportunity to observe wave-field behavior at a large scale. However, due to its rolling nature, measurements from any single event only cover a portion of the potential area. Compounding this, we cannot often make measurements for all stations in the TA for a given earthquake. The maximum number of stations at which we have measurements for a single earthquake is 436, but the average is 156. We wish to combine the observations from several events, here termed ``constituent'' events, with similar source locations that occur over the course of the TA lifetime, to take full advantage of the coverage of the array. This would result in a ``composite'' figure showing arrival-angle anomaly observations for a source location over the extent of the TA study region. In the following sections, we demonstrate the feasibility of this approach and show results for several composite examples.  

\subsection{Event consistency and selection}
The primary requirement for compiling results into composite figures is that observations are repeatable; that is, for similar events, we produce similar arrival-angle anomaly measurements on the same stations. We test this by comparing results for two earthquakes (Fig.~\ref{figaa:repeatexample}). The two earthquakes are separated in CMT location by 17 km, and in time by 9 days. 383 arrival-angle measurements were made for the event shown in Figure~\ref{figaa:repeatexample}a, and 401 measurements for the event in Figure~\ref{figaa:repeatexample}b. The results are nearly identical, with a correlation between the measured arrival angles of 95\% for the 375 overlapping stations, indicating good repeatability. 

% 4							
\begin{figure} 
\begin{center}
\includegraphics[scale=1]{Ch3Figs/FIG4_repeats.pdf} 
\caption[Arrival-angle estimates for similar event locations]{Example of arrival-angle anomaly observations from two events used in the construction of a composite map. The events are located at (a) \degree{20.36}S, \degree{168.72}E, and (b) \degree{20.23}S, \degree{168.81}E, and the results have a correlation of 95\%. Black lines indicate the direction of wave propagation along the great-circle path. }
\label{figaa:repeatexample}
\end{center}
\end{figure}
%

Such close pairs of events with nearly identical source locations are relatively rare, and therefore we cannot limit constituent events to this category. Examining large groups of events within a \degree{5} radius, we find that the range within which the arrival-angle observations are consistent varies. In many cases, events separated by \degree{3}--\degree{4} are consistent. However, for certain source areas, events separated by as little as \degree{2} show variations in the arrival-angle pattern. We therefore use a combination of proximity in earthquake source, correlation of arrival-angle deviations from events that were recorded on overlapping array configurations, and visual inspection to select constituent events. We find that source mechanism has no discernible effect on the pattern of anomalies for most events.

The remaining task for making composite figures is to find a source location with earthquakes meeting the above criteria, as well as having a reasonable distribution throughout the time period. This can be challenging, as even regions with high earthquake occurrence rates often experience quiescent periods. However, many locations do satisfy all the above criteria, particularly in the southwest Pacific. We compile all arrival-angle observations for constituent earthquakes, and average anomalies for stations with multiple observations. 

\subsection{Results}
We have constructed 10 composite maps using the above-mentioned criteria. Constituent events typically have correlation values between measured arrival-angle anomalies for overlapping events of 80\% or better. We have selected four examples to present here; locations and focal mechanisms of the central event (in color) and all constituent events (in grey) are shown in Figure~\ref{figaa:locationmap}. Composite observations are shown in Figure~\ref{figaa:composite_aa}. Locations of source areas, constituent events, and observed arrival-angle anomalies for all 10 composite maps can be found in Appendix~\ref{appendixB}.

% 5							
\begin{figure} 
\begin{center}
\includegraphics[scale=1]{Ch3Figs/FIG5_locationmaps.pdf} 
\caption[Source areas and constituent event locations and focal mechanisms for composite figures]{Top: Location map. All USArray TA stations for the time period 2006--2010 shown with black triangles. Source location area for the four example composite arrival-angle anomaly observations (Fig.~\ref{figaa:composite_aa}) shown by colored circles, with great-circle ray paths in black. Labels and colors correspond to the detailed maps at bottom, showing locations and focal mechanisms for the constituent events of composite figures from the following source areas: (a) Loyalty Islands, (b) Tonga Islands, (c) north of Ascension Island, and (d) Easter Island. Colored focal mechanism signifies the central event used in the calculation of synthetics. }
\label{figaa:locationmap}
\end{center}
\end{figure}
%
% 6							
\begin{figure} 
\begin{center}
\includegraphics[scale=1]{Ch3Figs/FIG6_compositeaa.pdf} 
\caption[Composite arrival-angle anomaly observations]{Composite arrival-angle anomaly observations for four source areas, in (a) Loyalty Islands, (b) Tonga Islands, (c) north of Ascension Island, and (d) Easter Island. Constituent event locations are shown in Figure~\ref{figaa:locationmap}. Arrival-angle anomaly estimates are averaged at each station location. Black lines indicate the direction of wave propagation along the great-circle path. For further details of the composite observations, see the text. }
\label{figaa:composite_aa}
\end{center}
\end{figure}
%

\textbf{Loyalty Islands} (Fig.~\ref{figaa:composite_aa}a): This composite figure uses 25 earthquakes, all located within \degree{2} of the central event (colored focal mechanism in Figure~\ref{figaa:locationmap}a). The earthquakes span the time period March 2007 to December 2010. This provides measurements of 1108 unique stations, covering nearly the entire study area. The source location is approximately \degree{100} away from the center of the array, with a backazimuth of approximately \degree{252}. 

Arrival-angle deviations for this composite figure range from \degree{-14} to \degree{15}. Bands of positive and negative anomalies, aligned roughly with the great-circle ray paths shown in black, are the dominant signal. For this event, wave-field propagation exhibits a range of behaviors. In the northernmost part of the study area, a wide band of strongly positive angles is very linear, and trending east-northeast. Immediately below is a narrower band of negative angles trending due east. Looking further south in the study area, the bands remain narrow, less than \degree{2} wide, and exhibit considerable curvature. Most of the bands begin at the edge and remain fairly uniform across the array, indicating that the arrival-angle anomalies originated outside of the array. 

\textbf{Tonga Islands} (Fig.~\ref{figaa:composite_aa}b): This figure is a compilation of 19 earthquakes, all within a radius of \degree{0.9} (Fig.~\ref{figaa:locationmap}b). Constituent earthquakes span the time period February 2006 through December 2010, covering 1063 unique stations. The source area is at a distance of approximately \degree{83} from the array, with a backazimuth of approximately \degree{244}. 

The great-circle path from the source area to the array is similar to that of the Loyalty Islands composite figure; despite this, the two events have very different arrival-angle anomaly patterns. For this composite figure, the arrival angles generally show smaller deviations from the great-circle path, ranging from \degree{-11} to \degree{12}. As for the Loyalty Islands, the northernmost band is wide and strongly positive, and the southern bands are narrower; however, all of the bands are linear and most of them have increasing deviations as the wave front progresses from southwest to northeast. Some are not even distinguishable as they enter the array at the west coast, which could indicate that they are due to heterogeneous structure within the array. The lack of strong anomalies at the western edge of the array may be due to the \degree{90} incidence of the wave field on the ocean-continent boundary for this source area. 

\textbf{North of Ascension Island} (Fig.~\ref{figaa:composite_aa}c): This composite figure uses only 4 earthquakes, with the farthest being \degree{1.4} from the central event (Fig.~\ref{figaa:locationmap}c). The constituent earthquakes span the time period February 2006 through November 2008, covering only 607 unique stations. The source area is at a distance of approximately \degree{91} from the array, with a backazimuth of approximately \degree{90}. This composite highlights the difficulty of finding suitable groups of events outside of the Pacific ring. Although the data do not cover the entire study region, they provide a valuable snapshot of wave propagation from the east, as well as evidence that composite figures do not require large amounts of averaging to produce robust patterns.  

In comparison with the previous two composite figures, the bands of anomalies from this source area are wider, and more negative. Arrival-angle deviations range from \degree{-15} to \degree{8.25}. Bands parallel the great-circle ray paths in the north, but are rotated clockwise in the south. Two negative bands appear to be joining up at \degree{38}N, \degree{115}W. 

\textbf{Easter Island} (Fig.~\ref{figaa:composite_aa}d): This figure is made up of results from 8 earthquakes within a radius of \degree{1.6} (Fig.~\ref{figaa:locationmap}d). Constituent earthquakes span the time period November 2006 through September 2009, yielding results at 838 unique stations. The source area is at a distance of approximately \degree{69} from the center of the array, with a backazimuth of approximately \degree{186}. 

Because the source area is directly south of USArray, we largely observe the effects of interaction with the Mexican coastline in this example. As the wave field enters the array, arrival-angle anomalies are all positive, except for the easternmost stations in Texas. As the wave field moves northward, bands of positive and negative angles appear, ranging from \degree{-14.5} to \degree{14}. These bands are very well aligned with the great-circle ray paths. The bands are \degree{3}--\degree{5} wide, wider than most of the bands in the previous examples. 

These examples illustrate the range of patterns that can be observed in the composite maps. Banded patterns may have short (\degree{2}) or long (\degree{5}) wavelengths. They may be linear or curved, and parallel the great-circle ray path or deviate from it. Additionally, the arrival-angle anomalies may most commonly be observed to originate outside the array, but some anomalies appear to be a result of heterogeneity within the array. 

%%%%%%%%%%%%%%%%%%%%%%%%%%%%%%%%%%%%%%%%%%%%%%%%%%%%%%%%%%%%%%%%%%%%%%%%%%%%%%%%%%%%
\section{Testing current tomographic models}
We use the arrival-angle observations as a metric for evaluating current velocity models. We focus on global models, since we infer that many anomalies result from structure outside of the array. In this section, we describe the velocity models used and the methods for predicting arrival-angle values, and present a comparison of the observations and model predictions. 

\subsection{Models}
\subsubsection*{S362ANI}
The radially anisotropic mantle model S362ANI \citep[][]{Kustowskietal2008} was derived using a large data set of multiple data types to ensure sensitivity at all mantle depths. The model was parameterized using 362 spherical splines to define lateral variations in shear-wave velocity, radial anisotropy, and topography of discontinuities. Sixteen B-splines, split at 650~km depth, describe isotropic velocity variations in the radial direction. The nominal lateral resolution of the model is 1000~km. We make waveform predictions using the full 3\nobreakdash-D model combined with the \degree{2}-by-\degree{2} crustal model CRUST2.0 \citep{CRUST2}. We also use phase-velocity maps derived from S362ANI with CRUST2.0 to predict surface-wave travel times and arrival angles. One version of the phase-velocity maps is expanded in spherical harmonics up to degree 40, resulting in smooth structure suitable for ray tracing. The other version of the maps is defined on \degree{2}-by-\degree{2} pixels, and expresses the full heterogeneity of CRUST2.0, making it comparable to the model used in the numerical waveform predictions. In all cases, sediments are accounted for by including sediment thicker than 2~km.

\subsubsection*{GDM52}
Global Dispersion Model GDM52 \citep[][]{Ekstrom2011} describes the Love and Rayleigh wave isotropic phase-velocity dispersion between 25--250~s, as well as the azimuthally anisotropic Rayleigh wave dispersion. A large number of earthquake phase-anomaly measurements, recorded at global stations, were inverted using a ray-theoretical framework. The model was parameterized horizontally with 1442 spherical splines and in frequency with 12 B-splines. The resulting nominal lateral resolution is 650~km. In this study, we use both the anisotropic model and the isotropic portion only to predict arrival angles. 

\subsection{Prediction methods}
\subsubsection*{Great-circle path predictions}
For a given earthquake and receiver, we calculate the predicted phase by integrating along the connecting great-circle path through a phase-velocity model. The result is a single-station phase prediction. We make arrival-angle measurements on the predicted data using the mini-array method. In this study, we make predictions for the smooth and pixel-based isotropic phase-velocity maps derived from 3\nobreakdash-D model S362ANI and CRUST2.0, and for the azimuthally anisotropic global phase-velocity model GDM52, and an isotropic version of GDM52. We note that the use of great-circle path integration may not be appropriate for use with such small-scale heterogeneities as are contained in the pixel-based phase-velocity map, and it is used here only as a demonstration of the possible anomalies resulting from small-scale structure. 

\subsubsection*{Ray tracing}
We use the exact ray-tracing algorithm presented in \citet{Larson1998}. Given an isotropic or anisotropic phase-velocity model, this method predicts the phase, arrival angle, and amplitude for a specified source and receiver location.The dynamical ray-tracing equations are derived from the surface-wave dispersion relation in an anisotropic Earth model. We perform ray-tracing through the smooth phase-velocity maps of S362ANI and GDM52. For this approach, both models are azimuthally isotropic. The output of this method is a direct prediction of the arrival angle.

\subsubsection*{SPECFEM3D Globe}
SPECFEM3D Globe is a freely available software package initially developed by \citet{KomatitschVilotte1998} and \citet{KomatitschTromp2002a,KomatitschTromp2002b}, and further developed and maintained through combined efforts with many others \citep[e.g.,][]{Tromp2010,Peter2011}. This spectral finite-element method calculates seismic wave propagation in a fully 3\nobreakdash-D Earth model. It includes effects due to lateral variations in compressional-wave speed, shear-wave speed, density, a 3\nobreakdash-D crustal model, ellipticity, topography and bathymetry, the oceans, rotation, and self-gravitation. A point source is applied at the earthquake source location, and the weak form of the wave equation is then solved for displacement at nodes in the mantle and inner core and in terms of the scalar potential in the liquid outer core. The output of this method is a synthetic waveform for each designated station location, which can be convolved with the desired source function. 

In this study, we use models S362ANI and CRUST2.0 \citep{CRUST2}, all available propagation effects (ellipticity, topography, etc.), with 5 processors and 240 surface elements for one chunk, allowing the resulting waveform to be accurate down to a period of approximately 18 seconds. The CMT solution for each earthquake is obtained from the Global CMT catalog \citep{Dziewonski1981,Ekstrom2012}. We make single-station phase measurements on the waveforms followed by mini-array arrival-angle estimates, in the same manner as with the TA data.  

\subsection{Comparisons}\label{sec:comparisons}
Comparisons between observations and predictions for the four selected events are presented here. Additional comparisons for all 10 composite events are presented in Appendix~\ref{appendixB}. 

\subsubsection*{S362ANI}
Predictions of arrival-angle anomalies from model S362ANI generally match the average direction of banding and the average sign of the observed arrival-angle anomalies (Fig.~\ref{figaa:S362ANIpredictions}). The magnitude of the arrival-angle anomalies is under-predicted by ray tracing through the smooth degree-40 phase-velocity map and by SPECFEM for events in the southwest Pacific (Fig.~\ref{figaa:S362ANIpredictions}a,b), but is very similar for the observed and predicted anomalies from events to the south and east (Fig.~\ref{figaa:S362ANIpredictions}c,d). Ray tracing produces anomalies that are very smooth, with a band width of approximately \degree{5}--\degree{12}. Great-circle path integration through the smooth degree-40 map (not shown) produces anomalies very similar to the ray-tracing results. Great-circle path integration through the pixel-based phase-velocity map produces narrowly banded anomalies with similar magnitudes to the observations. Measurements made on the SPECFEM synthetic seismograms more closely resemble the observed band widths and complexity. Correlations between the predicted arrival-angle anomalies from all methods range between 70--99\%, except for great-circle path integration through the pixel-based model, which has correlation values of 30--80\% with the other results. 

% 7							
\begin{figure} 
\begin{center}
\includegraphics[scale=1]{Ch3Figs/FIG7_S362ANI.pdf} 
\caption[Predictions of arrival-angle anomalies from model S362ANI with CRUST2.0]{Comparison of observed composite arrival-angle anomaly maps, as in Figure~\ref{figaa:composite_aa}, for source locations in (a) Loyalty Islands, (b) Tonga Islands, (c) north of Ascension Island, and (d) Easter Island, with predictions from model S362ANI. Model CRUST2.0 and sediments thicker than 2~km were also included. Predictions were made using exact ray tracing through a smooth phase-velocity map (RT, second column), great-circle-path phase integration through a pixel-based phase-velocity map (GCP, third column), and SPECFEM3D Globe with the full 3-D model (SEM, fourth column). }
\label{figaa:S362ANIpredictions}
\end{center}
\end{figure}
%

\subsubsection*{GDM52}
Predictions from GDM52 using great-circle-path integration and ray tracing are shown in Figure~\ref{figaa:GDM52predictions}, for both the anisotropic and isotropic versions of the model. In comparison with predictions from the smooth phase-velocity maps of S362ANI, predictions of arrival angles from model GDM52 show an improved match to the observations from events in the southwest Pacific (Fig.~\ref{figaa:GDM52predictions}a,b), in both sign and magnitude of the deviations. Predictions from the southern event (Fig.~\ref{figaa:GDM52predictions}d) are similar to predictions from S362ANI. Predictions from the event located to the east (Fig.~\ref{figaa:GDM52predictions}c), however, do not match the average banding direction, although the average sign is correct. The differences between the predictions from the isotropic and the azimuthally anisotropic models are of the same order as the differences resulting from the great-circle-path integration and ray-tracing prediction methods. Correlations between the predicted arrival-angle anomalies from all methods and models range between 90--99\%. 

% 8							
\begin{figure} 
\begin{center}
\includegraphics[scale=1]{Ch3Figs/FIG8_GDM52.pdf} 
\caption[Predictions of arrival-angle anomalies from model GDM52]{Comparison of observed composite arrival-angle anomaly maps, as in Figure~\ref{figaa:composite_aa}, for source locations in (a) Loyalty Islands, (b) Tonga Islands, (c) north of Ascension Island, and (d) Easter Island, with predictions from model GDM52. Predictions were made using exact ray tracing with the isotropic portion of the model (RT, second column), great-circle-path phase integration with the isotropic portion of the model (GCP, third column), and great-circle-path phase integration with the full anisotropic model (GCP, fourth column). }
\label{figaa:GDM52predictions}
\end{center}
\end{figure}
%

%%%%%%%%%%%%%%%%%%%%%%%%%%%%%%%%%%%%%%%%%%%%%%%%%%%%%%%%%%%%%%%%%%%%%%%%%%%%%%%%%%%%
\section{Discussion}\label{sec:discussion}
\subsection{Composite maps}
The construction of the composite maps of arrival-angle anomalies provides a novel way to view the regional wave field across the western half of the United States. The USArray TA is a unique type of seismic installation at present. However, these composite maps can be made for any similarly ``rolling'' array, or an array that has varied in size over time. Similar techniques could likely be used for other types of observations as well. 

Our observation that measurements from small clusters of earthquakes can be combined into a coherent pattern is evidence for smoothly varying velocity structure. The scale over which this is true, however, depends on the region; it is apparent that these patterns are not coherent for earthquakes that differ in location by more than a few degrees. This indicates that short-wavelength (\degree{2}--\degree{4}) velocity heterogeneity is important for 50-s Rayleigh waves. 

Additional information about the distribution of this heterogeneity can be inferred from the observed patterns. Many bands of anomalies clearly are well-developed at the edge of the array, indicating an origin from structure outside the array. This is consistent with observations from \citet{BungumCapon1974} and \citet{LevshinBerteussen1979} associating most refraction with continental boundaries. Structure inside the array creates additional deviations, indicated by bands of anomalies that begin or grow as the wave front progresses. One example of this can be seen in the two southwest Pacific events (Fig.~\ref{figaa:composite_aa}a,b). A strong positive anomaly to the north and a strong negative anomaly just south of the first both originate near \degree{43}N, \degree{115}W. This pattern is consistent with the expected deviations resulting from the slow velocities associated with the Snake River Plain and Yellowstone Hotspot \citep[e.g.,][]{LinRitzwoller2011} as the wave front crossing the anomaly is delayed. A second example can be seen in the Easter Island event (Fig.~\ref{figaa:composite_aa}d), where the band of negative anomalies that begins at roughly \degree{33}N, \degree{103}W may be associated with the fast velocities of the Great Plains and southern Rocky Mountains. 

\subsection{Effect of model structure}\label{secaa:effectofmodelstructure}
Based on the model predictions, then, the wavelength of heterogeneity included in a given model impacts the ability of that model to predict arrival angles. Exact ray-tracing predictions from model S362ANI (Fig.~\ref{figaa:S362ANIpredictions}, column 2) and from GDM52 (Fig.~\ref{figaa:GDM52predictions}, column 2) are both much smoother than the observed patterns. For events in the southwest Pacific, predictions from the dispersion model GDM52 are more similar to the observations, and, interestingly, smoothing the observed arrival-angle anomalies with a Gaussian function with a full width at half maximum of \degree{4.7} results in a pattern that is strikingly similar to the predictions from GDM52, as shown in Figure~\ref{figaa:lowpass}. 

% 9							
\begin{figure} 
\begin{center}
\includegraphics[scale=1]{Ch3Figs/FIG9_lowpass.pdf} 
\caption[Long-wavelength arrival-angle anomaly pattern]{The observed composite arrival-angle anomaly map for the source location in the Loyalty Islands (Fig.~\ref{figaa:composite_aa}a), left, is shown next to the smoothed observations, center. The observations are smoothed using a Gaussian function with a full width at half maximum of \degree{4.7}. The long-wavelength pattern is compared with ray-tracing predictions from model GDM52, right. }
\label{figaa:lowpass}
\end{center}
\end{figure}
%

For all composite maps, correlation values for the predicted and observed arrival-angle deviations from S362ANI are typically low, but positive (5--30\%). Correlations are lowest for measurements on SPECFEM synthetic data and predictions from the pixel-based phase-velocity map, likely because of the short-wavelength patterns that resemble the observations, but do not match them. Correlations between observed deviations and those predicted from GDM52 are higher (30--50\%). After smoothing the observations as described above, the correlations of anomalies with predictions from S362ANI increase to 5--70\%, and the correlations with predictions from GDM52 increase to 50--90\%.

The inclusion of the anisotropic structure in GDM52 has a small but noticeable effect on the predicted arrival angles (Fig.~\ref{figaa:GDM52predictions}, columns 3 and 4). It improves the predicted mean arrival-angle anomaly for the composites, typically bringing it to within \degree{0.3} of the observed mean value. Means from predictions from the isotropic version of GDM52 are typically within \degree{1.0} of the observed mean value. Mean arrival-angle anomaly values for predictions from S362ANI differ from the observed mean value by up to \degree{2.3}. 

Finally, the complicated effects of shorter-wavelength structure are illustrated in the comparison of the predictions from ray tracing through the smooth phase-velocity maps of S362ANI and CRUST2.0 with the predictions from great-circle path integration through the pixel-based phase-velocity maps from the same models, and the predictions based on the numerical waveforms (Fig.~\ref{figaa:S362ANIpredictions}). The latter two models retain all of the short-wavelength heterogeneity of CRUST2.0. The pattern resulting from the pixel-based model predictions has very narrow bands of anomalies. These anomalies do not match the overall observed anomaly patterns, but do strongly resemble them. This appears to indicate that crustal structure at the scale included in CRUST2.0 is important for predicting arrival angles at small scales; however, this same structure is included in the model used to make waveform predictions, and does not produce the same strong narrow bands of anomalies with this more accurate prediction method. This indicates that current models of small-scale velocity structure require modifications to produce better predictions of arrival-angle anomalies. 

\subsection{Limitations of prediction methods}
Model predictions made using great-circle-path phase integration and ray tracing typically capture the general character of the arrival-angle anomaly observations, though not their complexity or range of deviations. In contrast, measurements made on synthetic data calculated using SPECFEM have the same type of complexity as the observed measurements, though not the exact details of the pattern. This indicates that ray-theory-based prediction methods are not adequate to explain the small-scale intricacies of the wave field, and are likely not the best choice for including arrival-angle anomaly measurements for applications such as improving tomographic models. This is not a new problem, as \citet{LaskeMasters1996} pointed out that as models include higher-order structure, great-circle approximations and ray tracing may both be inadequate for shorter-period predictions. As mentioned in Section~\ref{secaa:effectofmodelstructure}, this is illustrated by the differences in the predictions based on the great-circle path integration of the pixel model of S362ANI with CRUST2.0 and those based on the synthetic waveforms calculated for the same model. 

On the other hand, predictions from great-circle-path integration of phase, ray tracing, and measurements from SPECFEM synthetics are remarkably similar in the large-scale patterns (Fig.~\ref{figaa:S362ANIpredictions}). The predictions from SPECFEM synthetics appear to be a smoother version of the great-circle path predictions from the pixel-based phase-velocity map, perhaps due to wave-front healing and interference effects that are not accounted for in the integration method. The ray-tracing and great-circle path predictions from the smooth phase-velocity map appear to be a smoother version of the predictions from the SPECFEM synthetics. The mean predicted arrival-angle deviations for the composite maps from each method are typically within \degree{0.5}, and the predicted arrival-angle anomaly maps typically have correlations of 80--95\% with each other, with the exception of the predictions from the pixel-based map. These large-scale similarities indicate that ray-theory-based methods are adequate for applications in which the wavelength of heterogeneity of interest is long relative to the surface-wave wavelength. 

The measurements made using synthetic data from SPECFEM3D Globe allow us to infer better which limitations result from the methods and which from the models. Due to requirements of smooth gradients in the model, the ray-tracing method cannot predict short-wavelength anomalies that might result from strong crustal heterogeneity, for example. Great-circle-path integration can be used with more heterogeneous models so long as the size of the heterogeneity is long compared to the wavelength, though it lacks some of the wave propagation effects we know to be important. These effects, included in SPECFEM, result in the complex character of the banding that can be observed in the predicted arrival-angle anomaly patterns. However, comparisons to the measurements on synthetic data show that these ray-theoretical methods can predict anomalies at a scale of roughly \degree{5} or more. The discrepancies between observations and predictions at long wavelengths, as in Figure~\ref{figaa:lowpass}, are therefore attributable to the models. This type of comparison highlights areas in which future models could be improved, in order to predict wave propagation in a heterogeneous Earth more accurately. 

%%%%%%%%%%%%%%%%%%%%%%%%%%%%%%%%%%%%%%%%%%%%%%%%%%%%%%%%%%%%%%%%%%%%%%%%%%%%%%%%%%%%
\section{Conclusions}
On a large and regular array such as the Transportable Array, arrival angles can be systematically measured using the mini-array technique. Arrival-angle anomalies are very sensitive to source location. For earthquakes within a source area of roughly \degree{2} radius, consistent arrival-angle anomaly patterns are produced; beyond this, velocity heterogeneity results in very different patterns. Using the observations from earthquakes within the same source region, measured anomalies can be combined into composite figures that describe the wave field over a larger area.

The resulting maps provide a snapshot of wave propagation across the array. Many of the large-scale bands of arrival-angle deviations appear to result from structure outside of the array. Much of the complexity in the wave field is thus acquired before reaching North America, and care must be taken when attributing it to structure within the array. Composite maps also show the evolution of arrival-angle anomaly patterns inside the array. This is the result of velocity structure in North America, and may be attributable to specific velocity anomalies. 

Predictions of arrival-angle anomalies from models S362ANI and GDM52 display the general characteristics of the observed arrival-angle anomaly patterns. Measurements of synthetic waveforms calculated using SPECFEM produce the most realistic patterns, though none of the methods or models produces the full range of deviations or the complex variations in the pattern. The phase-velocity structure in GDM52 produces better predictions of the observed anomalies than model S362ANI. Including the anisotropy in GDM52 improves the mean arrival-angle anomaly for the array, with small changes in the anomaly pattern. Including crustal structure at high resolution is important for reproducing the type of strong narrow bands of anomalies seen in the data. The comparisons indicate that current global models contain the necessary structure to produce the long-wavelength patterns in the observed arrival-angle anomalies, but not the small-scale velocity anomalies necessary to replicate the observed arrival-angle deviations. Comparisons with full synthetics are needed to use small-scale arrival-angle anomaly agreement as a benchmark for model fidelity. 

%%%%%%%%%%%%%%%%%%%%%%%%%%%%%%%%%%%%%%%%%%%%%%%%%%%%%%%%%%%%%%%%%%%%%%%%%%%%%%%%%%%%
\section*{Acknowledgments}
We are grateful to everyone involved in the deployment and operation of USArray. The consistent and high quality of the data from the array made the research here possible. We also thank the IRIS DMC for providing simple and efficient access to these data. We thank Editor Peter Shearer, an anonymous Guest Editor, an anonymous reviewer, and Gabi Laske for helpful comments on the manuscript. This research was supported by the EarthScope Program of the National Science Foundation, grant EAR-0952285.  


%%%%%%%%%%%%%%%%%%%%%%%%%%%%%%%%%%%%%%%%%%%%%%%%%%%%%%%%%%%%%%%%%%%%%%%%%%%%%%%%%%%%
%%%%%%%%%%%%%%%%%%%%%%%%%%%%%%%% CHAPTER 4/PAPER 3 %%%%%%%%%%%%%%%%%%%%%%%%%%%%%%%%%%%%%%%%%%%%%
\singlespacing
\chapter{Overtone interference in array-based Love wave phase measurements}
\label{ch:ot}
%\addtocontents{lof}{\textbf{Chapter \thechapter: Title}}
%\addtocontents{lot}{\textbf{Chapter \thechapter: Title}}
\thispagestyle{fancy}
\normalsize
\doublespacing

\begin{raggedright}
{\bf Note:} This chapter has been submitted to the Bulletin of the Seismological Society of America
\footnote{AUTHORS: Anna Foster$^{a}$*, Meredith Nettles$^{a}$, G\"oran Ekstr\"om$^{a}$\\
$a$ Department of Earth and Environmental Sciences, Columbia University, 61 Route 9W, Palisades, NY 10964, USA\\
* corresponding author:  afoster@ldeo.columbia.edu}
\end{raggedright}
%\linenumbers
\normalsize

%%%%%%%%%%%%%%%%%%%%%%%%%%%%%%%%%%%%%%%%%%%%%%%%%%%%%%%%%%%%%%%%%%%%%%%%%%%%%%%%%%%%
\section*{Abstract}
We examine the effect of overtone interference on fundamental-mode Love wave phase measurements made using single-station and array-based techniques. For single-station teleseismic measurements on USArray Transportable Array data, the contamination effects are small, less than 1\% of the phase velocity, consistent with previous studies. Single-station amplitude measurements provide complementary constraints on the interference pattern. For array-based measurements on the same data set, contamination effects are much larger: up to $\sim$10\% of the phase velocity for two-station measurements and up to $\sim$20\% for mini-array measurements. The interference pattern for single-station measurements can largely be explained by interactions between the fundamental mode and perturbing first-higher mode. This interpretation is confirmed using measurements on both mode-summation synthetic waveforms for a 1\nobreakdash-D Earth model and synthetic waveforms calculated using SPECFEM3D Globe and a 3\nobreakdash-D Earth model. Array-based phase measurements are calculated using gradients of the single-station phase delay, and we demonstrate that the overtone interference pattern for array-based measurements can be approximated using gradients of the single-station interference pattern. This relationship can lead to an overall bias to higher phase velocities when combined with some common quality-selection and data-reduction procedures for array measurements. Our results indicate that array-based Love wave phase measurements must be carefully scrutinized for overtone contamination, and suggest the possibility of new approaches for measuring overtone phase velocities. 

%%%%%%%%%%%%%%%%%%%%%%%%%%%%%%%%%%%%%%%%%%%%%%%%%%%%%%%%%%%%%%%%%%%%%%%%%%%%%%%%%%%%
%%%
\section{Introduction}
Fundamental- and higher-mode surface-wave phase velocities provide important information on Earth structure. Fundamental-mode surface waves provide constraints on lateral heterogeneity in the uppermost mantle. Higher-mode surface waves are particularly useful for constraining deeper Earth structure, including the low-velocity zone and transition zone \citep{LaskeWidmer2007}. Because of the differing particle motion for Love and Rayleigh waves, such observations can also be used to constrain radial anisotropy. However, obtaining high-quality measurements can be especially difficult for Love waves, due to misorientation of seismometers, high noise levels on the horizontal components, relatively weak dispersion, and the difficulty of separating signals from individual mode branches. In this study, we focus on the latter difficulty. 

Surface-wave phase-measurement methods typically attempt to isolate the signal of a desired mode, wave type, and period, to gain information that is attributable to Earth structure in a known way. Signals of the same wave type and similar period can be difficult to separate, and if it is not possible to isolate the desired mode, the summed waves interfere and measurements not accounting for such interference may be contaminated. Several methods have been developed to reduce measurement contamination due to mode interference. Among these are phase-matched filtering \citep[e.g.,][]{HerrinGoforth1977, HerrinGoforth1979}, time-variable filtering \citep[e.g.,][]{Cara1973, Dziewonskietal1969, Landismanetal1969, Pilant1964, Knopoff1966}, simultaneous estimation of the fundamental and first-overtone phase velocities \citep[e.g.,][]{Forsyth1975}, spatial filtering via array stacking \citep[e.g.,][]{Nolet1975, NoletPanza1976, Cara1978, OkalJo1987}, mode-branch stripping \citep[e.g.,][]{vanHeijstWoodhouse1997, vanHeijstWoodhouse1999}, branch cross-correlation with single-mode synthetics \citep[e.g.,][]{LernerLamJordan1983, CaraLeveque1987} and waveform inversion to retrieve path-average multimode phase dispersion \citep[e.g.,][]{GeeJordan1992, Gahertyetal1996, YoshizawaKennett2002modes}. Some waveform modeling approaches bypass the need to separate signals for individual phase measurement, and instead account for the sum of the signals \citep[e.g.,][]{WoodhouseDziewonski1984, StutzmannMontagner1993}. Other studies retrieve phase velocity and attenuation information using a spectral approach \citep[e.g.,][]{JobertRoult1976, Roultetal1990, RoultRomanowicz1984, OkalJo1987}. 

Many studies have successfully applied the aforementioned methods to make fundamental- and higher-mode Rayleigh wave phase measurements, but fewer have been successful for Love waves, a difficulty foreshadowed in the first observations of higher-mode dispersion \citep{OliverEwing1957}. This difficulty is due to the similar group velocities at which the fundamental- and higher-mode Love waves travel in the Earth, particularly at intermediate periods (Fig.~\ref{figot:grpvel}). The problem has long been known to be especially acute for oceanic velocity structure, which leads to a large frequency band in which the fundamental-mode and first-overtone group velocities overlap (Fig.~\ref{figot:grpvel}), resulting in more interference than for continental structure \citep{Thatcher&Brune1969, Forsyth1975, Nettles2011}. The level of interference also depends on the relative excitation of the different modes at the source, which is affected by the fault type, focal depth, velocity structure at the source, frequency, and take-off azimuth \citep{Jobert1964, Yoshida1983}. Though for most shallow earthquakes, the fundamental mode is the most strongly excited, dip-slip faults may in some cases produce equal or larger first-overtone amplitudes \citep{FukaoAbe1971, Yoshida1983}. Deep earthquakes, typically those with depths greater than 150~km, excite overtones more strongly than the fundamental modes \citep{LaskeWidmer2007}.

% 1						
\begin{figure} 
\begin{center}
\includegraphics[scale=1]{Ch4Figs/groupvelcurves} 
\caption[Love wave group velocity curves]{Group velocity curves for Love waves in (top) continental crustal structure G6 \citep{CRUST2}, and (bottom) oceanic crustal structure A0 \citep{Mooneyetal1998,CRUST2}, both with PREM mantle. Solid lines show dispersion for fundamental-mode surface waves, dashed lines show dispersion for the first overtone, and dotted lines show dispersion for the second overtone. }
\label{figot:grpvel}
\end{center}
\end{figure}
%

In fundamental-mode phase measurements, data selection is commonly used to minimize overtone interference: the data set is restricted to only shallow earthquakes, or to long paths that allow more time for the signals to disperse \citep{Knopoff1972}. A less desirable approach avoids Love wave measurements entirely in the period range most affected. Measurement methods for obtaining fundamental-mode phase at a single station are generally also designed to reduce the negative effects of overtone interference by incorporating some of the approaches described above. Though individual measurements may be affected by overtone interference, the effect varies with distance and a data set with a good path-length distribution should not generate biased results \citep{Boore1969}. Global averages from Love wave phase-velocity measurements and toroidal-mode frequencies are consistent, indicating no bias at this large spatial scale \citep{LaskeWidmer2007}.  \citet{Nettles2011} (hereafter referred to as ND2011) showed that single-station measurements from methods similar to that of \citet{Ekstrom1997}, which utilizes phase-matched filters, can be affected by overtone interference, but that the effect is typically much smaller than the signal of interest, and is averaged out in a global tomographic inversion. Predicted errors from contamination for single-station phase-velocity measurements in that study were less than 1\% for periods of 35--150 s.  

Array-based phase-measurement methods have not been systematically tested for bias in this way. This is particularly relevant as such methods increase in popularity along with the availability of large, high-quality array data sets like those from the USArray Transportable Array and other large temporary deployments. With the exception of the array-stacking techniques mentioned above, which were designed to minimize overtone-interference effects, many array-based methods were designed to reduce other types of measurement errors and do not account for overtone-interference effects. Although \citet{Boore1969} found that synthetically calculated inter-station phase velocities for a sufficiently random distribution of earthquakes should not be biased on average, he predicted perturbations to two-station phase-velocity measurements in the range of 5--7\% for periods between 30--90 s and a relative excitation of 0.5 for the first overtone. \citet{James1971} found errors in inter-station phase velocity of roughly 7\% for a single earthquake at 80--100 s by comparing the results from the G1 and G3 waves.  \citet{PoletKanamori1997} made single-station phase measurements that they then combined across the small TERRAscope array to obtain the regional phase velocity; from comparisons with measurements on mode-based synthetic seismograms, they estimated a bias of roughly 3--7\% in the period range 50--150 s. While this overall bias was likely due in part to the limited path-length distribution used, the role of the array method in producing this bias has not been discussed. 

A specific motivation for this study is a discrepancy we observed \citep{Fosteretal2014} between phase-velocity maps derived using two different array-based methods. In spite of careful averaging of observations, we observed a clear bias for Love waves at longer periods (50--100 s), in which measurements from a mini-array method similar to beamforming consistently showed higher phase velocities than estimates from a two-station method. This bias was not observed for Rayleigh waves, and \citet{Fosteretal2014} hypothesized that the cause might be overtone interference. 

In this paper, we analyze a large data set of observations from the USArray, along with synthetic measurements, to evaluate the effects of overtone interference on estimates of fundamental-mode Love wave phase velocities. We show that the effects of overtone interference on single-station phase measurements on the USArray data set are consistent with the results of ND2011, and expand the study to include the multiple-station methods used in \citet{Fosteretal2014}: a two-station method and an array-fitting method comparable to the method used in \citet{PoletKanamori1997}.  We demonstrate that overtone interference can be a significant source of error in array-based Love wave phase measurements. We argue, based on comparisons with simple theory, that for shallow earthquakes most of the contamination arises from interference of the first-higher mode with the fundamental mode. 

%%%%%%%%%%%%%%%%%%%%%%%%%%%%%%%%%%%%%%%%%%%%%%%%%%%%%%%%%%%%%%%%%%%%%%%%%%%%%%%%%%%%
%%%
\section{Theory}
A Love wave seismogram $S(\omega)$ represented as a sum of $N$ dispersed surface-wave mode branches $n=0,1,2,...,N$ can be written 
\begin{equation}\label{eqot:1}
S(\omega) = \sum_{n=0}^{N} A_{n}(\omega) \exp \left[ \frac{i \omega X}{c_{n}(\omega)} \right],
\end{equation}
where $A_n(\omega)$ represents the complex amplitude of each mode branch at frequency $\omega$, $c_{n}(\omega)$ is the phase velocity, and $X$ is the epicentral distance. The interference issue discussed in this study relates to the difficulty of isolating the fundamental-mode ($n=0$) seismogram and measuring its amplitude and phase. When group velocities of other mode branches ($n\geq1$) at a given frequency are similar to that of the fundamental mode, contamination of  measurements of the fundamental-mode phase and amplitude can result.

The effects of the overtone contamination on any dispersion or amplitude measurement will depend on the proximity of the mode branches, as well as details of the specific measurement technique. In general, we expect the measured fundamental-mode phase and amplitude at a given frequency to be different from the true values by some amount that is related to the relative phase and amplitude of the overtone branches at that frequency. For shallow sources, the fundamental mode will dominate, and the overtones are expected to modulate the amplitude and phase of the signal by a small amount (e.g., ND2011).

A simple conceptual model for the measured fundamental-mode phase velocity and amplitude at a given frequency can be written
\begin{equation}\label{eqot:2}
(1+\delta a) A_0(\omega) \exp \left[ \frac{i \omega X}{c_0(\omega) + 
\delta c(\omega)} \right] \sim A_0 \exp \left[ \frac{i \omega X}{c_0(\omega)}  \right] + \sum_{n=1}^{N} b_n A_{n}(\omega) \exp \left[ \frac{i \omega X}{c_{n}(\omega)} \right],
\end{equation}
where $\delta a$ and $\delta c$ refer to the perturbation away from the true values, and the coefficients $b_n$ represent the level of contamination by a given mode branch.

In order to assess how overtone interference may manifest itself in the real data, especially as a function of epicentral distance, we perform experiments on synthetic seismograms. These experiments extend those previously performed and described by ND2011, and are aimed at understanding the interference patterns seen in two-station and local-array measurements of Love wave dispersion.

We generate synthetic seismograms using the Preliminary Reference Earth Model \citep[PREM;][]{PREM} and normal-mode summation for a shallow strike-slip earthquake. Stations are located at different distances in the direction of maximum Love wave radiation.  Three sets of seismograms are generated. The first set contains only the fundamental mode, the second set contains the fundamental mode and the first overtone, and the third set contains the full catalog of modes. Phase-velocity and amplitude measurements are made using the same tool as that we use to measure real data \citep{Ekstrom1997}. 

Figure~\ref{figot:theoretical}a shows the results for the dispersion measurements for the different sets of synthetic seismograms plotted as a function of distance, for Love waves with a period of 100 s. The quantity shown is the phase anomaly $\delta \varphi$ corresponding to the measured phase-velocity perturbation $\delta c$ with respect to the ray-theoretical prediction based on WKBJ theory. Measurements made on the fundamental-mode-only synthetics (gray) show very small deviations ($< 0.2$ rad) from the expected value. At small distances from the source and close to the antipode, the deviations grow, reflecting the limitations of the ray-theoretical approximation. Measurements for the other two sets of synthetics (blue and green) show oscillating anomalies with a wavelength of approximately \degree{20}. Oscillations in the phase with distance have a saw-toothed pattern. A decrease in the contamination is seen with distance, reflecting the growing difference in group arrival time for the fundamental mode and the first overtone. The similarity of the two curves indicates that only the first overtone is important for the interference, and suggests that the main oscillatory pattern can be modeled by considering only the modulation of the phase by the interference of the first overtone. The figure also shows the prediction based on Equation~\ref{eqot:2} (red), using $N = 1$ and assuming a constant ratio $\frac{b_1A_1}{A_0} = 0.3$. The resulting prediction, as a function of distance $X$, captures the observed oscillation of the phase anomaly well. Our results are similar for other geometries of the source focal mechanism, and for other frequencies in the range 25--150 s. 

% 2
\begin{figure}
\centerline{
\includegraphics[scale=1.0]{Ch4Figs/theoryexample}
}
\caption[Theoretical overtone interference in single-station phase and amplitude measurements]{\label{figot:theoretical}
(a) Phase shift measured at 100-s period from synthetic Love wave seismograms, as a function of epicentral distance, for a shallow strike-slip earthquake observed at evenly spaced stations in the maximum of the radiation pattern. Gray triangles show results for synthetic seismograms containing only the fundamental mode; green triangles show results for synthetic seismograms containing the fundamental mode and first overtone; blue triangles show results for synthetic seismograms containing all modes. The red triangles represent predictions from a model that includes only the interference of the fundamental mode and first overtone, and an amplitude ratio between the two of 1/0.3. (b) Amplitude anomaly measured from synthetic seismograms, and predicted from the interference of the fundamental and first overtone. Symbols as in (a).
}
\end{figure}
%

Figure~\ref{figot:theoretical}b shows the corresponding amplitude measurements. The quantity shown is the log of the ratio of the measured amplitude with respect to the WKBJ theoretical prediction. The patterns are consistent with those in Figure~\ref{figot:theoretical}a, with the peaks in amplitude and phase offset by approximately $\frac{\pi}{2}$~rad. Peak amplitudes occur at locations between advanced phase at shorter distances and delayed phase at longer distances, while minimum amplitudes occur between delayed phase at shorter distances and advanced phase at longer distances. The addition of the higher modes after the first (blue) contributes to a slight asymmetry in the opposite sense from the phase anomalies, but again only the first overtone is important for generating the main oscillation in the anomaly pattern. Predictions based on Equation~\ref{eqot:2} (red) again explain the observed oscillation well.

Multiple-station methods rely on differences in phase over short distances. For a steeply oscillating interference pattern like that predicted for single-station measurements, differential  measurements from multiple-station methods might be expected to have larger errors. We investigate this possibility using observed data and synthetic seismograms incorporating higher-order wave-propagation effects. 

%%%%%%%%%%%%%%%%%%%%%%%%%%%%%%%%%%%%%%%%%%%%%%%%%%%%%%%%%%%%%%%%%%%%%%%%%%%%%%%%%%%%
%%%
\section{Data and methods}
We compare the effects of overtone interference on three different types of surface-wave phase-velocity-measurement methods: single-station, two-station, and mini-array. We briefly review these methods, and refer readers to the cited papers for details. 

Data used in this study were recorded on the USArray Transportable Array from January 2006 through December 2012 and were accessed through the IRIS database. During this time period, the array integrally covered the region between \degree{70}--\degree{125}W longitude and \degree{25}--\degree{50}N latitude, with roughly 400 stations typically operating at any given time and each station operating for approximately two years. We restrict source earthquakes to those shallower than 50 km depth, with a magnitude of 5.5 or greater. We consider epicentral distances in the range \degree{10}--\degree{165}. This results in more than 1600 events for which we can make Love wave measurements. 

Single-station phase and amplitude measurements are made using the method of \citet{Ekstrom1997}. This allows us to compare our results with those of ND2011, and provides a baseline for comparison with multiple-station measurements. The measurement method was designed to reduce the effects of overtone interference \citep{Ekstrom1997, Nettles2011}. For each measurement, beginning at longer periods, a model seismogram is constructed and used to create a whitening phase-matched filter. This filter is cross-correlated with the observed band-passed seismogram, and phase and amplitude are iteratively adjusted until a sufficiently delta-like cross-correlation function is achieved. The procedure yields phase and amplitude anomalies with respect to the starting model, and the process is continued with the same seismogram for shorter periods, with the requirement that the dispersion curve remain smooth. In this study, we accept only the highest-quality measurements, resulting in 1586 unique station locations for which we have collected Love wave measurements, and up to 56,116 total phase measurements at a single period. The amplitude data set contains up to 43,215 measurements for a single period. 

Two-station phase measurements are calculated using the differencing method presented in \citet{Fosteretal2014}. Single-station phase measurements at two stations recording the same earthquake and lying roughly on the same great-circle path to the event are differenced, and the inter-station phase velocity is calculated. The method accounts for small variations in the alignment of the inter-station path with the great-circle path to the event, but not for off-great-circle arrivals. The angular difference between the inter-station path and the great-circle path to the farther station is restricted to $\leq \degree{5}$. At a single period, this results in up to 66,312 measurements with path lengths in the range 350~km--750~km, the same range used by \citet{Fosteretal2014}. 

The mini-array approach estimates the local phase velocity and arrival angle using single-station phase measurements from multiple nearby stations \citep{Fosteretal2014,Fosteretal2014aa}. The estimates are made by selecting all stations within a \degree{2} radius, and determining the local phase velocity and arrival angle that best predict the observed phase at all stations within the mini array. Mini-array measurements with the largest misfits are discarded, as in \citet{Fosteretal2014}, resulting in up to 41,951 acceptable mini-array measurements at a single period. 

These three phase-velocity data sets are updated versions of the data sets used in \citet{Fosteretal2014}, which used TA data through December 2010. We also consider single-station amplitude data updated from measurements collected by \citet{EddyEkstrom2014}. Based on the group velocity curves shown in Figure~\ref{figot:grpvel}, we expect that all our Love wave measurements in the period range 25--150~s may be affected by overtone interference. For simplicity, we show only results from Love waves at 100-s period in this study. We observe similar patterns at other frequencies in the range we study, with the effects most pronounced near $\sim$100~s. 

%%%%%%%%%%%%%%%%%%%%%%%%%%%%%%%%%%%%%%%%%%%%%%%%%%%%%%%%%%%%%%%%%%%%%%%%%%%%%%%%%%%%
%%%
\section{Observations of overtone interference}
\subsection{Single-station measurements}
The effect of overtone contamination on single-station phase measurements made with modern waveform-fitting methods \citep[e.g.,][]{Trampert1995, Ekstrom1997} is expected to be small in comparison with the variations in phase due to 3\nobreakdash-D structural heterogeneity in the Earth. Consistent with this expectation, we do not observe an interference pattern with distance when the phase-velocity measurements are plotted with respect to predictions from a 1\nobreakdash-D Earth model like PREM. Single-station phase-velocity measurements for stations of the USArray TA vary within $\pm5\%$ at 100-s period and from $+3\%$ to $-15\%$ at shorter periods with respect to PREM-predicted values.   

However, once long-wavelength structure is accounted for, an oscillatory pattern similar to that predicted in Figure~\ref{figot:theoretical}a is apparent in the residual signal. In Figure~\ref{figot:dETL}a, we show 100-s Love wave phase-velocity measurements with respect to predictions from the global dispersion model GDM52 \citep{Ekstrom2011}, which has a nominal lateral resolution of 650~km. The values predicted by integrating along great-circle paths through GDM52 account for the majority of the variations due to structure, and the residual phase-velocity values display the effects of overtone contamination. Similar effects are observed for Love waves at other periods in the range 25--100 s. This type of residual pattern is not observed for Rayleigh waves. 

% 3						
\begin{figure} 
\begin{center}
\includegraphics[scale=.155]{Ch4Figs/hist_nummeas_L100_v2} 
\caption[Overtone interference in single-station phase-velocity and amplitude measurements]{(a) 100-s Love wave single-station phase-velocity measurements with respect to phase velocity predicted from GDM52. (b) 100-s Love wave single-station amplitude anomaly measurements, as the natural logarithm of the ratio with the predicted amplitude from PREM. Measurements updated from \citet{EddyEkstrom2014}. (c) Number of single-station phase measurements used in this study by epicentral distance for 100 s Love waves, in \degree{2}-wide bins. }
\label{figot:dETL}
\end{center}
\end{figure}
%

Consistent with the results of ND2011, overtone contamination in single-station Love wave phase measurements on the TA data set produces small oscillations, typically less than $\pm1\%$, about the expected value. The variations decrease with increasing epicentral distance. As in ND2011, these oscillations should average out if a sufficiently wide distance range is used, and should not bias global tomographic inversions. The mean value of the residuals is $-0.0002$\%. 

The interference pattern with distance is shown for the single-station amplitude anomalies in Figure~\ref{figot:dETL}b). The plotted values are the log of the ratio between the measured amplitude and that predicted using PREM. The amplitude anomalies therefore have been corrected for effects of instrument response functions, source radiation patterns, and propagation on a reference spherical Earth. Remaining variations are due to unmodeled receiver effects, focusing, and lateral variations in attenuation. Overtone contamination adds another effect that can be observed as an oscillation around zero. As predicted, the amplitude interference pattern is out of phase with the phase-velocity interference pattern by roughly $\frac{\pi}{2}$~rad. 

It has been suggested that data quality selection, including discarding low-amplitude signals, may result in a data set biased towards high Love wave velocities \citep{Thatcher&Brune1969}. Examining the number of high-quality, hence acceptable, measurements we obtain as a function of epicentral distance (Fig.~\ref{figot:dETL}c) shows that our TA data set is dominated by measurements with epicentral distances of \degree{80}--\degree{110}, a consequence of the global earthquake distribution and the array location. However, we also observe a slight oscillation in the number of measurements, especially  at shorter distances. Distance ranges with increased numbers of measurements correspond approximately to the zero crossings of the observed interference pattern in phase. At all distance ranges there are at least 250 measurements in each \degree{2}-wide bin, and the effect should, if anything, increase the data set in favor of less contaminated measurements. For single-station measurements, data selection is unlikely to lead to a bias.

In the simplest case of a spherically symmetric Earth, the interference pattern we expect to find in phase-velocity measurements is a clear oscillation (Fig.~\ref{figot:theoretical}a). In the real Earth, the observations exhibit scatter about the oscillatory pattern (Fig.~\ref{figot:dETL}a). The scatter is the result of several factors: lateral heterogeneity in fundamental-mode phase velocity not accounted for in our reference model; lateral heterogeneity in the overtone phase velocity which, combined with the fundamental-mode phase velocity, controls the wavelength of the interference pattern; and variations in the relative mode excitation, which controls the amplitude of the interference pattern. We illustrate these variations by selecting subsets of earthquake-station pairs with similar paths. Such subsets usually produce well-defined interference patterns which, for different paths, are slightly offset from each other. Figure~\ref{figot:dETL_sel}a shows the interference patterns for 6 such subsets. Great-circle paths and earthquake locations are shown in Figure~\ref{figot:dETL_sel}b. We note that, although continental paths are generally expected to be less affected by overtone interference than oceanic paths, the Baja event (orange), with a largely continental path, shows interference at a level similar to the mixed and oceanic paths. 

% 4					
\begin{figure} 
\begin{center}
\includegraphics[scale=1.0]{Ch4Figs/selected_dETL} 
\caption[Selected earthquake subsets of overtone interference in single-station phase-velocity measurements]{(a) Colored dots represent single-station phase-velocity measurements with respect to phase velocity predicted from GDM52, as in Figure~\ref{figot:dETL}, for selected stations and events that lie roughly on the same great-circle path. Horizontal axis shows epicentral distance. Black dots in background show all single-station phase-velocity variations. (b) Map showing the station locations (black circles), earthquake locations (colored circles), and approximate great-circle paths between the two (colored lines) for the selected measurements shown in (a). }
\label{figot:dETL_sel}
\end{center}
\end{figure}
%

\subsection{Two-station measurements}
Because two-station measurements are constructed from differential measurements over small distances, errors due to overtone contamination might be expected to be larger than those observed in the single-station measurements. To examine the effect on two-station measurements, we compare the observations to predictions from a model incorporating long-wavelength heterogeneity, as we did for the single-station measurements. We again use model GDM52 as our reference, and plot our observed inter-station phase velocities at the epicentral distance corresponding to the midpoint of the inter-station path. We show the residual inter-station phase velocity with respect to this model-predicted value for all two-station measurements with an inter-station path length between 350--750 km in Figure~\ref{figot:darray}a. The full data set of two-station measurements includes path lengths of 100~km to greater than 1500~km. Including all path lengths produces a similar oscillation pattern, but with additional scatter. The mean value of the residuals shown in Figure~\ref{figot:darray}a is $0.42$\%.

% 5						
\begin{figure} 
\begin{center}
\includegraphics[scale=.155]{Ch4Figs/L100_dtsm_dmini} 
\caption[Overtone interference in two-station and mini-array phase-velocity measurements]{(a) Two-station phase-velocity measurements with inter-station path lengths of 350--750~km, with respect to the inter-station phase velocity predicted from GDM52. (b) Mini-array phase-velocity measurements with respect to the local phase velocity predicted from \citet{Fosteretal2014}. Horizontal axis shows epicentral distance. }
\label{figot:darray}
\end{center}
\end{figure}
%

The resulting interference pattern is very similar in character to that of the single-station measurements. It oscillates about the predicted value with a slightly saw-toothed shape. The peaks in the two-station residuals align approximately with the peaks in the single-station residuals (Fig.~\ref{figot:dETL}a). However, the amplitude of the interference signal does not decrease as epicentral distance increases. The most important difference between the two interference patterns is the magnitude: while single-station measurement variations were in the range of $\pm1\%$, the two-station measurement variations are in the range of $\pm10\%$. 

\subsection{Mini-array measurements}
Analogous to the two-station measurements, mini-array measurements also depend on differential values, in this case, over even smaller distances (less than \degree{2}). We obtain expected values for each mini-array measurement from the regional phase-velocity maps in \citet{Fosteretal2014}, which have a nominal lateral resolution of $\sim$150~km. We sample the phase-velocity map at the point at which the mini-array measurement is made, corresponding to the central station in the mini array. The mini-array phase-velocity anomalies with respect to the predicted phase velocities are shown in Figure~\ref{figot:darray}b, as a function of epicentral distance. The mean value of the residuals is $0.48$\%. The overtone interference pattern again shows a saw-toothed oscillation about the expected value, with peaks roughly aligned with the peaks in the single- and two-station measurement plots. The magnitude of the oscillations is even larger than it was for the two-station measurements, reaching $\pm20\%$ of the reference velocity. 

%%%%%%%%%%%%%%%%%%%%%%%%%%%%%%%%%%%%%%%%%%%%%%%%%%%%%%%%%%%%%%%%%%%%%%%%%%%%%%%%%%%%
%%%
\section{Discussion}
\subsection{Influence of higher modes}
The character of the observed phase-velocity and amplitude interference patterns for single-station measurements (Fig.~\ref{figot:dETL}) is similar to that of the patterns predicted by the simple interference of the fundamental mode and the first overtone (Fig.~\ref{figot:theoretical}). Tests using measurements on synthetic waveforms calculated using mode summation in a spherically symmetric Earth support this interpretation, with the addition of higher modes beyond the first changing the interference pattern very little.  

To evaluate the extent to which more complicated interactions might be important in a heterogeneous Earth, we perform an experiment in which we measure phase-velocity perturbations on synthetic seismograms calculated numerically in a 3\nobreakdash-D Earth structure. 

The synthetic waveforms are calculated using the radially anisotropic mantle model S362ANI \citep{Kustowskietal2008} with the crustal model CRUST2.0 \citep{CRUST2}, using earthquake source parameters from the Global CMT catalog \cite{Ekstrom2012} and the software package SPECFEM3D Globe \citep{KomatitschTromp2002a,KomatitschTromp2002b}. We take advantage of synthetic waveforms calculated by the Princeton University group, and available through their website \citep{Tromp2010}. We make single-station measurements on the synthetic waveforms using the method of \citet{Ekstrom1997}, as before. The results are shown as residual values with respect to great-circle-path-integrated predictions from the fundamental-mode phase-velocity map derived from the same Earth model, S362ANI with CRUST2.0 (Fig.~\ref{figot:specfempred_short}). 

% 6						
\begin{figure} 
\begin{center}
\includegraphics[scale=1.0]{Ch4Figs/shortdist_synthpredint_logamp} 
\caption[Comparison of contamination in measured synthetic, predicted, and observed single-station phase and amplitude]{(Continued on the following page.)}
\label{figot:specfempred_short}
\end{center}
\end{figure}
\begin{figure}[t]
\contcaption{(Left) Single-station phase-anomaly measurements made on synthetic waveforms calculated from model S362ANI with CRUST2.0, shown with respect to the fundamental-mode phase velocity predicted from the same model (black dots), and predictions of the interference from great-circle integration through the phase-velocity maps for the fundamental and first higher modes from the same model (red dots), calculated as in Equation~\ref{eqot:2}, plotted with epicentral distance. Observed phase anomalies with respect to predictions from GDM52 are shown in blue. (Center) Scatter plot between the phase anomalies measured on the synthetic waveforms and those predicted by great-circle integration through the Earth model. (Right) Amplitude anomalies measured on the synthetic waveforms versus phase anomalies measured on the synthetic waveforms. Each panel lists the earthquake and the corresponding location (color) in Figure~\ref{figot:dETL_sel} for which synthetic waveforms were calculated, except for the Central America earthquake not shown in that figure. }
\end{figure}
%

We compare the measured values with the interference predicted from Equation~\ref{eqot:2}. Fundamental-mode and first-overtone phase velocities are obtained using great-circle-path-integrated predictions from phase-velocity maps derived from the S362ANI--CRUST2.0 model. The relative amplitudes of the fundamental mode and first overtone are adjusted so that the amplitude of the predicted phase difference roughly matches the measured phase difference, with $\frac{b_1 A_1}{A_0} = 0.4$ (Fig.~\ref{figot:specfempred_short}). 

The predicted anomalies match the measured anomalies from the synthetic seismograms well in both the character and the wavelength of the interference pattern. Several example measurements and predictions are shown in Figure~\ref{figot:specfempred_short} for earthquakes with epicentral distances between \degree{10}--\degree{80}. Most of the selected earthquakes correspond to the examples of single-station overtone interference shown for real data in Figure~\ref{figot:dETL_sel}. 

The left column of Figure~\ref{figot:specfempred_short} shows the measured synthetic interference pattern, the predicted interference pattern, and the observed interference pattern, all as a function of distance. Note that the measurements are shown with respect to predictions from the model that best accounts for variations in phase due to structural heterogeneity: model S362ANI with CRUST2.0 for the synthetic measurements, and model GDM52 for the observed measurements. Generally, the synthetic measurements, the predicted values, and the observed measurements are similar in wavelength and amplitude. In some cases, such as the Hawaii event, the predicted peak is horizontally offset from the two measured peaks. Other examples, such as the Aleutian event, show good agreement between the synthetic measurements and the predicted values, but a vertical offset with the observed measurements. The northwest Baja event shows good agreement in the pattern, but a vertical offset between all three data sets. 

The center column (Fig.~\ref{figot:specfempred_short}) contains scatterplots between the measured synthetic interference pattern and the predicted pattern. The trend of the scatter plots is generally parallel to the 1:1 line, indicating the relative excitation of the two modes used is reasonable. Differences in the slope of the scatter, such as in the southeast Baja event, indicate that the measurements might be better predicted using a different relative excitation. Several of the examples, such as the northwest Baja event, show an offset of the synthetic measurements to higher phase velocities in comparison with the predicted values. 

The right column (Fig.~\ref{figot:specfempred_short}) contains scatterplots between the measured phase and amplitude anomalies from the synthetic waveforms. Most of the events show an elliptical pattern, as expected for two quantities with a \degree{90} phase shift. This relationship could be a useful indicator of the presence of overtone contamination. 

\subsection{Relation between single- and multiple-station methods}
Two-station measurements are obtained by differencing single-station phase measurements separated by the inter-station path distance, which in \citet{Fosteretal2014} ranged from 350--750 km or roughly \degree{3.15}--\degree{6.75}. Mini-array measurements are obtained by comparing phase differences within the mini array, equivalent to inter-station distances of \degree{2} or less. In both cases, this is similar to taking the spatial derivative of the single-station phase measurements, and we hypothesize that the differential nature of the measurements is largely responsible for the larger amplitude of the interference signal in the multiple-station measurements. 

We evaluate this interpretation by smoothing the single-station phase deviations by averaging in \degree{1} bins, and then taking the derivative of that empirical function. To approximate the inter-station path lengths used in the two-station measurements, we calculate the derivative by differencing over a \degree{5} width. The result is overlain on the two-station measurements, and compares well with the smoothed two-station phase-velocity deviations (Fig.~\ref{figot:derivatives}). 

% 7						
\begin{figure} 
\begin{center}
\includegraphics[scale=.155]{Ch4Figs/derivatives_new} 
\caption[Single-station phase anomalies and array-measurement phase-velocity anomalies related by the spatial derivative]{(a) Single-station phase-anomaly measurements with respect to predictions from GDM52, as in Figure~\ref{figot:dETL} but plotted as phase perturbations (black dots). Blue line shows the \degree{1}-binned average. (b) Two-station phase-velocity anomalies with respect to predictions from GDM52, as in Figure~\ref{figot:darray}a (black dots). Blue line shows \degree{1}-binned average of the two-station measurements; red line shows numerical derivative of smoothed single-station measurements (blue line in (a)), differenced over \degree{5}. (c) Mini-array phase-velocity anomalies with respect to predictions from \citet{Fosteretal2014}, as in Figure~\ref{figot:darray}b (black dots). Blue line shows \degree{1}-binned average of the mini-array measurements; red line shows numerical derivative of smoothed single-station measurements (blue line in (a)), differenced over \degree{2}. Horizontal axis in all panels is epicentral distance. }
\label{figot:derivatives}
\end{center}
\end{figure}
%

Similarly, for the mini-array measurements, we take the derivative of the smoothed single-station phase deviations, this time differencing over a \degree{2} width. This results in a slightly rougher version of the derivative, which fits well the smoothed mini-array phase-velocity anomalies (Fig.~\ref{figot:derivatives}). For comparison with both the two-station and the mini-array phase-velocity anomalies, we convert from the derivative of the single-station phase anomalies to the desired array-based phase-velocity percentage by multiplication of a factor of $\frac{-c}{\omega}$. Explicitly, the derivative of the phase anomaly can be written
\begin{equation}
\frac{d(\delta\varphi)}{dX} = \frac{d}{dX} \left( \varphi - \varphi_0 \right)  = \frac{d}{dX} \left( \frac{\omega X}{c} - \frac{\omega X}{c_0} \right) = \frac{\omega}{c} - \frac{\omega}{c_0},
\end{equation}
such that the fractional phase-velocity anomaly can be obtained by
\begin{equation}
\frac{-c}{\omega} \cdot \frac{d(\delta\varphi)}{dX} = \frac{c -c_0}{c_0}.
\end{equation}
For simplicity, we approximate this conversion factor using the PREM reference velocity, $\frac{-c_{\mathrm{PREM}}}{\omega}$.

Typically, uncertainties for two-station methods are expected to be smaller than those for single-station methods, because uncertainties associated with the source and with propagation outside of the array should cancel out. \citet{Fosteretal2014} showed that this was true for Rayleigh waves, with two-station uncertainties smaller by a factor of 4--8 at all periods. However, Love wave two-station measurements showed only a small improvement in the uncertainty in that study, by a factor of 2 or less. This relatively poor reduction in the uncertainty can now be attributed to the increased variability in individual measurements due to overtone contamination in two-station Love wave measurements. 

\subsection{Bias in multiple-station measurement data sets}
The observed overtone contamination in multiple-station measurements is an order of magnitude larger than it is for single-station measurements. Such large errors may not average out in the way that single-station interference effects do, particularly given that array-based methods may not have the advantage of the global distribution of stations and earthquakes that single-station studies normally have. A limited spatial or temporal extent for any array could produce observations only in a limited range of epicentral distances, possibly resulting in a large bias. 

For the USArray dataset, which includes observations over a large distance range, the values are roughly balanced between positive and negative anomalies, despite the large amplitude of the oscillations for the array-based measurements. For such a nearly ideal path-length distribution, it is not obvious that any bias would result. However, the mean value of the residuals for the two-station and mini-array measurements is roughly 0.5\% in phase velocity. Comparisons between the regional phase-velocity maps derived from two-station measurements by \citet{Fosteretal2014} and the global phase-velocity maps derived from single-station measurements in GDM52 show a small but consistently positive mean difference at all periods for Love waves, corresponding to a roughly 0.1--0.6\% higher phase velocity in the two-station maps (0.3\% at 100 s). In contrast, for Rayleigh waves, the two-station models show slightly lower mean phase velocities than GDM52 at all periods, by 0.1--0.4\%.  

The differences between the two-station Love wave phase-velocity maps and GDM52 could be indicative of a slight bias to higher phase velocities for array-based Love wave measurements. A similar bias is observed between the two-station phase-velocity maps and the maps derived from mini-array measurements: for Love waves, the mini-array phase velocity maps are on average 0.3--0.9\% faster than the two-station maps at all periods. This difference is not observed for Rayleigh waves, which have average phase-velocity differences ranging between -0.03\% and 0.06\% at all periods. 

The origin of the bias is likely related to the asymmetric shape of the oscillation of the single-station phase residuals. The saw-toothed shape causes more of the distance range to be covered by a gently sloping down-going trend than a steeply sloping up-going trend, as shown in Figure~\ref{figot:derivatives}a. Though the values average to zero for the single-station anomalies, taking the gradient of this pattern and converting to phase-velocity anomalies, as in Figure~\ref{figot:derivatives}b and \ref{figot:derivatives}c, results in more of the distance range being covered by positive values corresponding to gradients of the down-going trends. In theory, this should be offset by fewer, but more negative, values corresponding to the gradients of the steeper up-going trends. In practice, these negative measurements are so far from the average values that quality selection or data reduction using the median reduces their influence, biasing the array-based measurements to higher phase velocities.  These procedures affected both array-based measurement data sets in \citet{Fosteretal2014}: outliers were removed from the two-station data set, with twice as many negatively contaminated measurements as positively contaminated measurements removed; the two-station data set was reduced by taking the median measurement for each inter-station path; and the median mini-array phase-velocity estimate was used at each station, shifting the local phase-velocity maps to higher values. 

Clearly, some action is needed to prevent such errors from biasing array-based Love wave measurements and models, if we are to constrain short-wavelength velocity structure and radial anisotropy. While the observed bias in the final models described above \citep{Fosteretal2014} was found to be small, at less than 1\%, those models were based on large data sets collected on a nearly ideal array. Measurements on smaller or more short-lived arrays, such as PASSCAL deployments, may be subject to larger errors. The simplest solution may be a targeted data-selection approach, designed to produce a flat distribution of measurements across all epicentral distances. Other options might include reducing the effect of overtone contamination in individual measurements through better filtering or more selective use of measurement methods for intermediate-period Love waves, or making some correction to the measurements based on the predicted interference, as in \citet{Forsyth1975}. While we do not know the actual Love wave phase-velocity structure of the Earth for the fundamental and first higher modes, as we do in the synthetic case, the oscillations of the observations about predictions from long-wavelength models indicate that we have reasonable starting estimates. 

Correcting or fitting an interference pattern is likely easiest for a given earthquake when the stations cover a wide path-length range, as in the TA; this would be more difficult for smaller arrays. Amplitude measurements, which also have a large signal from overtone contamination, and a diagnostic relation to the phase contamination signal (Fig.~\ref{figot:specfempred_short}, right column), may be important additional constraints for developing interference corrections. 

%%%%%%%%%%%%%%%%%%%%%%%%%%%%%%%%%%%%%%%%%%%%%%%%%%%%%%%%%%%%%%%%%%%%%%%%%%%%%%%%%%%%
%%%
\section{Conclusions}
Consistent with previous studies, we have shown that errors from overtone interference in single-station phase measurements on USArray data are small, on the order of 1\%, and average out with a sufficiently random distribution of epicentral distances. Amplitude measurements show a clear contamination signal, complementary to the phase interference. For all paths, overtone interference patterns are similar in character, but vary based on the difference between the fundamental-mode and overtone phase velocity for an individual path. The interference pattern is well represented by the interaction of the fundamental- and first-higher-mode waves only.

However, we find that array-based measurements that rely on observation of the spatial gradient of the phase over short distances magnify these errors. We observe errors of up to 20\% of the phase velocity. This is an important effect for regional-scale measurements of Love wave phase speeds, and a high degree of caution is needed when interpreting these measurements. 

Additionally, due to the asymmetric shape of the single-station phase-anomaly interference pattern, array-based measurements may have an overall bias. The original motivation for this study, an observed discrepancy between phase-velocity maps derived from two-station measurements and maps derived from mini-array measurements, can be explained by this effect. Spatial gradients of the single-station interference pattern result in larger numbers of positive phase-velocity anomalies, and fewer, but larger magnitude, negative anomalies. Typical quality-selection approaches disproportionately remove some of these negative values, and selection of a median measurement for a path or location, rather than a mean, exacerbates the problem. In this case, the final model is shifted to higher phase velocities. Future array-based studies of Love wave phase velocities will require careful consideration of overtone interference effects. We note, however, that a better understanding of the character of overtone interference also opens the possibility of improved measurements of fundamental- and higher-mode Love waves. 

%%%%%%%%%%%%%%%%%%%%%%%%%%%%%%%%%%%%%%%%%%%%%%%%%%%%%%%%%%%%%%%%%%%%%%%%%%%%%%%%%%%%
%%%
\section*{Data and Resources}
Waveforms for this study were acquired through the IRIS Data Management Center. Synthetic waveforms for available earthquakes between 2006 and 2012 were acquired through the Princeton data portal Global ShakeMovie \citep{Tromp2010}: global.shakemovie.princeton.edu (last accessed January 2014). Plots were made using the Generic Mapping Tools version 3.4 \citep{GMT}.
%%%%%%%%%%%%%%%%%%%%%%%%%%%%%%%%%%%%%%%%%%%%%%%%%%%%%%%%%%%%%%%%%%%%%%%%%%%%%%%%%%%%
%%%
\section*{Acknowledgments}
We thank everyone involved in the deployment and operation of the USArray Transportable Array for their commitment to collecting high-quality data, and the IRIS DMC for providing simple and consistent access to these data. We thank Celia Eddy for providing amplitude measurements, and the team at Princeton University responsible for making and distributing synthetic waveforms. This work was funded by the National Science Foundation, Award EAR-0952285. 


%%%%%%%%%%%%%%%%%%%%%%%%%%%%%%%% CHAPTER 5/CONCLUSIONS %%%%%%%%%%%%%%%%%%%%%%%%%%%%%%%%%%%%%%%%%%
\singlespacing
\chapter{Concluding Remarks}
\label{ch:conclusions}
\doublespacing

\thispagestyle{fancy}

Taking full advantage of the USArray Transportable Array (TA), we have used the densely spaced, high-quality data to improve models of the regional structure, and to make new observations of arrival-angle anomalies and overtone interference. In doing so, we hope to have provided concrete tools for future use in the field. 

In Chapter~\ref{ch:pv}, we created a large data set of two-station phase measurements from earthquakes recorded on the TA, as well as a data set of mini-array local phase-velocity measurements and arrival-angle estimates. The arrival-angle information is used to correct the Rayleigh wave two-station phase measurements. The two-station measurements are used to produce high-quality phase-velocity maps of the western US for both Love and Rayleigh waves at periods from 25--100 s. These maps can be used to constrain 3\nobreakdash-D structure, investigate radial anisotropy, and gain insight into mantle processes. 

As the first phase-velocity maps derived from TA data using a two-station method, the maps provide an increase in the quality of information on the velocity structure of the western US. The Love wave maps in particular provide new information on the area, as few Love wave studies have been done. The phase-velocity maps are consistent with previous results, and for Rayleigh waves, are consistent with the local phase-velocity estimates. This study used the coverage of the TA through December 2010, and could be updated to include the velocity structure of the eastern US. This would provide an interesting contrast between stable and actively deforming continental structure. 

In Chapter~\ref{ch:aa}, we demonstrated that earthquakes from a common source area have consistent arrival-angle values, but earthquakes separated by more than a few degrees in distance can produce different arrival-angle patterns. This indicates the scale at which structural heterogeneity is important for intermediate-period surface waves. We used the arrival-angle measurements from earthquakes with common source areas to create composite figures showing how waves deviate from the great-circle path as they travel across the array. These figures illustrate the importance of both global structure, which results in arrival-angle anomalies as waves enter the array, and local structure, which results in variations in the arrival-angle-anomaly pattern within the array. 

We have used the arrival-angle composite figures as a benchmark against which we compare predictions from current models. We have demonstrated the utility of using SPECFEM3D Globe to make these model predictions, and shown that, for arrival-angle measurements, the full synthetics are an important improvement over ray-theoretical prediction methods. The comparisons showed that current models can successfully predict long-wavelength arrival-angle patterns but do not contain the structure necessary to predict all of the heterogeneity of the observed patterns. It is interesting to note that the model predictions matched well with the smoothed observations, indicating that long-wavelength structure may be responsible for long-wavelength arrival-angle patterns, and short-wavelength structure may result in short-wavelength patterns, in a somewhat additive manner.

These new observations of arrival-angle anomalies on a large scale, and the discrepancies between the observations and model predictions, open avenues for future research. The data set of arrival-angle measurements can be applied to structural studies, via tomography or backprojection to find specific features. The model predictions can be used to update models and determine what type of heterogeneity is lacking, whether that is additional short-wavelength structure or increased magnitudes of the velocity anomalies already present in the models. SPECFEM predictions can also be used to evaluate models with more detailed regional structure. 

In Chapter~\ref{ch:ot}, we investigated the effect of overtone contamination on fundamental-mode phase-velocity measurements made using array methods. We confirmed that the errors due to overtone interference in Love wave fundamental-mode single-station phase measurements are small, and should not affect global inversions of structure, provided a sufficient path length distribution is used. We presented the theoretical description of the interference using the fundamental mode and first overtone, which explains the majority of the signal. We showed that, because array-based methods typically rely on the gradient of the phase over a small area, the errors from overtone interference in array-based measurements can be much larger than in single-station measurements. Additionally, if preventive techniques or corrections are not used, overtone contamination combined with common quality-selection techniques risk biasing the array-method results to higher phase velocities. 

Using a comparison between single-station measurements on SPECFEM synthetic waveforms and ray-theoretical predictions from fundamental-mode and first-overtone phase velocity maps, we have shown that the overall interference pattern is accounted for by the simple predictions, but additional variations occur in the measurements on the full synthetics. While some of this is likely due to multipathing and scattering, a more in-depth investigation would be useful. Further studies could also determine the extent to which the predicted bias in the array-based measurements carries over into a final model. In particular, we have seen that there is an offset between the mini-array and two-station measurements in Chapter~\ref{ch:pv}, which we attribute to the varying separation distances over which the differential measurements are made in Chapter~\ref{ch:ot}. We have also observed a small offset between the two-station- and single-station-derived models. Additional studies on this topic could determine the best data-selection criteria, measurement methods, and inversion methods to minimize the effects of overtone contamination for Love waves. 
  

\raggedbottom
\pagebreak


%%%%%%%%%%%%%%%%%%%%%%%%%%%%%%%% REFERENCES %%%%%%%%%%%%%%%%%%%%%%%%%%%%%%%%%%%%%%%%%%%%%
%\backmatter
\cleardoublepage
\normalsize
\singlespacing
\bibliographystyle{elsarticle-harv}
\addcontentsline{toc}{chapter}{References}
\bibliography{dissertationbib}
\clearpage

%%%%%%%%%%%%%%%%%%%%%%%%%%%%%%%% APPENDICES %%%%%%%%%%%%%%%%%%%%%%%%%%%%%%%%%%%%%%%%%%%%%
\appendix
\addcontentsline{toc}{chapter}{Appendices}
%%%%%%%%%%%%%%%%%%%%%%%%%%%%%%%% APPENDIX A %%%%%%%%%%%%%%%%%%%%%%%%%%%%%%%%%%%%%%%%%%%%%
\chapter[Phase-velocity maps, all periods]{Phase-velocity maps of the western United States from a two-station method at all periods}
\label{appendixA}
%\addcontentsline{lof}{chapter}{\protect\numberline{\thechapter\quad{}Introduction}}
%\addtocontents{lof}{\textbf{Appendix A: Title}}
%\addtocontents{lot}{\textbf{Appendix A: Title}}
\thispagestyle{fancy}
\doublespacing

\begin{raggedright}
This appendix contains supplementary material for Chapter~\ref{ch:pv}: ``Surface-wave phase velocities of the western United States from a two-station method''. 
\end{raggedright}
\vspace{1cm}

Included are phase-velocity maps for all available periods (25, 27, 30, 32, 35, 40, 45, 50, 60, 75, 100 s). Rayleigh wave maps are derived from two-station measurements corrected for the arrival angle (Figs.~\ref{figA:R_S}, \ref{figA:R_M}, \ref{figA:R_L}); Love wave maps are derived from baseline two-station measurements (Figs.~\ref{figA:L_S}, \ref{figA:L_M}, \ref{figA:L_L}). 

% 1						
\begin{figure} 
\begin{center}
\includegraphics[scale=1.0]{AppendixAFigs/Rayl_wt01wcqAsA_all_S} 
\caption[Rayleigh wave phase-velocity maps: 25-, 27-, 30-, and 32-s]{Rayleigh wave phase-velocity models from two-station phase measurements corrected for arrival angle at 25-, 27-, 30-, and 32-s period. The model includes the area with path coverage, expanded by one pixel in each direction. Values are given as a percent deviation with respect to the model mean at each period.}
\label{figA:R_S}
\end{center}
\end{figure}
%
% 2						
\begin{figure} 
\begin{center}
\includegraphics[scale=1.0]{AppendixAFigs/Rayl_wt01wcqAsA_all_M} 
\caption[Rayleigh wave phase-velocity maps: 35-, 40-, 45-, and 50-s]{Rayleigh wave phase-velocity models from two-station phase measurements corrected for arrival angle at 35-, 40-, 45-, and 50-s period. The model includes the area with path coverage, expanded by one pixel in each direction. Values are given as a percent deviation with respect to the model mean at each period.}
\label{figA:R_M}
\end{center}
\end{figure}
%
% 3						
\begin{figure} 
\begin{center}
\includegraphics[scale=1.0]{AppendixAFigs/Rayl_wt01wcqAsA_all_L} 
\caption[Rayleigh wave phase-velocity maps: 60-, 75-, and 100-s]{Rayleigh wave phase-velocity models from two-station phase measurements corrected for arrival angle at 60-, 75-, and 100-s period. The model includes the area with path coverage, expanded by one pixel in each direction. Values are given as a percent deviation with respect to the model mean at each period.}
\label{figA:R_L}
\end{center}
\end{figure}
%
% 4						
\begin{figure} 
\begin{center}
\includegraphics[scale=1.0]{AppendixAFigs/Love_wt01ncqAsA_all_S} 
\caption[Love wave phase-velocity maps: 25-, 27-, 30-, and 32-s]{Love wave phase-velocity models from baseline two-station phase measurements at 25-, 27-, 30-, and 32-s period. The model includes the area with path coverage, expanded by one pixel in each direction. Values are given as a percent deviation with respect to the model mean at each period.}
\label{figA:L_S}
\end{center}
\end{figure}
%
% 5						
\begin{figure} 
\begin{center}
\includegraphics[scale=1.0]{AppendixAFigs/Love_wt01ncqAsA_all_M} 
\caption[Love wave phase-velocity maps: 35-, 40-, 45-, and 50-s]{Love wave phase-velocity models from baseline two-station phase measurements at 35-, 40-, 45-, and 50-s period. The model includes the area with path coverage, expanded by one pixel in each direction. Values are given as a percent deviation with respect to the model mean at each period.}
\label{figA:L_M}
\end{center}
\end{figure}
%
% 6						
\begin{figure} 
\begin{center}
\includegraphics[scale=1.0]{AppendixAFigs/Love_wt01ncqAsA_all_L} 
\caption[Love wave phase-velocity maps: 60-, 75-, and 100-s]{Love wave phase-velocity models from baseline two-station phase measurements at 60-, 75-, and 100-s period. The model includes the area with path coverage, expanded by one pixel in each direction. Values are given as a percent deviation with respect to the model mean at each period.}
\label{figA:L_L}
\end{center}
\end{figure}
%

%%%%%%%%%%%%%%%%%%%%%%%%%%%%%%%% APPENDIX B %%%%%%%%%%%%%%%%%%%%%%%%%%%%%%%%%%%%%%%%%%%%%
\singlespacing
\chapter[Composite arrival-angle observations and predictions]{Composite arrival-angle observations and predictions}
\label{appendixB}
%\addtocontents{lof}{\textbf{Appendix B: Title}}
%\addtocontents{lot}{\textbf{Appendix B: Title}}
\doublespacing

\begin{raggedright}
This appendix contains supplementary material for Chapter~\ref{ch:aa}: ``Arrival-angle anomalies across the USArray Transportable Array''. 
\end{raggedright}
%\vspace{1cm}

\subsubsection*{Mini-array method}
Figure~\ref{figB:aa_radius} demonstrates the effect of varying the mini-array radius on arrival-angle estimates. In this dissertation, a radius of \degree{1} was used for all Rayleigh wave measurements, and a radius of \degree{2} was used for all Love wave measurements. 

% 1						
\begin{figure} 
\begin{center}
\includegraphics[scale=1]{AppendixBFigs/compare_radii_approx_ellipses} 
\caption[Effect of mini-array size on arrival-angle estimates]{Demonstration of the effect of varying mini-array sizes on arrival-angle estimates, here shown for Rayleigh waves at 50 s period for an event located in the southwest Pacific. Black ellipse shows mini-array size, for a radius of \degree{0.5}, \degree{1.0}, \degree{2.0}, and \degree{3.0}, left to right. Black lines show estimated propagation direction.  }
\label{figB:aa_radius}
\end{center}
\end{figure}
%

\subsubsection*{Observations}
Selected examples of composite figures were presented in Chapter~\ref{ch:aa}. All 10 composite figures are presented here, including source area locations (Fig.~\ref{figB:largemap}), constituent event locations and focal mechanisms (Figs.~\ref{figB:zoommap1}, \ref{figB:zoommap2}), and composite arrival-angle observations for 50-s Rayleigh waves (Figs.~\ref{figB:obs1}, \ref{figB:obs2}). Detailed information on the constituent events, including location, magnitude, and number of stations for which measurements were made, is included in Table~\ref{tableB:composites}. 

\subsubsection*{Comparison with predictions}
We make additional comparisons between observed and predicted arrival angles to those found in the text. The long-wavelength arrival-angle patterns for all 10 composite figures are obtained by smoothing the observed anomalies with a Gaussian function with a full width at half maximum of \degree{4.7}. These are compared with ray-tracing predictions \citep{Larson1998} from models GDM52 \citep{Ekstrom2011} and S362ANI \citep{Kustowskietal2008} (Figs.~\ref{figB:pred_long1}, \ref{figB:pred_long2}). 

The observed short-wavelength arrival-angle patterns are compared with measured arrival-angle anomalies from phase values obtained in three ways (Figs.~\ref{figB:pred_short1}, \ref{figB:pred_short2}). First, we use a 50-s Rayleigh wave phase-velocity map derived from the radially anisotropic 3\nobreakdash-D shear-velocity model of \citet{Nettles&Dziewonski2008}, combined with the model CRUST2.0 \citep{CRUST2}. Sediments are accounted for by including sediment thicker than 2~km. The shear-velocity model was derived using a large data set of both global and regional single-station phase anomalies, and has lateral resolution of a few hundred kilometers within North America and a few thousand kilometers globally. We define the resulting phase-velocity map on \degree{2}-by-\degree{2} pixels, retaining the full heterogeneity of CRUST2.0. We make phase predictions from this model by integrating through the phase-velocity map along the great-circle path, and measure the arrival-angle using the mini-array method (Figs.~\ref{figB:pred_short1}, \ref{figB:pred_short2}, column 2). 

Second, we predict the phase using great-circle-path integration through the phase-velocity map derived from S362ANI with CRUST2.0, also defined on \degree{2}-by-\degree{2} pixels. Arrival-angle estimates are made on the phase values using the mini-array method (Figs.~\ref{figB:pred_short1}, \ref{figB:pred_short2}, column 3). Lastly, we calculate synthetic waveforms using SPECFEM3D Globe \citep{KomatitschTromp2002a,KomatitschTromp2002b} and model S362ANI with CRUST2.0. We make single-station phase measurements on the synthetic waveforms using the method of \citet{Ekstrom1997}, and make arrival-angle estimates using the mini-array method, as with the real data (Figs.~\ref{figB:pred_short1}, \ref{figB:pred_short2}, column 4). As noted in Chapter~\ref{ch:aa}, the latter two models are comparable in their velocity structure, and the use of great-circle-path integration with the highly heterogeneous pixel-based models may not be appropriate, and is used solely as an illustration of the possible effects of small-scale structure on arrival-angle anomaly patterns. 

% 2						
\begin{figure} 
\begin{center}
\includegraphics[scale=1]{AppendixBFigs/R050_allevents_map} 
\caption[Source area locations for composite figures]{Location map showing all possible USArray Transportable Array stations through December 2010 (black triangles) and composite figure source areas (colored circles). From west to east, source areas are: Solomon Islands 1 (dark green), Solomon Islands 2 (light green), Santa Cruz Islands (dark blue), Vanuatu Islands (light blue), Loyalty Islands (red), Samoa Islands (pink), Tonga Islands 1 (orange), Tonga Islands 2 (black), Easter Island (purple), and Ascension Island (yellow). Black lines show approximate great-circle path from source area to array. }
\label{figB:largemap}
\end{center}
\end{figure}
%
% 3						
\begin{figure} 
\begin{center}
\includegraphics[scale=1]{AppendixBFigs/locations1} 
\caption[Constituent event locations and focal mechanisms for composite figures]{Maps showing location and focal mechanism for all constituent events in each composite figure. Colored focal mechanism indicates the central event used in the calculation of synthetics. Colors correspond to locations in Figure~\ref{figB:largemap}. Source areas are: (a) Solomon Islands 1, (b) Solomon Islands 2, (c) Santa Cruz Islands, (d) Vanuatu Islands, and (e) Loyalty Islands. }
\label{figB:zoommap1}
\end{center}
\end{figure}
%
% 4						
\begin{figure} 
\begin{center}
\includegraphics[scale=1]{AppendixBFigs/locations2} 
\caption[Constituent event locations and focal mechanisms for composite figures, continued]{Maps showing location and focal mechanism for all constituent events in each composite figure. Colored focal mechanism indicates the central event used in the calculation of synthetics. Colors correspond to locations in Figure~\ref{figB:largemap}. Source areas are: (f) Samoa Islands, (g) Tonga Islands 1, (h) Tonga Islands 2, (i) Easter Island, and (j) Ascension Island. }
\label{figB:zoommap2}
\end{center}
\end{figure}
%
% 5						
\begin{figure} 
\begin{center}
\includegraphics[scale=1]{AppendixBFigs/obs1} 
\caption[Composite arrival-angle anomaly observations]{Composite figures showing observed arrival-angle anomalies. Black lines indicate great-circle path propagation direction. Source areas correspond to Figure~\ref{figB:zoommap1}: (a) Solomon Islands 1, (b) Solomon Islands 2, (c) Santa Cruz Islands, (d) Vanuatu Islands, and (e) Loyalty Islands. }
\label{figB:obs1}
\end{center}
\end{figure}
%
% 6						
\begin{figure} 
\begin{center}
\includegraphics[scale=1]{AppendixBFigs/obs2} 
\caption[Composite arrival-angle anomaly observations, continued]{Composite figures showing observed arrival-angle anomalies. Black lines indicate great-circle path propagation direction. Source areas correspond to Figure~\ref{figB:zoommap2}: (f) Samoa Islands, (g) Tonga Islands 1, (h) Tonga Islands 2, (i) Easter Island, and (j) Ascension Island. }
\label{figB:obs2}
\end{center}
\end{figure}
%
% 7					
\begin{figure} 
\begin{center}
\includegraphics[scale=1]{AppendixBFigs/predictions_longwavelength_1} 
\caption[Predictions of long-wavelength arrival-angle anomalies]{Comparison of observed arrival-angle anomalies with model predictions. Left to right: composite figures showing observed arrival-angle anomalies; smoothed composite figures, filtered by a Gaussian function with a full width at half maximum of \degree{4.7}; predictions of arrival-angle anomalies via ray tracing from model GDM52; predictions of arrival-angle anomalies via ray tracing from model S362ANI. Listed source areas correspond to Figure~\ref{figB:zoommap1}. }
\label{figB:pred_long1}
\end{center}
\end{figure}
%
% 8					
\begin{figure} 
\begin{center}
\includegraphics[scale=1]{AppendixBFigs/predictions_longwavelength_2} 
\caption[Predictions of long-wavelength arrival-angle anomalies, continued]{Comparison of observed arrival-angle anomalies with model predictions. Left to right: composite figures showing observed arrival-angle anomalies; smoothed composite figures, filtered by a Gaussian function with a full width at half maximum of \degree{4.7}; predictions of arrival-angle anomalies via ray tracing from model GDM52; predictions of arrival-angle anomalies via ray tracing from model S362ANI. Listed source areas correspond to Figure~\ref{figB:zoommap2}. }
\label{figB:pred_long2}
\end{center}
\end{figure}
%
% 9						
\begin{figure} 
\begin{center}
\includegraphics[scale=1]{AppendixBFigs/predictions_shortwavelength_1_epix} 
\caption[Predictions of short-wavelength arrival-angle anomalies]{Comparison of observed arrival-angle anomalies with model predictions. Left to right: composite figures showing observed arrival-angle anomalies; predictions of arrival-angle anomalies via great-circle path integration from phase-velocity maps derived from \citet{Nettles&Dziewonski2008} with CRUST2.0; predictions of arrival-angle anomalies via great-circle path integration from phase-velocity maps derived from S362ANI with CRUST2.0; measured arrival-angle anomalies from synthetic waveforms calculated using SPECFEM with S362ANI and CRUST2.0. Listed source areas correspond to Figure~\ref{figB:zoommap1}. }
\label{figB:pred_short1}
\end{center}
\end{figure}
%
% 10						
\begin{figure} 
\begin{center}
\includegraphics[scale=1]{AppendixBFigs/predictions_shortwavelength_2_epix} 
\caption[Predictions of short-wavelength arrival-angle anomalies, continued]{Comparison of observed arrival-angle anomalies with model predictions. Left to right: composite figures showing observed arrival-angle anomalies; predictions of arrival-angle anomalies via great-circle path integration from phase-velocity maps derived from \citet{Nettles&Dziewonski2008} with CRUST2.0; predictions of arrival-angle anomalies via great-circle path integration from phase-velocity maps derived from S362ANI with CRUST2.0; measured arrival-angle anomalies from synthetic waveforms calculated using SPECFEM with S362ANI and CRUST2.0. Listed source areas correspond to Figure~\ref{figB:zoommap2}. }
\label{figB:pred_short2}
\end{center}
\end{figure}
%

%%%%%%%%%%%%%%%%%%%%%%%%%%%%%%%%%%%%%%%%%%%%%%%%%%%%%%%%%%%%%%%%
\clearpage
\singlespacing
\footnotesize
\begin{longtable}{ c c c c c c c }
\caption{Constituent event details for composite figures.}\label{tableB:composites} \\
\hline
Event &\multicolumn{3}{c}{Location$^a$} &Magnitude &Dist. from &No. of arrival- \\
			&Lat. &Lon. &Depth 			&(Mw) 		&central event$^b$ &angle meas. \\
\hline 
\endfirsthead
\multicolumn{7}{l}{\tablename\ \thetable\ -- \textit{Continued from previous page}} \\
\hline
Event &\multicolumn{3}{c}{Location$^a$} &Magnitude &Dist. from &No. of arrival- \\
			&Lat. &Lon. &Depth 			&(Mw)		&central event$^b$ &angle meas. \\
\hline 
\endhead
\hline 
\multicolumn{7}{r}{\textit{Continued on next page}} \\
\endfoot
\hline
\endlastfoot
\multicolumn{3}{l}{\textbf{Solomon Islands 1 (dark green$^c$)}} & & & &\textbf{1030} \\
200702170743A &-7.560 &155.920 &24.5 &5.7 &0.722 &100 \\
200702171243A &-7.580 &155.850 &18.5 &5.6 &0.675 &270 \\
200704031204A &-8.080 &155.800 &12.0 &6.1 &1.022 &272 \\
200704031527A &-7.510 &155.480 &12.0 &4.8 &0.373 &263 \\
200704031940A &-7.250 &155.280 &23.4 &5.6 &0.072 &211 \\
200704032217A &-7.030 &154.910 &24.1 &4.9 &0.415 &238 \\
200705010015A &-7.270 &155.020 &14.4 &5.6 &0.292 &228 \\
200705010145A &-7.510 &155.140 &12.1 &5.8 &0.364 &325 \\
\textbf{200705101115A} &-7.180 &155.300 &32.5 &0.0 &0.000 &215 \\
200706160424A &-7.520 &155.490 &12.1 &5.8 &0.387 &346 \\
200706280252A &-7.950 &154.700 &18.0 &6.7 &0.969 &31 \\
200711012145A &-6.880 &154.760 &28.0 &5.6 &0.613 &319 \\
200805031901A &-6.960 &155.080 &38.0 &5.8 &0.309 &228 \\
200805241324A &-7.390 &156.020 &40.8 &5.9 &0.744 &307 \\
200904160043A &-6.780 &154.170 &13.1 &5.8 &1.190 &407 \\
200908181759A &-6.850 &154.640 &15.8 &5.6 &0.733 &174 \\
201002100718A &-7.320 &154.890 &12.0 &5.5 &0.430 &220 \\
201007070650A &-6.530 &154.420 &32.4 &5.9 &1.087 &346 \\
201008201756A &-6.840 &154.230 &12.9 &6.1 &1.115 &407 \\ \hline 
\multicolumn{3}{l}{\textbf{Solomon Islands 2 (light green)}} & & & &\textbf{913} \\
200704021202A &-8.910 &157.700 &12.0 &6.2 &0.323 &316 \\
200704022320A &-8.880 &157.550 &12.0 &6.2 &0.258 &316 \\
200705240107A &-9.320 &157.420 &18.6 &5.5 &0.201 &226 \\
200806220722A &-9.030 &157.860 &14.4 &5.8 &0.415 &386 \\
200809091222A &-9.420 &158.360 &18.5 &5.9 &0.946 &410 \\
200910092250A &-9.130 &157.900 &13.9 &5.6 &0.444 &346 \\
\textbf{201001032148A} &-9.120 &157.450 &12.0 &6.6 &0.000 &433 \\
201001032236A &-8.880 &157.210 &12.0 &7.1 &0.336 &435 \\
201001041128A &-8.620 &157.230 &12.6 &5.8 &0.542 &423 \\
201001051215A &-9.110 &157.770 &12.0 &6.8 &0.316 &428 \\
201001090551A &-9.260 &157.820 &12.0 &6.2 &0.391 &431 \\ \hline 
\multicolumn{3}{l}{\textbf{Santa Cruz Islands (dark blue)}} & & & &\textbf{985} \\
200608281424A &-10.730 &164.940 &12.0 &5.7 &1.708 &185 \\
200709020105A &-11.740 &165.680 &18.3 &7.3 &0.469 &404 \\
200807190927A &-11.120 &164.470 &13.5 &6.6 &1.764 &360 \\
200807191101A &-11.180 &164.450 &12.0 &6.2 &1.747 &228 \\
200809192249A &-11.230 &164.480 &12.0 &5.6 &1.695 &287 \\
200910080828A &-13.140 &166.090 &13.8 &6.8 &1.013 &412 \\
200910090914A &-11.790 &165.820 &18.0 &5.6 &0.361 &312 \\
200910231514A &-12.210 &165.920 &17.6 &6.0 &0.085 &359 \\
\textbf{200911140950A} &-12.130 &165.950 &19.5 &5.6 &0.000 &346 \\
200911231836A &-12.500 &166.080 &25.1 &5.7 &0.389 &284 \\
201001181609A &-12.570 &166.060 &21.0 &5.7 &0.450 &308 \\
201004110219A &-12.990 &166.240 &23.2 &5.8 &0.900 &355 \\
201007020604A &-13.710 &166.440 &28.0 &6.3 &1.641 &387 \\ \hline 
\multicolumn{3}{l}{\textbf{Vanuatu Islands (light blue)}} & & & &\textbf{1107} \\
200707150927A &-15.400 &168.660 &12.0 &6.1 &2.146 &361 \\
200803121123A &-16.590 &167.220 &12.0 &6.4 &0.830 &400 \\
200803121136A &-16.590 &167.170 &12.2 &6.3 &0.854 &397 \\
200806100413A &-18.190 &167.790 &20.9 &5.9 &0.877 &382 \\
200811271731A &-17.990 &167.480 &13.0 &5.5 &0.684 &260 \\
200905290620A &-16.900 &168.420 &12.0 &5.7 &0.855 &396 \\
200906020217A &-17.730 &167.640 &19.8 &6.3 &0.408 &407 \\
200906120944A &-17.520 &167.630 &19.5 &6.0 &0.199 &395 \\
201006011647A &-17.720 &169.190 &12.0 &5.6 &1.532 &389 \\
201006092323A &-18.410 &169.590 &12.4 &6.0 &2.150 &413 \\
201007220504A &-15.150 &168.260 &12.0 &6.1 &2.238 &384 \\
\textbf{201008110335A} &-17.320 &167.640 &18.0 &5.8 &0.000 &376 \\
201012290654A &-19.650 &168.100 &12.0 &6.4 &2.358 &421 \\ \hline 
\multicolumn{3}{l}{\textbf{Loyalty Islands (red)}} & & & &\textbf{1108} \\
200703312155A &-20.540 &168.830 &24.0 &5.7 &0.157 &198 \\
200704032026A &-20.810 &168.740 &16.5 &6.2 &0.343 &317 \\
200709271957A &-21.200 &169.210 &14.0 &6.3 &0.879 &404 \\
200709280101A &-21.380 &169.200 &19.3 &6.4 &1.028 &409 \\
200709280135A &-21.320 &169.170 &21.0 &6.6 &0.962 &406 \\
200709280821A &-21.240 &169.300 &12.0 &5.2 &0.961 &349 \\
200710131745A &-21.180 &169.220 &12.9 &6.1 &0.868 &392 \\
200804072254A &-20.060 &168.460 &20.8 &5.8 &0.457 &351 \\
200804091113A &-20.330 &168.620 &21.0 &6.4 &0.150 &240 \\
200804091123A &-20.260 &168.630 &31.7 &6.3 &0.214 &346 \\
200804100110A &-20.440 &168.710 &27.2 &5.8 &0.041 &334 \\
\textbf{200804111745A} &-20.470 &168.680 &21.6 &6.0 &0.000 &382 \\
200804111636A &-20.280 &168.810 &15.1 &5.9 &0.225 &372 \\
200804111745A &-20.470 &168.680 &21.6 &6.0 &0.000 &382 \\
200804190558A &-20.360 &168.720 &12.0 &6.3 &0.116 &383 \\
200804240119A &-20.220 &168.580 &19.2 &5.7 &0.266 &309 \\
200804282026A &-20.230 &168.810 &17.1 &6.1 &0.268 &401 \\
200811181403A &-18.900 &169.660 &12.0 &5.5 &1.814 &366 \\
200904210053A &-19.720 &170.040 &12.0 &5.8 &1.480 &405 \\
200910110312A &-22.050 &169.990 &24.5 &6.0 &1.991 &390 \\
201001290919A &-18.840 &169.700 &12.0 &5.8 &1.885 &415 \\
201009080801A &-20.440 &170.010 &12.9 &5.6 &1.248 &339 \\
201009081137A &-20.520 &170.060 &14.1 &6.3 &1.295 &401 \\
201012271940A &-19.300 &167.820 &12.0 &5.5 &1.418 &190 \\
201012290654A &-19.650 &168.100 &12.0 &6.4 &0.981 &421 \\ \hline 
\multicolumn{3}{l}{\textbf{Samoa Islands (pink)}} & & & &\textbf{1042} \\
200602260418A &-15.200 &-175.920 &12.9 &5.9 &0.590 &111 \\
200603051712A &-15.430 &-174.420 &18.1 &5.4 &0.881 &78 \\
200603171946A &-15.190 &-175.120 &14.0 &5.3 &0.188 &98 \\
200709140546A &-15.210 &-176.020 &15.0 &5.7 &0.686 &274 \\
200801220755A &-15.250 &-175.170 &14.6 &6.0 &0.137 &179 \\
\textbf{200801221049A} &-15.230 &-175.310 &13.6 &6.1 &0.000 &254 \\
200809010706A &-15.190 &-176.120 &12.0 &5.7 &0.783 &402 \\
200810092308A &-15.150 &-173.380 &12.0 &0.0 &1.865 &129 \\
200901300347A &-15.140 &-174.450 &18.0 &5.7 &0.835 &379 \\
200904142329A &-16.250 &-177.520 &12.0 &5.9 &2.357 &63 \\
200910141800A &-14.900 &-174.610 &12.7 &6.3 &0.752 &411 \\
201001131621A &-15.440 &-174.740 &13.3 &5.5 &0.588 &321 \\
201007171620A &-15.450 &-175.160 &12.0 &5.6 &0.262 &145 \\
201009071249A &-14.370 &-175.940 &15.4 &5.5 &1.050 &317 \\
201012182224A &-14.900 &-173.440 &12.0 &5.7 &1.836 &240 \\ \hline 
\multicolumn{3}{l}{\textbf{Tonga Islands 1 (orange)}} & & & &\textbf{1063} \\
200602161454A &-16.000 &-172.460 &15.0 &5.8 &0.863 &90 \\
200604301508A &-15.480 &-172.600 &17.7 &5.5 &0.348 &85 \\
200604301933A &-15.540 &-172.580 &18.5 &5.9 &0.407 &67 \\
200704111241A &-14.900 &-173.210 &26.9 &0.0 &0.479 &167 \\
200711021240A &-15.280 &-172.660 &15.1 &5.7 &0.178 &359 \\
200711021330A &-15.240 &-172.760 &17.2 &5.8 &0.074 &319 \\
200712131551A &-15.240 &-171.970 &21.3 &6.1 &0.831 &388 \\
200808191337A &-14.830 &-173.170 &13.4 &5.7 &0.501 &354 \\
200808191630A &-14.850 &-173.230 &12.0 &6.1 &0.527 &383 \\
200903060701A &-15.040 &-172.950 &15.0 &5.8 &0.205 &348 \\
200908301451A &-15.120 &-172.280 &18.3 &6.6 &0.539 &389 \\
200910010613A &-15.110 &-172.890 &22.2 &5.8 &0.115 &388 \\
200910101941A &-15.550 &-172.590 &17.9 &5.9 &0.410 &413 \\
200910190741A &-15.550 &-172.730 &12.0 &5.8 &0.352 &398 \\
200910192249A &-15.370 &-171.960 &32.5 &6.0 &0.855 &295 \\
201002090103A &-15.000 &-173.070 &15.0 &6.0 &0.312 &396 \\
\textbf{201004211720A} &-15.210 &-172.830 &45.3 &6.1 &0.000 &391 \\
201007250339A &-14.840 &-173.260 &13.1 &5.9 &0.555 &390 \\
201012182224A &-14.900 &-173.440 &12.0 &5.7 &0.665 &240 \\ \hline 
\multicolumn{3}{l}{\textbf{Tonga Islands 2 (black)}} & & & &\textbf{1060} \\
200605050616B &-19.870 &-174.040 &12.0 &5.9 &1.078 &148 \\
200605170306A &-20.740 &-173.390 &12.0 &6.0 &0.185 &53 \\
200610081350A &-23.690 &-174.920 &12.3 &5.8 &3.098 &198 \\
200611301133A &-21.580 &-173.600 &12.0 &6.1 &0.728 &230 \\
200711202352A &-17.860 &-172.790 &34.7 &5.7 &3.060 &246 \\
200804151724A &-18.590 &-175.600 &12.8 &5.7 &2.964 &349 \\
200804160035A &-18.450 &-175.580 &12.0 &6.3 &3.059 &387 \\
200810091750A &-21.220 &-173.810 &18.2 &5.9 &0.441 &351 \\
200904061216A &-22.500 &-174.190 &12.0 &5.5 &1.746 &363 \\
200905260049A &-21.260 &-175.580 &12.0 &0.0 &1.939 &383 \\
\textbf{200911241247A} &-20.850 &-173.550 &25.9 &6.8 &0.000 &436 \\
201010121202A &-20.690 &-173.250 &17.3 &5.9 &0.323 &195 \\
201010162008A &-20.520 &-173.190 &12.0 &5.9 &0.471 &213 \\
201011032334A &-20.590 &-173.660 &20.3 &6.1 &0.278 &356 \\ \hline 
\multicolumn{3}{l}{\textbf{Easter Island (purple)}} & & & &\textbf{838} \\
200605240032A &-27.680 &-113.000 &16.0 &5.6 &1.631 &139 \\
200611302338A &-29.570 &-112.340 &14.4 &5.6 &0.340 &233 \\
200612312248A &-28.850 &-112.520 &12.0 &5.6 &0.395 &55 \\
200702140129A &-29.740 &-112.170 &17.1 &5.9 &0.551 &293 \\
200805312316A &-28.900 &-112.400 &12.0 &5.8 &0.341 &249 \\
200808191058A &-28.650 &-112.790 &12.8 &5.7 &0.663 &396 \\
200909101946A &-30.110 &-111.780 &12.0 &5.5 &1.040 &378 \\
\textbf{200909172321A} &-29.240 &-112.440 &12.7 &6.2 &0.000 &379 \\ \hline 
\multicolumn{3}{l}{\textbf{Ascension Island (yellow)}} & & & &\textbf{607} \\
200602171324A &-1.160 &-15.130 &18.8 &5.1 &1.368 &47 \\
200705041206A &-1.120 &-14.920 &12.7 &6.2 &1.155 &299 \\
\textbf{200809051907A} &-0.880 &-13.790 &12.0 &5.8 &0.000 &334 \\
200811221849A &-0.770 &-13.800 &17.8 &6.3 &0.110 &340 \\ \hline 
\end{longtable}
\normalsize
\begin{flushleft}
$a$ Centroid location and magnitude from the Global CMT Catalog \citep{Dziewonski1981,Ekstrom2012}.\\
$b$ Central event used in the calculation of synthetics for each composite event is marked in bold. \\
$c$ Letter corresponds to location maps (Figs.~\ref{figB:largemap}, \ref{figB:zoommap1}, \ref{figB:zoommap2}), observations (Figs.~\ref{figB:obs1},\ref{figB:obs2}), and predictions (Figs.~\ref{figB:pred_long1}, \ref{figB:pred_long2}, \ref{figB:pred_short1}, \ref{figB:pred_short2}). \\

\end{flushleft}
\clearpage

%%%%%%%%%%%%%%%%%%%%%%%%%%%%%%%% REFERENCES FOR APPENDICES %%%%%%%%%%%%%%%%%%%%%%%%%%%%%%%%%%%%%%%%%%%%%
%\backmatter
\normalsize
\singlespacing
\begin{thebibliography}{43}
\expandafter\ifx\csname natexlab\endcsname\relax\def\natexlab#1{#1}\fi
\expandafter\ifx\csname url\endcsname\relax
  \def\url#1{\texttt{#1}}\fi
\expandafter\ifx\csname urlprefix\endcsname\relax\def\urlprefix{URL }\fi

\bibitem[{Bassin et~al.(2000)Bassin, Laske, and Masters}]{CRUST2}
Bassin, C., Laske, G., Masters, G., 2000. {The current limits of resolution for
  surface wave tomography in North America}. Eos Trans. AGU 81~(F897).

\bibitem[{Dziewonski et~al.(1981)Dziewonski, Chou, and
  Woodhouse}]{Dziewonski1981}
Dziewonski, A.~M., Chou, T.-A., Woodhouse, J.~H., 1981. {Determination of
  earthquake source parameters from waveform data for studies of global and
  regional seismicity}. J. Geophys. Res. 86, 2825--2852.

\bibitem[{Ekstr\"{o}m et~al.(1997)Ekstr\"{o}m, Tromp, and Larson}]{Ekstrom1997}
Ekstr\"{o}m, G., Tromp, J., Larson, E. W.~F., 1997. {Measurements and global
  models of surface wave propagation}. J. Geophys. Res. 102~(B4), 8137--8157.

\bibitem[{Ekstr\"{o}m(2011)}]{Ekstrom2011}
Ekstr\"{o}m, G., 2011. {A global model of Love and Rayleigh surface wave
  dispersion and anisotropy, 25--250 s}. Geophys. J. Int. 187, 1668--1686.

\bibitem[{Ekstr\"{o}m et~al.(2012)Ekstr\"{o}m, Nettles, and
  Dziewonski}]{Ekstrom2012}
Ekstr\"{o}m, G., Nettles, M., Dziewonski, A.~M., 2012. {The global CMT
  project 2004--2010: Centroid-moment tensors for 13,017 earthquakes}. Phys.
  Earth Planet. Inter. 200--201, 1--9.
  
\bibitem[{Komatitsch and Tromp(2002{\natexlab{a}})}]{KomatitschTromp2002a}
Komatitsch, D., Tromp, J., 2002{\natexlab{a}}. {Spectral-element simulations of
  global seismic wave propagation-I. Validation}. Geophys. J. Int. 149~(2),
  390--412.

\bibitem[{Komatitsch and Tromp(2002{\natexlab{b}})}]{KomatitschTromp2002b}
Komatitsch, D., Tromp, J., 2002{\natexlab{b}}. {Spectral-element simulations of
  global seismic wave propagation-II. 3-D models, oceans, rotation, and
  self-gravitation}. Geophys. J. Int. 150~(1), 303--318.

\bibitem[{Kustowski et~al.(2008)Kustowski, Ekstr\"{o}m, and
  Dziewo\'{n}ski}]{Kustowskietal2008}
Kustowski, B., Ekstr\"{o}m, G., Dziewo\'{n}ski, A.~M., 2008. {Anisotropic
  shear-wave velocity structure of the Earth's mantle: A global model}. J.
  Geophys. Res. 113~(B06306), doi: 10.1029/2007JB005169.

\bibitem[{Larson et~al.(1998)Larson, Tromp, and Ekstr\"{o}m}]{Larson1998}
Larson, E. W.~F., Tromp, J., Ekstr\"{o}m, G., 1998. {Effects of slight
  anisotropy on surface waves}. Geophys. J. Int. 132, 654--666.

\bibitem[{Nettles and Dziewonski(2008)}]{Nettles&Dziewonski2008}
Nettles, M., Dziewo\'{n}ski, Adam M., 2008. {Radially anisotropic shear velocity structure of the upper mantle globally and beneath North America}. J. Geophys. Res. 113~(B02303).

\end{thebibliography}

\end{document}  